\subsection{Circular Orbit: ISS Configuration}
\label{subsec:iss_orbit}

[STUB: Demonstrates natural language → physics calculation → VR visualization pipeline]

\subsubsection{User Interaction}

[STUB: Voice transcript: "Create an orbit matching the ISS"]

[STUB: Figure - Screenshot showing ISS circular orbit rendered as cyan 3D curve around Earth]

\subsubsection{System Execution}

[STUB: Agent interpretation - selects create\_circular\_orbit tool]

[STUB: Physics calculation - altitude 420 km → velocity 7.66 km/s via vis-viva equation]

[STUB: Scale compression - rendered at 0.33 Unity units from surface (k=0.000785)]

[STUB: Agent response synthesis - educational explanation with ISS orbital parameters]

\subsubsection{Capabilities Demonstrated}

\begin{itemize}
    \item Natural language interpretation (Objective \#2)
    \item Tool-calling architecture (Section~\ref{subsec:agent_implementation})
    \item Vis-viva equation implementation (Appendix~\ref{app:physics_implementation})
    \item VR trajectory visualization (Section~\ref{subsec:vr_implementation})
\end{itemize}

\subsection{Elliptical Orbit: Exploring Eccentricity}
\label{subsec:elliptical_orbit}

[STUB: Demonstrates vague request interpretation, eccentricity visualization, geometric understanding]

\subsubsection{User Interaction}

[STUB: Voice transcript: "Show me a highly elliptical orbit"]

[STUB: Figure - Screenshot showing elliptical trajectory with visible oblong shape]

\subsubsection{System Execution}

[STUB: Agent chooses periapsis 200 km, apoapsis 35,786 km for "highly elliptical"]

[STUB: Eccentricity calculation e = 0.99]

[STUB: Visual demonstration - learner can walk around ellipse, see periapsis/apoapsis markers]

[STUB: Agent explains velocity variation - Kepler's Second Law]

\subsubsection{Capabilities Demonstrated}

\begin{itemize}
    \item Parameter interpretation from vague input (Objective \#2)
    \item Elliptical orbit physics (Section~\ref{subsec:physics_implementation})
    \item Spatial cognition - observing geometry from multiple angles (Section~\ref{sec:ar_vr_review})
    \item Educational explanation grounded in literature review concepts
\end{itemize}

\subsection{Comparing Orbital Configurations}
\label{subsec:comparing_orbits}

[STUB: Demonstrates multiple simultaneous orbits, comparative learning]

\subsubsection{User Interaction}

[STUB: Creating ISS (420 km) then Hubble (540 km) orbits]

[STUB: Figure - Screenshot showing both orbits visible simultaneously]

\subsubsection{Educational Value}

[STUB: Visual comparison shows altitude-radius relationship]

[STUB: Agent explains velocity difference: ISS 7.66 km/s vs Hubble 7.59 km/s]

[STUB: Demonstrates misconception correction - higher orbit = slower speed]

\subsubsection{Capabilities Demonstrated}

\begin{itemize}
    \item Workspace management (clear\_orbit tool)
    \item Comparative visualization supporting conceptual understanding
    \item Physics accuracy verified against published data (Appendix~\ref{app:physics_implementation})
\end{itemize}
