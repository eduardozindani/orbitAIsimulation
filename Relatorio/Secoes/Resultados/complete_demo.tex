The previous sections demonstrate individual interaction scenarios highlighting specific capabilities. This section presents a complete end-to-end demonstration showing the full system workflow—from initial user contact through orbital exploration, mission navigation, educational dialogue, and simulation control—in continuous operation.

\subsection{Full Demonstration Video}

A comprehensive demonstration video is available online, showcasing uninterrupted interaction with the platform across all environments and functionalities:

[STUB: YouTube video link - complete walkthrough showing all scenarios in sequence]

[STUB: Video duration and content overview]

The demonstration validates the seamless integration of all four core modules described in Section~\ref{sec:core_components}: conversational agent (GPT-4.1), orbital physics simulation (Unity + vis-viva equation), voice pipeline (ElevenLabs bidirectional speech), and VR environment (Quest 3 immersive visualization).

\subsection{Demonstration Coverage}

The complete video showcases:

\begin{itemize}
    \item Initial Hub environment entry and Mission Control greeting
    \item ISS circular orbit creation via natural language (``Create an orbit matching the ISS'')
    \item Elliptical orbit exploration demonstrating eccentricity visualization
    \item Scene transition to ISS Mission Space with voice character switching
    \item Educational dialogue with Anastasia about orbital mechanics concepts
    \item Navigation to Voyager Mission Space
    \item Karl's philosophical explanation of escape trajectories
    \item Simulation time control (speed manipulation)
    \item Workspace clearing and orbit reconstruction
    \item Return navigation to Hub via voice command
    \item Contextual conversation referencing earlier interactions
\end{itemize}

\subsection{Integration Validation}

The continuous demonstration proves real-time system coherence (Objective \#4 from Section~\ref{sec:objetivos}): voice transcription, agent reasoning, tool selection, physics calculation, VR rendering, and audio synthesis all execute reliably without manual intervention or workflow breaks. The user controls the entire experience through voice alone, validating the multimodal interaction paradigm described in Section~\ref{sec:motivation}.

\subsection{Educational Effectiveness Demonstration}

Beyond technical validation, the demonstration illustrates the educational value proposition: a learner with no prior orbital mechanics knowledge can explore concepts like altitude-velocity relationships, eccentricity, inclination, and escape trajectories through natural conversation and spatial visualization. The platform transforms abstract physics equations into tangible, manipulable experiences accessible through spoken dialogue.

This complete system demonstration fulfills the thesis's core promise established in Section~\ref{sec:motivation}: creating a new interface paradigm that is spatial, conversational, and adaptive—making space education experiential rather than instructional.
