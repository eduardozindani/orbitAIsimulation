The complete platform source code is publicly available on GitHub at \texttt{https://github.com/eduardomezencio/orbitAIsimulation}, fulfilling Objective~\#6 from Section~\ref{sec:objetivos}. The repository includes Unity C\# scripts (31 source files organized by module: AI/, Core/, Orbital/, Scenes/, Simulation/), conversational agent prompts (460-line tool selection prompt in \texttt{PromptSettings.asset}), configuration assets (\texttt{MissionConfig} ScriptableObjects for each specialist), and Quest 3 deployment settings (Android build configuration with OpenXR support). Comprehensive documentation (\texttt{README.md}, \texttt{CONTRIBUTING.md}) covers system architecture, Unity configuration procedures, Quest 3 build instructions, and community contribution guidelines. Users provide their own OpenAI and ElevenLabs API keys through Unity Inspector or environment variables, ensuring platform accessibility without imposed service costs.

This open-source delivery enables three validation modes supporting the thesis's broader impact. \textbf{Technical reproducibility:} Researchers can verify physics implementations (vis-viva equation in \texttt{OrbitController.cs}), audit prompt engineering strategies, test performance on alternative VR hardware (Quest 2, Quest Pro, PC-based systems), and measure educational effectiveness through controlled studies. \textbf{Educational adaptation:} Educators can deploy the platform in classrooms without licensing costs, customize mission content to align with curriculum standards, add new physics domains (interplanetary trajectories, Lagrange points, station-keeping), and translate for non-English-speaking students through prompt template modification. \textbf{Research extension:} Developers can experiment with alternative LLM architectures (local models, fine-tuned agents), integrate additional orbital mechanics libraries (SPICE kernels, SGP4 propagation), implement AR passthrough mode (architectural foundation exists in \texttt{ARHub.unity}), and add multiplayer collaboration features for shared orbital workspace exploration.

The modular architecture (Objective~\#5) supports straightforward extensions without core modifications. New mission specialists require five steps: create \texttt{MissionConfig} ScriptableObject with personality and orbit parameters, create Unity scene (duplicate existing mission scene), assign \texttt{MissionSpaceController} component, register mission in \texttt{MissionRegistry.cs}—no core script changes required. New agent tools require four steps: define schema in \texttt{ToolSchemas.json}, implement physics method in \texttt{OrbitController.cs}, map tool to method in \texttt{ToolExecutor.cs}, update agent prompts. Language adaptation requires modifying prompt templates in \texttt{PromptSettings.asset} and configuring ElevenLabs voice IDs for target language models—the LLM automatically generates educational content in the prompt's language.

The repository is licensed under ITA Academic License, which permits educational use and modification with proper attribution to Instituto Tecnológico de Aeronáutica (ITA) while requiring prior written permission for commercial use. This licensing model balances open access with institutional credit, maximizing potential impact on global space education through responsible community collaboration. Contribution infrastructure (\texttt{CONTRIBUTING.md} with PR templates, coding standards, testing checklists; \texttt{OPEN\_SOURCE\_CHECKLIST.md} with release procedures) establishes this as a production-grade open source project enabling sustainable community growth, not merely published code.

By removing proprietary barriers, the thesis contribution becomes a \emph{platform for future research} rather than a closed demonstration—enabling the broader academic and educational community to validate, critique, and extend the work documented in this thesis.
