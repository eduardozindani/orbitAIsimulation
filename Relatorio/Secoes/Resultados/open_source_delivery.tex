The complete platform source code, including Unity C\# scripts, conversational agent prompts, configuration assets, and Quest 3 deployment settings, is publicly available on GitHub, fulfilling Objective~\#6 from Section~\ref{sec:objetivos}. The repository provides comprehensive documentation covering system architecture, Unity configuration procedures, and Quest 3 build instructions, enabling community validation, adaptation, and extension. This open-source delivery ensures that the educational innovations demonstrated in Sections~\ref{sec:entering_environment}--\ref{sec:escape_concept} are accessible to educators, researchers, and developers worldwide without proprietary barriers.

\subsection{Repository Structure and Organization}

The GitHub repository (\texttt{https://github.com/[author]/orbitAIsimulation}) follows standard Unity project conventions with educational-specific additions:

\textbf{Core Unity Project}:
\begin{itemize}
    \item \texttt{Assets/Scripts/}: 31 C\# source files organized by module (AI/, Core/, Orbital/, Scenes/, Simulation/)
    \item \texttt{Assets/Scenes/}: 5 Unity scene files (Hub.unity, ISS.unity, Hubble.unity, Voyager.unity, ARHub.unity)
    \item \texttt{Assets/Resources/}: Configuration JSON files (ToolSchemas.json, mission-specific configs)
    \item \texttt{Assets/Prefabs/}: Reusable GameObjects (Earth, satellite models, UI elements)
    \item \texttt{ProjectSettings/}: Android build configuration, XR packages, rendering pipeline settings
    \item \texttt{Packages/}: Unity Package Manager manifest (Meta XR SDK, OpenXR, URP dependencies)
\end{itemize}

\textbf{Educational Documentation}:
\begin{itemize}
    \item \texttt{Relatorio/}: Complete thesis LaTeX source (this document), enabling reproducibility of academic claims
    \item \texttt{README.md}: Quick-start guide for building and deploying to Quest 3
    \item \texttt{AR\_HUB\_ARCHITECTURE\_BRAINSTORM.md}: Design notes for future AR passthrough development
    \item \texttt{VR\_PASSTHROUGH\_FEASIBILITY.md}: Technical feasibility analysis of AR features
    \item \texttt{INVESTIGATION\_SUMMARY.txt}: Development debugging notes and audio timing fixes
\end{itemize}

\textbf{Key Implementation Files}:
\begin{itemize}
    \item \texttt{PromptConsole.cs}: Main conversation controller, voice pipeline integration (381 lines)
    \item \texttt{OrbitController.cs}: Physics engine, vis-viva equation implementation (450+ lines)
    \item \texttt{OpenAIClient.cs}: GPT-4.1 API integration, tool-calling architecture
    \item \texttt{ElevenLabsClient.cs}: Voice synthesis and speech-to-text integration
    \item \texttt{MissionContext.cs}: Conversation history and scene transition context manager
    \item \texttt{SceneTransitionManager.cs}: Asynchronous scene loading with visual overlays
    \item \texttt{TimeController.cs}: Simulation time manipulation (pause, speed, reset)
    \item \texttt{ToolExecutor.cs}: Agent tool invocation and Unity method mapping
\end{itemize}

This organization separates concerns (AI, physics, VR infrastructure) while maintaining cohesion through shared \texttt{ScriptableObject} configurations, validating Objective~\#5 (modular architecture).

\subsection{User API Key Configuration}

Users provide their own OpenAI and ElevenLabs API keys through Unity Inspector configuration, ensuring platform accessibility without imposed service costs. The documentation (\texttt{README.md}) guides users through:

\textbf{OpenAI Configuration}:
\begin{enumerate}
    \item Obtain API key from \texttt{platform.openai.com/api-keys}
    \item In Unity Editor, navigate to \texttt{Assets/Resources/Config/OpenAISettings.asset}
    \item Paste API key into Inspector field \texttt{apiKey} (serialized as secure string)
    \item Set model: \texttt{gpt-4-turbo-2024-04-09} or later (requires tool-calling support)
\end{enumerate}

\textbf{ElevenLabs Configuration}:
\begin{enumerate}
    \item Obtain API key from \texttt{elevenlabs.io/app/subscription}
    \item In Unity Editor, navigate to \texttt{Assets/Resources/Config/ElevenLabsSettings.asset}
    \item Paste API key into Inspector field \texttt{xiApiKey}
    \item Configure voice IDs for character customization:
    \begin{itemize}
        \item \texttt{CAPCOMVoiceSettings.voiceId = "NOpBlnGInO9m6vDvFkFC"} (Mission Control)
        \item \texttt{AnastasiaVoiceSettings.voiceId = "ZF6FPAbjXT4488VcRRnw"} (ISS specialist)
        \item \texttt{HubbleVoiceSettings.voiceId = [user-selected voice]} (Hubble specialist)
        \item \texttt{KarlVoiceSettings.voiceId = [user-selected voice]} (Voyager specialist)
    \end{itemize}
\end{enumerate}

\textbf{Prompt Template Customization}:
Educational contexts requiring different pedagogical approaches can modify prompts via:
\begin{itemize}
    \item \texttt{PromptSettings.asset}: Hub agent prompts (tool selection, response generation)
    \item \texttt{SpecialistPromptSettings.asset}: Mission Space specialist system prompts
    \item Each \texttt{MissionConfig} asset: Mission-specific knowledge domains and personality descriptions
\end{itemize}

This configuration flexibility enables educators to adapt the platform for different grade levels, languages, or curriculum standards without modifying source code.

\subsection{Quest 3 Deployment Documentation}

The repository includes step-by-step Quest 3 build instructions (\texttt{README.md}, Section ``VR Deployment''):

\textbf{Prerequisites}:
\begin{itemize}
    \item Unity Editor 6000.0.47f1 or later (Unity 6 with OpenXR support)
    \item Android SDK (API level 32 minimum, configured via Unity Hub $\rightarrow$ Android Build Support)
    \item Meta Quest 3 headset with Developer Mode enabled
    \item USB-C cable for sideloading (or ADB wireless debugging)
\end{itemize}

\textbf{Build Configuration} (detailed in Appendix~\ref{app:vr_implementation}):
\begin{enumerate}
    \item Open Unity project
    \item File $\rightarrow$ Build Settings $\rightarrow$ Android platform
    \item Player Settings:
    \begin{itemize}
        \item Minimum API Level: Android 12L (API 32)
        \item Graphics API: OpenGL ES 3.0
        \item Stereo Rendering Mode: Single Pass Instanced
        \item XR Plugin Management: Enable Oculus
    \end{itemize}
    \item Build and Run (deploys to connected Quest 3)
\end{enumerate}

\textbf{Troubleshooting Common Issues}:
\begin{itemize}
    \item \textit{Missing XR packages}: Install via Package Manager $\rightarrow$ XR Plugin Management, Meta XR Core SDK
    \item \textit{Android SDK path not found}: Configure via Edit $\rightarrow$ Preferences $\rightarrow$ External Tools
    \item \textit{``Device unauthorized'' error}: Enable Developer Mode and USB debugging in Quest 3 settings
\end{itemize}

The documentation assumes basic Unity familiarity but provides sufficient detail for educators or students to deploy the platform without advanced technical knowledge.

\subsection{Extensibility Examples and Community Adaptation}

The modular architecture (Objective~\#5) enables straightforward extensions without architectural changes. The documentation provides concrete extension scenarios:

\textbf{Adding New Mission Specialists (Example: Mars Mission)}:
\begin{enumerate}
    \item Create \texttt{Mars\_Config.asset} (ScriptableObject of type \texttt{MissionConfig}):
    \begin{itemize}
        \item \texttt{missionId}: ``Mars''
        \item \texttt{missionName}: ``Mars Transfer Trajectory''
        \item \texttt{specialistName}: ``Dr. Zubrin''
        \item \texttt{specialistPersonality}: ``Enthusiastic Mars colonization advocate''
        \item \texttt{knowledgeDomain}: ``Hohmann transfer orbits, Mars atmospheric entry, in-situ resource utilization...''
        \item \texttt{orbitType}: Elliptical (Earth departure $\rightarrow$ Mars arrival)
        \item \texttt{periapsisKm}: 160, \texttt{apoapsisKm}: 200,000 (interplanetary trajectory)
    \end{itemize}
    \item Create \texttt{Mars.unity} scene (duplicate ISS.unity, modify visuals)
    \item Add \texttt{MissionSpaceController} component, assign \texttt{Mars\_Config.asset}
    \item Register in \texttt{MissionRegistry.cs}: \texttt{availableMissions.Add(marsConfig)}
\end{enumerate}

This 5-step process creates a new specialist without modifying core scripts—demonstrating architectural extensibility.

\textbf{Extending Tool Suite (Example: Gravity Assist Tool)}:
\begin{enumerate}
    \item Add tool definition to \texttt{ToolSchemas.json}:
\begin{verbatim}
{
  "name": "simulate_gravity_assist",
  "parameters": {
    "flyby_altitude_km": {"type": "number", "min": 100},
    "approach_velocity_km_s": {"type": "number"}
  }
}
\end{verbatim}
    \item Implement method in \texttt{OrbitController.cs}: \texttt{SimulateGravityAssist(float altitude, float velocity)}
    \item Add case in \texttt{ToolExecutor.cs}: map tool name to method invocation
    \item Update agent prompt (\texttt{PromptSettings.toolSelectionPrompt}) to describe new capability
\end{enumerate}

\textbf{Language Adaptation (Example: Portuguese Educational Content)}:
\begin{enumerate}
    \item Modify prompt templates in \texttt{PromptSettings.asset}: translate English system prompts to Portuguese
    \item Update \texttt{MissionConfig} knowledge domains with Portuguese mission descriptions
    \item Configure ElevenLabs voice IDs to use Portuguese voice models
    \item Educational content (specialist explanations, CAPCOM responses) automatically generated in Portuguese by LLM
\end{enumerate}

This demonstrates the platform's flexibility for international educational contexts—critical for global accessibility.

\subsection{Community Validation and Academic Reproducibility}

The open-source nature fulfills multiple validation objectives:

\textbf{Technical Reproducibility}: Researchers can:
\begin{itemize}
    \item Verify physics implementations (vis-viva equation in \texttt{OrbitController.cs})
    \item Audit prompt engineering (460-line tool selection prompt in \texttt{PromptSettings.asset})
    \item Test system performance on different VR hardware (Quest 2, Quest Pro, PC-based VR)
    \item Measure educational effectiveness through controlled studies
\end{itemize}

\textbf{Educational Adaptation}: Educators can:
\begin{itemize}
    \item Deploy the platform in classrooms without licensing costs (API keys are user-provided)
    \item Customize mission content to align with curriculum standards
    \item Add new physics domains (interplanetary, Lagrange points, station-keeping)
    \item Translate for non-English-speaking students
\end{itemize}

\textbf{Research Extension}: Developers can:
\begin{itemize}
    \item Experiment with alternative LLM architectures (local models, fine-tuned agents)
    \item Integrate additional orbital mechanics libraries (SPICE kernels, SGP4 propagation)
    \item Implement AR passthrough mode (architectural foundation exists in \texttt{ARHub.unity})
    \item Add multiplayer collaboration features (shared orbital workspace)
\end{itemize}

This open-source delivery validates the platform's educational accessibility and community extensibility objectives established in Section~\ref{sec:motivation}. By removing proprietary barriers, the thesis contribution becomes a \emph{platform for future research} rather than a closed demonstration—enabling the broader academic and educational community to validate, critique, and extend the work.

The repository includes licensing under MIT License (permissive open-source), explicitly allowing commercial educational use, modification, and redistribution with attribution—maximizing potential impact on global space education.
