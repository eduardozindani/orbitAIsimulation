To enable independent verification and replication of this feasibility study, the complete implementation is publicly available at \url{https://github.com/ezindani/orbitAIsimulation}. The repository contains all Unity C\# source code (31 core runtime scripts in Assets/Scripts/ across AI/, Orbital/, Core/ modules, plus 7 editor utilities, totaling 72 C\# files), conversational prompts, configuration assets, and Quest 3 deployment settings. Documentation covers system architecture, API configuration (users provide their own OpenAI/ElevenLabs keys), and build procedures.

This availability serves three research purposes: \textbf{(1) Technical verification}—researchers can audit physics implementations (vis-viva equations in \texttt{OrbitController.cs}) and prompt engineering strategies; \textbf{(2) Replication studies}—the platform can be deployed in controlled educational experiments to measure learning outcomes; \textbf{(3) Extension research}—the modular architecture (Section~\ref{sec:design_philosophy}) enables investigation of alternative configurations (local LLMs, different VR hardware, additional physics domains) without invalidating baseline feasibility claims.

The implementation is released under MIT License to minimize barriers to academic use. The focus of this work is demonstrating operational feasibility through working prototype; questions of pedagogical effectiveness, scalability, and widespread adoption remain subjects for future controlled studies.
