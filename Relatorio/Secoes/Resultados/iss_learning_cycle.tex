This section demonstrates the platform's complete pedagogical cycle: a learner asks a conceptual question, the agent routes to a mission specialist for expert guidance, the user returns to create an orbit based on that knowledge, and observes the resulting physics in immersive VR. This cycle embodies the learning paradigm described in Section~\ref{sec:motivation}: spatial, conversational, and adaptive education that transforms abstract orbital mechanics into tangible experience.

The demonstration follows a natural learning progression: the user wants to create a circular orbit but doesn't know what altitude to choose. Rather than providing a formula or arbitrary number, CAPCOM connects the user with ANASTASIA—an ISS crew member who explains real-world altitude selection through operational context. This specialist consultation transforms ``altitude'' from an abstract parameter into a meaningful engineering decision with practical trade-offs.

\subsection{The Learner's Question: Seeking Conceptual Foundation}
\label{subsec:learner_question}

Having learned that circular orbits require an altitude parameter (Section~\ref{subsec:satellite_discovery}), the user recognizes a knowledge gap and asks a foundational question:

\begin{quote}
\textit{USER:} Cool. Let's start with a circular orbit. I need an altitude—but what's a good altitude? What determines a good altitude?
\end{quote}

This question represents genuine curiosity-driven learning. The user is not asking for procedural instruction (``how do I create an orbit?'') but conceptual understanding (``what makes an altitude good?''). This distinction is pedagogically significant: the learner seeks not just to operate the simulation but to understand orbital design principles.

CAPCOM's agent reasoning (GPT-4.1 with \texttt{toolSelectionPrompt} from Appendix~\ref{app:agent_implementation}) analyzes this input and recognizes:
\begin{itemize}
    \item The question is \emph{educational}, not \emph{operational}—no tool execution needed yet
    \item The topic (altitude selection for circular orbits) falls within the ISS specialist's knowledge domain
    \item Routing to a real-mission context will provide more meaningful learning than a generic physics explanation
\end{itemize}

The agent selects the \texttt{route\_to\_mission} tool with parameter \texttt{mission\_id = "ISS"} and prepares a routing announcement:

\begin{quote}
\textit{CAPCOM:} [Music] Connecting you to the ISS mission for real-world examples. Stand by.
\end{quote}

This response demonstrates several sophisticated capabilities:
\begin{itemize}
    \item \textbf{Intent recognition}: Agent understood that ``what determines a good altitude?'' requires educational context, not simulation control
    \item \textbf{Specialist selection}: Correctly matched the question topic (LEO circular orbit altitude) to ISS expertise
    \item \textbf{Contextual routing}: The phrase ``real-world examples'' sets expectations that the user will receive mission-specific, applied knowledge
\end{itemize}

The tool execution triggers \texttt{ToolExecutor.ExecuteRouteToMission()} (Appendix~\ref{app:agent_implementation}), which invokes \texttt{SceneTransitionManager.TransitionToMission("ISS")}. Before the scene transition begins, the routing context is preserved in \texttt{MissionContext.Instance} (Section~\ref{sec:system_architecture}), storing that the user arrived with the question ``what determines a good altitude?''—ensuring ANASTASIA can tailor her introduction appropriately.

\subsection{Scene Transition to ISS Mission Space: Modular Architecture in Action}
\label{subsec:scene_transition_iss}

The scene transition from Hub to ISS Mission Space demonstrates Objective~\#5 (modular architecture) and Objective~\#4 (real-time coherence). During the 4-second asynchronous load:

\begin{itemize}
    \item \textbf{Visual feedback}: ISS mission logo displays as transition overlay, providing visual continuity during the load
    \item \textbf{Audio continuity}: Background music persists through \texttt{ExperienceManager.DontDestroyOnLoad()} (Section~\ref{sec:system_architecture})
    \item \textbf{Context preservation}: \texttt{MissionContext} maintains the conversation history (last 10 exchanges) and routing reason
    \item \textbf{Character preparation}: \texttt{MissionSpaceController} (Section~\ref{sec:core_components}) configures ANASTASIA's voice profile and specialist prompt
\end{itemize}

The user experiences no interruption in presence—VR immersion continues smoothly as the environment changes from Hub's open orbital workspace to ISS Mission Space's pre-built orbit visualization.

\begin{figure}[h]
\centering
\includegraphics[width=0.85\textwidth]{Imagens/iss_mission.png}
\caption{ISS Mission Space environment after scene transition. The pre-configured ISS orbit (420 km altitude, 51.6° inclination) is visible as a cyan trajectory. ANASTASIA, the ISS specialist, greets the user with context awareness, knowing they came to learn about altitude selection. This environment demonstrates the \texttt{MissionConfig} ScriptableObject system (Appendix~\ref{app:agent_implementation}) that defines mission-specific orbits, characters, and knowledge domains.}
\label{fig:iss_mission}
\end{figure}

\FloatBarrier

Upon scene load completion, \texttt{MissionSpaceController.Start()} executes the initialization sequence (lines 44-79 of MissionSpaceController.cs):
\begin{enumerate}
    \item Creates the ISS circular orbit (420 km, 51.6° inclination) via \texttt{OrbitController.CreateCircularOrbit()}
    \item Updates \texttt{MissionContext.Instance.currentLocation = "ISS"}
    \item Sets ANASTASIA's voice profile (ElevenLabs voice ID: \texttt{ZF6FPAbjXT4488VcRRnw})
    \item Triggers specialist introduction after 2.5-second delay (allowing for visual fade-in)
\end{enumerate}

ANASTASIA's introduction demonstrates context-aware greeting generation:

\begin{quote}
\textit{ISS\_CREW (ANASTASIA):} [Music] Hello from the ISS. For circular LEO orbits, engineers balance drag, launch energy, accessibility, and mission needs. The ISS flies around ~420 km—high enough to limit atmospheric drag so we don't slow down too quickly, but not so high that crew/cargo launches become inefficient. It's also reachable from multiple launch sites. Too low: frequent reboosts. Too high: launches get harder and more expensive.
\end{quote}

\subsection{Specialist Educational Dialogue: Real-World Context and Engineering Trade-offs}
\label{subsec:specialist_dialogue}

ANASTASIA's explanation demonstrates the educational quality of specialist knowledge integration. Rather than citing the vis-viva equation ($v = \sqrt{\mu/r}$ from Section~\ref{sec:orbital_mechanics}), she provides \emph{applied engineering rationale}:

\textbf{Trade-off Analysis}:
\begin{itemize}
    \item \textbf{Lower bound constraint}: ``Too low: frequent reboosts'' references atmospheric drag effects below ~300 km
    \item \textbf{Upper bound constraint}: ``Too high: launches get harder and more expensive'' reflects launch vehicle $\Delta v$ budget limitations
    \item \textbf{Operational optimum}: ``~420 km'' emerges as a balanced compromise, not an arbitrary choice
    \item \textbf{Accessibility consideration}: ``reachable from multiple launch sites'' explains why ISS uses 51.6° inclination (Section~\ref{sec:orbital_mechanics}—accommodates Baikonur launches)
\end{itemize}

This explanation transforms altitude from a numerical input into an \emph{engineering decision} with multidimensional constraints. The learner now understands that ``good altitude'' depends on mission priorities—crew safety, launch accessibility, observational requirements, and operational costs.

The user continues the conversation with follow-up questions, demonstrating natural dialogue flow:

\begin{quote}
\textit{USER:} Would it be absurd to choose the same altitude as the ISS?

\textit{ISS\_CREW (ANASTASIA):} Not absurd at all—quite sensible. ~420 km is a proven compromise. Many Earth-observation and crewed missions use similar altitudes for the same reasons. You'd be following decades of flight heritage.
\end{quote}

This exchange validates the learner's reasoning and reinforces that ISS's altitude is not unique but representative of a broader LEO operational regime. ANASTASIA's response uses encouraging language (``quite sensible,'' ``flight heritage'') that builds confidence while educating.

\begin{quote}
\textit{USER:} Is the ISS circular?

\textit{ISS\_CREW (ANASTASIA):} Nearly. The path is close to circular with small variations (drag, gravity field). We keep it tight with periodic reboosts. Near-circular simplifies rendezvous and ops.
\end{quote}

This technical clarification introduces realistic complexity: the ISS's orbit is \emph{nearly} circular, not perfectly so, due to perturbations. The mention of ``reboosts'' and ``rendezvous'' connects to operational concerns (visiting vehicles, station-keeping). This dialogue demonstrates that specialist characters provide mission-specific depth beyond generic physics explanations.

When ready to return, the user explicitly requests navigation:

\begin{quote}
\textit{USER:} What was the altitude again? Please route me back to the hub.

\textit{ISS\_CREW (ANASTASIA):} ~420 km. Handing you back. Safe travels.
\end{quote}

ANASTASIA's concise recap (``~420 km'') and graceful handoff (``Safe travels'') demonstrate conversational coherence and character consistency. The agent recognizes the routing request and executes \texttt{return\_to\_hub} tool, triggering scene transition back to the Hub environment.

\paragraph{Pedagogical Insights}
This specialist dialogue exemplifies the educational design principles from Section~\ref{sec:motivation}:
\begin{itemize}
    \item \textbf{Contextual knowledge over abstract formulas}: Altitude explained through operational constraints (drag, launch costs) rather than equations
    \item \textbf{Adaptive explanation depth}: ANASTASIA adjusts detail level across answers—comprehensive on first question, concise on recap
    \item \textbf{Confidence building}: Validating the user's proposal (``quite sensible'') encourages engagement and reduces intimidation
    \item \textbf{Real-world grounding}: References to ``flight heritage,'' ``reboosts,'' ``rendezvous'' connect simulation to actual spaceflight operations
\end{itemize}

\subsection{Returning to Hub and Creating the Circular Orbit: Applying Knowledge}
\label{subsec:orbit_creation}

After returning to the Hub (via \texttt{return\_to\_hub} tool and scene transition), the user applies the learned knowledge by creating a circular orbit at ISS altitude:

\begin{quote}
\textit{USER:} I'll build a circular orbit at 422 km.
\end{quote}

The user chose 422 km—close to the ISS's ~420 km that ANASTASIA explained. This choice demonstrates knowledge transfer: the learner is not blindly following instructions but applying the engineering rationale learned in the ISS Mission Space.

CAPCOM executes \texttt{create\_circular\_orbit} tool with parameters \texttt{altitude\_km = 422}, \texttt{inclination\_deg = 0} (defaulting to equatorial for simplicity in this first orbit):

\begin{quote}
\textit{CAPCOM:} Copy. Circular orbit set at 422 km, inclination 0° (equatorial). Orbital speed ~7.66 km/s—like the ISS, but over the equator.
\end{quote}

This response validates several implementation details:

\textbf{Physics Calculation} (Appendix~\ref{app:physics_implementation}):
\begin{align*}
r &= R_\oplus + h = 6{,}371 + 422 = 6{,}793 \text{ km} \\
v_{\text{circular}} &= \sqrt{\frac{\mu}{r}} = \sqrt{\frac{398{,}600}{6{,}793}} = 7.66 \text{ km/s}
\end{align*}

The calculated velocity (7.66 km/s) matches the real ISS orbital velocity, validating Objective~\#1 (physically accurate simulation). \texttt{OrbitController.CreateCircularOrbit()} performs this calculation automatically and renders the trajectory as a cyan 3D curve in VR space.

\textbf{Educational Response Quality}: CAPCOM's confirmation includes:
\begin{itemize}
    \item Echoing parameters (``422 km, inclination 0°'') for clarity
    \item Providing derived physics (``orbital speed ~7.66 km/s'') that the user didn't specify but the system calculated
    \item Contextual comparison (``like the ISS, but over the equator'') connecting this custom orbit to the reference mission
\end{itemize}

The user's next question tests a common misconception:

\begin{quote}
\textit{USER:} Can I choose the speed?

\textit{CAPCOM:} In this sim, speed is derived from altitude by physics. At 422 km, you need ~7.66 km/s for a stable circular orbit. If you want a different speed, pick a different altitude.
\end{quote}

This exchange demonstrates \textbf{misconception prevention} through agent design. Many learners initially believe orbital speed is an independent variable that can be freely adjusted. CAPCOM's explanation clarifies the physics constraint: for a stable circular orbit, velocity is \emph{determined by} altitude through the vis-viva equation (Section~\ref{sec:orbital_mechanics}). This prevents the misconception while teaching the fundamental relationship: higher altitude $\Rightarrow$ slower speed.

The 460-line \texttt{toolSelectionPrompt} (Appendix~\ref{app:agent_implementation}) explicitly guides the agent to distinguish between:
\begin{itemize}
    \item \textbf{Orbital velocity}: Physics-calculated parameter from altitude (7.66 km/s for 422 km)
    \item \textbf{Simulation time speed}: User-controllable playback multiplier (addressed in Section~\ref{subsec:observing_motion})
\end{itemize}

This disambiguation appears in multiple demonstration moments, proving that the prompt architecture successfully prevents tool confusion.

\subsection{Observing and Understanding Orbital Motion: Visual Physics and Time Control}
\label{subsec:observing_motion}

With the circular orbit created, the user explores its characteristics through observation and time manipulation:

\begin{quote}
\textit{USER:} How long does one lap take? It looks fast, but Earth is massive.
\end{quote}

This question reveals spatial-cognitive engagement: the user is \emph{observing} the satellite's motion in VR and forming intuitions (``it looks fast'') that prompt quantitative inquiry. This demonstrates the embodied learning described in Section~\ref{sec:ar_vr_review}—spatial visualization generating conceptual questions.

CAPCOM routes back to ISS for period context:

\begin{quote}
\textit{ISS\_CREW (ANASTASIA):} At ~420 km, the ISS completes a lap in ~92.8 minutes—about 15 orbits/day. Higher altitude → slower speed → longer period. That's why we see a sunrise/sunset ~every 45 minutes.
\end{quote}

ANASTASIA provides both quantitative data (92.8 minutes, 15 orbits/day) and experiential context (sunrise/sunset every 45 minutes). The latter grounds abstract orbital period in lived experience—ISS crew members experience rapid day-night cycles as a direct consequence of their 90-minute orbit. This connection between physics and experience exemplifies educational contextualization.

The relationship ``higher altitude → slower speed → longer period'' summarizes Kepler's Third Law consequences without citing the mathematical form ($T^2 \propto a^3$ from Section~\ref{sec:orbital_mechanics}).

\begin{figure}[h]
\centering
\includegraphics[width=0.85\textwidth]{Imagens/orbit_view1.png}
\caption{Close perspective view of the 422 km circular orbit after creation. The cyan trajectory traces a perfect circle around Earth's equator (inclination 0°). The satellite model is visible at one point along the orbit. This visualization enables spatial observation of circular geometry that would be difficult to comprehend from 2D diagrams. The user can move their VR viewpoint to observe the orbit from multiple angles, supporting the spatial cognition discussed in Section~\ref{sec:ar_vr_review}.}
\label{fig:orbit_view1}
\end{figure}

\FloatBarrier

After returning to Hub again, the user requests time acceleration:

\begin{quote}
\textit{USER:} With our current orbit (~92 minutes), can we speed up the simulation so I can see it go around?

\textit{CAPCOM:} Time acceleration: 10×. You'll see the ~92-minute trip much faster.
\end{quote}

This interaction demonstrates successful tool disambiguation. The user explicitly says ``speed up the simulation,'' avoiding ambiguous phrasing like ``make it faster.'' CAPCOM correctly selects \texttt{set\_simulation\_speed} tool with \texttt{speed\_multiplier = 10}, invoking \texttt{TimeController.SetSpeed(10.0f)} (Appendix~\ref{app:physics_implementation}).

Critically, this does \emph{not} change the orbital velocity (7.66 km/s)—that physics parameter remains constant. Instead, Unity's \texttt{Time.timeScale} increases to 10×, making the satellite complete its 92-minute orbit in 9.2 real-world minutes of playback. This distinction—\emph{simulation playback speed} vs \emph{orbital physics speed}—is preserved throughout the agent's responses, proving that the prompt engineering successfully prevents confusion.

The user can now observe multiple complete orbits, watching the satellite trace its circular path repeatedly. This temporal compression makes the 92-minute period perceptible within a reasonable observation window, supporting experiential understanding of orbital motion.

\paragraph{Learning Cycle Complete}

At this point, the user has completed a full pedagogical cycle:
\begin{enumerate}
    \item \textbf{Question}: What determines a good altitude?
    \item \textbf{Guidance}: ISS specialist explains engineering trade-offs (drag, launch costs, accessibility)
    \item \textbf{Application}: User creates 422 km circular orbit based on ISS example
    \item \textbf{Observation}: User explores orbit properties (velocity, period) through VR visualization
    \item \textbf{Control}: User manipulates time to observe complete orbital cycles
    \item \textbf{Understanding}: User develops intuition for altitude-velocity-period relationships
\end{enumerate}

This cycle validates the thesis's central claim (Section~\ref{sec:motivation}): combining immersive VR with generative AI enables spatial, conversational, adaptive learning that transforms abstract physics into tangible experience. The user learned orbital mechanics not through textbook formulas but through contextual dialogue, hands-on creation, and embodied observation.

\paragraph{Capabilities Validated}

This section demonstrates integration of all four core modules:
\begin{itemize}
    \item \textbf{Agent System}: Question interpretation → tool selection (\texttt{route\_to\_mission}, \texttt{create\_circular\_orbit}, \texttt{set\_simulation\_speed}, \texttt{return\_to\_hub}) → educational responses
    \item \textbf{Physics Simulation}: Accurate circular orbit calculation (vis-viva equation) → trajectory rendering → time-scaled playback
    \item \textbf{Voice Pipeline}: Multiple bidirectional conversations → character switching (CAPCOM ↔ ANASTASIA) → response latency under 7 seconds
    \item \textbf{VR Environment}: Scene transitions (Hub ↔ ISS) → spatial observation → 90 Hz rendering maintained throughout
\end{itemize}

Objectives validated:
\begin{itemize}
    \item \textbf{\#1 (Physics accuracy)}: 422 km → 7.66 km/s matches real ISS velocity
    \item \textbf{\#2 (Natural language)}: Questions like ``what's a good altitude?'' correctly interpreted and routed
    \item \textbf{\#3 (Multimodal VR)}: Voice + spatial presence + stereoscopic rendering enabling embodied learning
    \item \textbf{\#4 (Real-time coherence)}: Scene transitions, tool execution, voice synthesis all operating without interruption
    \item \textbf{\#5 (Modular architecture)}: Hub and ISS Mission Space as separate scenes sharing conversation context
\end{itemize}

The next section (Section~\ref{sec:elliptical_exploration}) demonstrates how the platform supports iterative exploration: the user progresses from circular orbits to elliptical orbits, consulting the Hubble specialist to understand eccentricity and speed variation.
