This section demonstrates the platform's complete pedagogical cycle: conceptual question → specialist consultation → hands-on orbit creation → VR observation. The user wants to create a circular orbit but doesn't know what altitude to choose.

\subsection{Conceptual Question and Specialist Routing}
\label{subsec:learner_question}

The user asks a foundational question (see Appendix~\ref{appendix:transcript} for complete dialogue):

\begin{quote}
\textit{USER:} Cool. Let's start with a circular orbit. I need an altitude—but what's a good altitude? What determines a good altitude?
\end{quote}

CAPCOM recognizes this as an educational question requiring mission-specific context and routes to the ISS specialist:

\begin{quote}
\textit{CAPCOM:} Connecting you to the ISS mission for real-world examples. Stand by.
\end{quote}

\subsection{ISS Mission Space and Specialist Dialogue}
\label{subsec:scene_transition_iss}

The scene transitions to ISS Mission Space with conversation context preserved. The environment loads with the pre-configured ISS orbit (420 km altitude, 51.6° inclination) visible as a cyan trajectory.

\begin{figure}[h]
\centering
\includegraphics[width=0.85\textwidth]{Imagens/iss_mission.png}
\caption{ISS Mission Space showing the pre-configured ISS orbit. ANASTASIA greets the user with context awareness about altitude selection.}
\label{fig:iss_mission}
\end{figure}

\FloatBarrier

ANASTASIA provides context-aware guidance:

\begin{quote}
\textit{ISS\_CREW (ANASTASIA):} Hello from the ISS. For circular LEO orbits, engineers balance drag, launch energy, accessibility, and mission needs. The ISS flies around ~420 km—high enough to limit atmospheric drag so we don't slow down too quickly, but not so high that crew/cargo launches become inefficient. It's also reachable from multiple launch sites. Too low: frequent reboosts. Too high: launches get harder and more expensive.
\end{quote}

The user continues with follow-up questions, validating their reasoning by asking if choosing ISS altitude would be ``absurd.'' ANASTASIA confirms it's ``quite sensible'' and representative of ``decades of flight heritage.'' When asked if the ISS orbit is circular, she clarifies: ``Nearly. The path is close to circular with small variations [...] Near-circular simplifies rendezvous and ops.'' After recapping the altitude (~420 km), the user requests return to Hub.

\subsection{Orbit Creation and Physics Validation}
\label{subsec:orbit_creation}

After returning to the Hub, the user creates a circular orbit:

\begin{quote}
\textit{USER:} I'll build a circular orbit at 422 km.

\textit{CAPCOM:} Copy. Circular orbit set at 422 km, inclination 0° (equatorial). Orbital speed ~7.66 km/s—like the ISS, but over the equator.
\end{quote}

The calculated velocity (7.66 km/s) matches the real ISS orbital speed, validating physics accuracy. The user then tests a common misconception:

\begin{quote}
\textit{USER:} Can I choose the speed?

\textit{CAPCOM:} In this sim, speed is derived from altitude by physics. At 422 km, you need ~7.66 km/s for a stable circular orbit. If you want a different speed, pick a different altitude.
\end{quote}

CAPCOM's response prevents a common misconception by clarifying that orbital speed is determined by altitude through physics, not an independent variable.

\subsection{Orbital Period and Time Acceleration}
\label{subsec:observing_motion}

The user observes the created orbit and asks about period:

\begin{quote}
\textit{USER:} How long does one lap take? It looks fast, but Earth is massive.

\textit{ISS\_CREW (ANASTASIA):} At ~420 km, the ISS completes a lap in ~92.8 minutes—about 15 orbits/day. Higher altitude → slower speed → longer period. That's why we see a sunrise/sunset ~every 45 minutes.
\end{quote}

\begin{figure}[h]
\centering
\includegraphics[width=0.85\textwidth]{Imagens/orbit_view1.png}
\caption{Close perspective view of the 422 km circular orbit after creation. The cyan trajectory traces a perfect circle around Earth's equator (inclination 0°). The satellite model is visible at one point along the orbit. This visualization enables spatial observation of circular geometry that would be difficult to comprehend from 2D diagrams. The user can move their VR viewpoint to observe the orbit from multiple angles, supporting the spatial cognition discussed in Section~\ref{sec:ar_vr_review}.}
\label{fig:orbit_view1}
\end{figure}

\FloatBarrier

After returning to Hub, the user requests time acceleration to observe multiple orbits:

\begin{quote}
\textit{USER:} With our current orbit (~92 minutes), can we speed up the simulation so I can see it go around?

\textit{CAPCOM:} Time acceleration: 10×. You'll see the ~92-minute trip much faster.
\end{quote}

This temporal compression enables observation of the complete 92-minute orbital cycle within a reasonable viewing window, demonstrating successful disambiguation between simulation playback speed and orbital velocity.
