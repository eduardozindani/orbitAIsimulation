The user's conceptual progression continues to escape trajectories (full transcript in Appendix~\ref{appendix:transcript}):

\subsection{Voyager Specialist: Beyond Bound Orbits}
\label{subsec:boundary_question}

\begin{quote}
\textit{USER:} Wow—huge difference between near and far. For deep-space missions, I assume there aren't circular orbits at all. Is everything elliptical?
\end{quote}

CAPCOM routes to Voyager specialist:

\begin{quote}
\textit{CAPCOM:} Routing to Voyager for escape-trajectory talk.
\end{quote}

Karl explains the distinction between bound and unbound trajectories:

\begin{quote}
\textit{VOYAGER\_SPECIALIST (Karl):} Great question. To leave Earth, you don't stay in an Earth orbit. Voyager 1 began in a near-circular parking orbit briefly, then received a final boost to a hyperbolic escape trajectory—no longer bound to Earth. Satellites around Earth use circular or elliptical orbits to remain nearby; escape missions transition from a staging orbit to hyperbolic and head out—past Earth, past the Sun's planets, toward interstellar space.
\end{quote}

\begin{figure}[h]
\centering
\includegraphics[width=0.75\textwidth]{Imagens/voyager_mission.png}
\caption{Voyager Mission Space environment. Karl, the Voyager specialist, embodies a contemplative character distinct from the technical specialists encountered earlier. The voice synthesis uses slower pacing (speed 0.9, configured in \texttt{KarlVoiceSettings}) to convey thoughtful reflection. This environment demonstrates that the platform supports both experiential learning (Sections~\ref{sec:iss_learning} and~\ref{sec:elliptical_exploration}) and conceptual dialogue.}
\label{fig:voyager_mission}
\end{figure}

\FloatBarrier

The user continues exploring:

\begin{quote}
\textit{USER:} So it's like breaking free from gravity?

\textit{VOYAGER\_SPECIALIST (Karl):} Exactly. Think of it as climbing out of Earth's gravitational well. Circular and elliptical orbits stay in the well, perpetually falling around Earth. A hyperbolic trajectory has enough energy to climb out completely. Voyager didn't just orbit—it escaped.
\end{quote}

Karl uses accessible metaphors to explain the physics. When asked about mission execution:

\begin{quote}
\textit{USER:} How did they actually do it?

\textit{VOYAGER\_SPECIALIST (Karl):} Voyager launched from Earth, first reaching a temporary parking orbit around 160-200 km. Then, at the optimal moment, the upper stage fired—adding enough velocity to exceed escape speed. That burn transitioned the trajectory from elliptical to hyperbolic. Once beyond Earth's sphere of influence, Voyager's path was governed by the Sun's gravity, heading toward Jupiter for the first gravitational assist.
\end{quote}

The final exchange adds philosophical context:

\begin{quote}
\textit{USER:} That's incredible. It's really leaving everything behind.

\textit{VOYAGER\_SPECIALIST (Karl):} Past Earth, past the Sun's planets, toward interstellar space. Voyager carries humanity's first message to the cosmos. The golden record, the images, the sounds of Earth—all riding on a hyperbolic trajectory that will never return. It's physics, yes, but it's also a statement of who we are and what we dream.
\end{quote}

Karl's contemplative voice pacing and philosophical framing differentiate this specialist from Anastasia's operational focus and Dr. Harrison's engineering precision. Unlike previous sections, this interaction remains conceptual—the user does not create a hyperbolic trajectory—demonstrating the platform's support for both experiential and theoretical learning modes.
