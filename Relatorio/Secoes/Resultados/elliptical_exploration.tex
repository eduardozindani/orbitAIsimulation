Building on circular orbit knowledge, the user explores elliptical geometries with visible speed variation.

\subsection{Hubble Specialist Consultation}
\label{subsec:elliptical_question}

The user asks about elliptical orbits (Appendix~\ref{appendix:transcript}):

\begin{quote}
\textit{USER:} Great. I built a circular orbit. What's an elliptical orbit?

\textit{CAPCOM:} Perfect question for Hubble—routing.
\end{quote}

DR\_HARRISON provides a comparative explanation:

\begin{quote}
\textit{HUBBLE\_ENGINEER (DR\_HARRISON):} Circular means constant distance and speed—like Hubble ($\approx$540 km) for steady imaging and predictable servicing. Elliptical stretches the path: closer at periapsis (faster), farther at apoapsis (slower). Some science missions pick elliptical to linger where conditions are best or to avoid interference (e.g., high apogee above the radiation belts).
\end{quote}

\begin{figure}[h]
\centering
\includegraphics[width=0.85\textwidth]{Imagens/hubble_mission.png}
\caption{Hubble Mission Space with DR\_HARRISON explaining elliptical orbital geometry. The environment shows Hubble's circular orbit (540 km altitude, 28.5° inclination) as a reference for comparison.}
\label{fig:hubble_mission}
\end{figure}

\FloatBarrier

When asked why missions choose elliptical orbits:

\begin{quote}
\textit{USER:} Why choose elliptical if circular is simpler?

\textit{HUBBLE\_ENGINEER (DR\_HARRISON):} To optimize time in a region. Example: Chandra uses a highly elliptical orbit to spend long stretches far from Earth's radiation belts for cleaner X-ray observations. Trade-off: more complex tracking/operations.
\end{quote}

\subsection{Iterative Orbit Creation}
\label{subsec:first_elliptical}

After returning to Hub, the user requests elliptical orbit creation:

\begin{quote}
\textit{USER:} Please build an elliptical orbit—anything interesting so I can compare.

\textit{CAPCOM:} Elliptical orbit set: periapsis 400 km, apoapsis 2,000 km, inclination 0°. Watch it speed up near Earth and slow down far away.
\end{quote}

\begin{figure}[h]
\centering
\includegraphics[width=0.85\textwidth]{Imagens/orbit_view2.png}
\caption{First elliptical orbit (periapsis 400 km, apoapsis 2,000 km) showing visible elongation compared to the circular orbit.}
\label{fig:orbit_view2}
\end{figure}

\FloatBarrier

After observing at 100× time acceleration, the user provides qualitative feedback:

\begin{quote}
\textit{USER:} Make it more elliptical; it still looks circular.
\end{quote}

CAPCOM creates a more eccentric orbit:

\begin{quote}
\textit{CAPCOM:} Roger. New orbit: periapsis 200 km, apoapsis 1,000 km, inclination 0°. The speed contrast will be much more dramatic.
\end{quote}

The user observes visible speed variation:

\begin{quote}
\textit{USER:} Wow—huge difference between near and far.
\end{quote}

\begin{figure}[h]
\centering
\includegraphics[width=0.85\textwidth]{Imagens/orbit_view3.png}
\caption{Satellite at apoapsis (farthest point, slowest velocity).}
\label{fig:orbit_view3}
\end{figure}

\begin{figure}[h]
\centering
\includegraphics[width=0.85\textwidth]{Imagens/orbit_view4.png}
\caption{Satellite at periapsis (closest point, fastest velocity), demonstrating Kepler's Second Law through visible speed contrast.}
\label{fig:orbit_view4}
\end{figure}

\FloatBarrier
