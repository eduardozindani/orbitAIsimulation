\subsection{Navigating to ISS Mission Space}
\label{subsec:iss_navigation}

[STUB: Demonstrates scene transitions, character switching, context persistence]

\subsubsection{User Interaction}

[STUB: Voice transcript: "Take me to the ISS mission"]

[STUB: Figure - Transition overlay showing ISS logo during 4-second async scene load]

\subsubsection{System Execution}

[STUB: Agent selects route\_to\_mission tool with parameter "ISS"]

[STUB: SceneTransitionManager loads ISS.unity asynchronously]

[STUB: Voice switches from Mission Control (NOpBlnGInO9m6vDvFkFC) to Anastasia (ZF6FPAbjXT4488VcRRnw)]

[STUB: Conversation history preserved - 10-exchange window maintained]

\subsubsection{Capabilities Demonstrated}

\begin{itemize}
    \item Navigation tools (Objective \#5 - modular architecture)
    \item Scene transitions without stuttering (Appendix~\ref{app:vr_implementation})
    \item Character voice switching (Appendix~\ref{app:voice_implementation})
    \item Context persistence across scenes (Section~\ref{subsec:agent_implementation})
\end{itemize}

\subsection{Educational Dialogue with Specialists}
\label{subsec:specialist_dialogue}

[STUB: Demonstrates non-tool responses, specialist knowledge domains, educational guidance]

\subsubsection{User Interaction}

[STUB: Voice transcript in ISS Space: "Why is the ISS at 51.6 degrees inclination?"]

[STUB: Figure - ISS Mission Space environment screenshot]

\subsubsection{System Execution}

[STUB: Agent recognizes educational question - no tool execution]

[STUB: Draws from ISS specialist knowledge domain configured in ISS\_Config.asset]

[STUB: Anastasia explains Baikonur launch site latitude constraint]

[STUB: Response maintains professional engineer personality]

\subsubsection{Agent Response Example}

[STUB: Extended voice transcript showing educational explanation quality]

\subsubsection{Capabilities Demonstrated}

\begin{itemize}
    \item Conversational vs tool-calling mode distinction
    \item Specialist knowledge integration (MissionConfig system)
    \item Educational dialogue rooted in orbital mechanics literature review
    \item Character personality consistency
\end{itemize}

\subsection{Voyager Escape Trajectory}
\label{subsec:voyager_escape}

[STUB: Demonstrates advanced physics, hyperbolic trajectories, character differentiation]

\subsubsection{User Interaction}

[STUB: After routing to Voyager Space: "Show me how Voyager escaped Earth"]

[STUB: Figure - Hyperbolic departure trajectory visualization]

\subsubsection{System Execution}

[STUB: Karl (Voyager specialist) responds in philosophical voice]

[STUB: Creates extreme elliptical orbit - periapsis 200 km, apoapsis 800,000 km]

[STUB: Agent explains escape velocity physics: v\_escape = sqrt(2$\mu$/r) = 11.2 km/s]

[STUB: Voice synthesis at speed 0.9 (contemplative pacing from KarlVoiceSettings)]

\subsubsection{Agent Response Example}

[STUB: Transcript showing Karl's poetic, Sagan-like character: "humanity's first steps beyond the solar system..."]

\subsubsection{Capabilities Demonstrated}

\begin{itemize}
    \item All three mission specialists functional (Mission Control, Anastasia, Karl)
    \item Character voice differentiation (technical vs philosophical personalities)
    \item Physics range: LEO to interplanetary escape
    \item Advanced orbital mechanics concepts from literature review
\end{itemize}
