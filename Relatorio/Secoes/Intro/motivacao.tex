\section{Motivation}
\label{sec:motivation}

For decades, popular media and speculative fiction have envisioned futuristic interfaces for exploration and control, from holographic command centers to immersive planetary navigation tools. Films such as \textit{Minority Report} (2002) and \textit{Iron Man} (2008) popularized visions of humans interacting with vast information systems through gestures, speech, and spatial manipulation. These visions were once confined to science fiction, but today, the convergence of Augmented Reality (AR), Virtual Reality (VR), and Artificial Intelligence (AI) is bringing such interfaces into the realm of technological feasibility.

In particular, the past few years have seen rapid advances in consumer-grade AR/VR hardware. Devices like the Meta Quest and Apple Vision Pro represent significant milestones in accessibility and visual fidelity, enabling immersive environments that are no longer confined to laboratory research or elite applications. The implications for interface design, interaction paradigms, and knowledge acquisition are profound. AR and VR are no longer speculative technologies, they are present, evolving, and increasingly democratized.

Concurrently, the emergence of generative AI and language-based agents has introduced a paradigm shift in how humans interact with complex systems. Large Language Models (LLMs), such as those powering conversational agents, can now interpret natural language, generate multimodal content, and coordinate sequences of actions across software environments. This represents a departure from deterministic, rule-based systems toward stochastic and adaptive workflows, where agents interpret intention, negotiate uncertainty, and build dynamically responsive experiences.

When these technologies - AR/VR and generative agents - are combined, they form the foundation for a new kind of interface: one that is spatial, conversational, and adaptive. Such interfaces do not rely on code or static menus; they respond to voice, gesture, and embodied input. They transform abstract data into manipulable space, and procedural complexity into natural dialogue.

This is particularly relevant in the domain of education. Traditional educational systems remain bound to text, diagrams, and symbolic representation. While these tools are powerful, they often fall short when applied to fields that are inherently spatial, dynamic, or non-intuitive. Orbital mechanics, for example, involves motion through three-dimensional space governed by non-linear physical laws. Launch trajectories, gravitational slingshots, inclination changes, these are difficult to visualize and even harder to intuit.

In this context, immersive simulation becomes more than a visual aid: it becomes a cognitive bridge. A learner can rotate a globe, speak a question, and witness a launch trajectory materialize. They can observe orbits evolve in real time, ask about inclinations or transfer windows, and receive explanations grounded in physics. Education becomes experiential, a process of exploration rather than instruction.

Moreover, generative agents provide a layer of accessibility that is historically absent in technical domains. They can guide the learner, interpret vague queries, correct misconceptions, and explain phenomena in adaptive ways. They act as intelligent mediators between curiosity and formal knowledge.

Given these technological conditions, the maturity of AR/VR, the rise of stochastic AI agents, and the persistent limitations of traditional educational media, this project is motivated by a clear opportunity: to construct a new type of educational experience. One that is not constrained by interface conventions, disciplinary jargon, or static presentation. One that invites the user to learn by seeing, asking, moving, and listening.

The convergence of embodied interaction and generative intelligence allows for a simulation system that is not only technically rigorous, but experientially meaningful. It enables a form of learning in which the abstract becomes tangible, the distant becomes near, and the user is placed at the center of the scientific process. This project emerges from the belief that space education, and scientific education more broadly, can and must evolve to meet the possibilities of our time.