\section{Organisation}
\label{sec:organisation}

This work is organised into three main chapters, each addressing a distinct aspect of the project. The breakdown is as follows:

\begin{itemize}
    \item \textbf{Chapter \ref{sec:introduction}: Introduction.} This chapter sets the stage for the research and development.
    \begin{itemize}
        \item \textit{Motivation} (\S\ref{sec:motivation}): Presents the core argument that the convergence of Augmented Reality and generative AI enables a new, more intuitive paradigm for educational interfaces.
        \item \textit{Objectives} (\S\ref{sec:objetivos}): Defines the project's specific, actionable goals, centered on the development and evaluation of an interactive, agent-guided simulation platform.
    \end{itemize}

    \item \textbf{Chapter \ref{sec:lit_review}: Literature Review.} This chapter provides the theoretical and technical foundation for the work by reviewing three key domains.
    \begin{itemize}
        \item \textit{Augmented and Virtual Reality} (\S\ref{sec:ar_vr_review}): Reviews the evolution of immersive hardware and software ecosystems and establishes their pedagogical value for spatial learning.
        \item \textit{Generative Agents} (\S\ref{sec:generative-agents}): Defines the architecture of modern LLM-powered agents, detailing their ability to use planning, memory, and external tools to reason through and execute complex tasks.
        \item \textit{Orbital Mechanics} (\S\ref{sec:orbital_mechanics}): Outlines the fundamental physics of celestial motion, including the two-body problem, classical orbital elements, and impulsive maneuvers, which form the mathematical basis for the simulation.
    \end{itemize}

    \item \textbf{Chapter \ref{sec:methodology}: Methodology.} This chapter details the practical design, implementation, and assessment of the project.
    \begin{itemize}
        \item \textit{System Architecture and Data Flow} (\S\ref{sec:system_architecture}): Describes the end-to-end fluxogram of the system, illustrating how user voice input is captured, processed by the agent, and rendered in the augmented reality simulation in a continuous loop.
        \item \textit{Core Component Implementation} (\S\ref{sec:core_components}): Details the specific development plan and tools for the two primary modules: the Generative Agent ("Brain") and the Simulation and AR Visualisation ("World").
        \item \textit{Evaluation Plan} (\S\ref{sec:evaluation_plan}): Defines the two-pronged approach for assessment, covering the technical validation of the system's performance and a qualitative user study to gauge its potential as an effective educational tool.
    \end{itemize}
\end{itemize}