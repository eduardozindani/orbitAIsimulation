\section{Organisation}
\label{sec:organisation}

This work is organised into five main chapters, each addressing a distinct aspect of the project. The breakdown is as follows:

\begin{itemize}
    \item \textbf{Chapter \ref{sec:introduction}: Introduction.} This chapter sets the stage for the research and development.
    \begin{itemize}
        \item \textit{Motivation} (\S\ref{sec:motivation}): Presents the core argument that the convergence of immersive reality technologies—particularly Virtual Reality—and generative AI enables a new, more intuitive paradigm for educational interfaces.
        \item \textit{Objectives} (\S\ref{sec:objetivos}): Defines the project's specific, actionable goals, centered on the development and demonstration of an interactive, agent-guided simulation platform.
    \end{itemize}

    \item \textbf{Chapter \ref{sec:lit_review}: Literature Review.} This chapter provides the theoretical and technical foundation for the work by reviewing three key domains.
    \begin{itemize}
        \item \textit{Augmented and Virtual Reality} (\S\ref{sec:ar_vr_review}): Reviews the evolution of immersive hardware and software ecosystems and establishes their pedagogical value for spatial learning.
        \item \textit{Generative Agents} (\S\ref{sec:generative-agents}): Defines the architecture of modern LLM-powered agents, detailing their ability to use planning, memory, and external tools to reason through and execute complex tasks.
        \item \textit{Orbital Mechanics} (\S\ref{sec:orbital_mechanics}): Outlines the fundamental physics of celestial motion, including the two-body problem, classical orbital elements, and impulsive maneuvers, which form the mathematical basis for the simulation.
    \end{itemize}

    \item \textbf{Chapter \ref{sec:methodology}: Methodology.} This chapter details the design philosophy, system architecture, and technical implementation approach.
    \begin{itemize}
        \item \textit{Design Philosophy and Approach} (\S\ref{sec:design_philosophy}): Establishes the iterative, prototype-driven development methodology guided by principles of modularity and exploratory research into novel human-computer interaction paradigms.
        \item \textit{System Architecture and Data Flow} (\S\ref{sec:system_architecture}): Describes the end-to-end interaction cycle, illustrating how user voice input flows through speech recognition, agent reasoning, tool execution, physics simulation, and VR rendering in a continuous loop.
        \item \textit{Mixed Reality Design Rationale} (\S\ref{sec:mr_rationale}): Explains the pedagogical reasoning for focusing exclusively on immersive VR for orbital mechanics education, with discussion of why AR passthrough was not implemented and remains future work, addressing the mismatch between cosmic-scale phenomena and room-scale spatial contexts.
        \item \textit{Technical Implementation} (\S\ref{sec:core_components}): Summarizes the integration of conversational AI (OpenAI GPT-4.1, ElevenLabs voice synthesis), Unity 3D physics simulation, and Meta Quest 3 deployment with character-driven specialist personas.
        \item \textit{Core Module Implementation} (\S\ref{sec:core_implementation}): Details the four primary subsystems—Agent System, Orbital Physics Simulation, Voice Integration Pipeline, and Virtual Reality Environment—with emphasis on their educational design rationale and how each component supports the learning objectives.
        \item \textit{Development and Version Control} (\S\ref{sec:version_control}): Documents the systematic development practices using GitHub for version control, branching strategies, and iterative refinement cycles.
        \item \textit{Validation Strategy} (\S\ref{sec:evaluation_plan}): Outlines the approach for validating system functionality through complete interaction scenarios, with implementation made available for independent verification and replication studies.
    \end{itemize}

    \item \textbf{Chapter \ref{chap:results}: Results and Demonstration.} This chapter demonstrates the complete functional platform through an actual user learning journey, validating all six specific objectives through integrated scenarios.
    \begin{itemize}
        \item \textit{Entering the Orbital Environment} (\S\ref{sec:entering_environment}): Documents the initial Hub arrival experience, including the introductory cutscene, spatial orientation in VR, and first interaction with Mission Control.
        \item \textit{Learning Through Mission Specialist Dialogue: The ISS Circular Orbit} (\S\ref{sec:iss_learning}): Demonstrates the complete pedagogical cycle of conceptual question, specialist consultation, hands-on orbit creation, VR observation, and time-accelerated visualization using the ISS mission as a concrete example.
        \item \textit{Exploring Orbital Geometry: Elliptical Orbits and Eccentricity} (\S\ref{sec:elliptical_exploration}): Shows iterative orbit refinement through Hubble specialist consultation, demonstrating how learners progress from circular to elliptical geometries and observe speed variation through immersive visualization.
        \item \textit{Conceptual Extension: Escape Trajectories and Mission Context} (\S\ref{sec:escape_concept}): Illustrates theoretical dialogue mode through Voyager specialist consultation on interplanetary escape trajectories, validating the platform's flexibility to support conceptual learning without requiring hands-on manipulation.
        \item \textit{System Integration and Technical Validation} (\S\ref{sec:system_integration}): Presents quantitative performance metrics (tool execution reliability, voice pipeline latency, VR rendering stability, physics accuracy validation) and complete system demonstration evidence through continuous video recording.
        \item \textit{Implementation Availability} (\S\ref{sec:implementation_availability}): Documents the publicly available implementation, including source code organization, deployment procedures, and reproducibility guidelines for verification studies.
    \end{itemize}

    \item \textbf{Chapter \ref{chap:conclusion}: Conclusion.} This chapter synthesizes the thesis contribution, reflecting on what was demonstrated, what insights emerged about spatial educational interfaces, what limitations bound the validation scope, what research trajectories this work enables, and whether the central feasibility question has been answered.
\end{itemize}
