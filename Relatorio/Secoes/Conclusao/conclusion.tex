This work demonstrates that immersive virtual reality combined with generative AI agents enables a new mode of scientific education. The platform fulfills its six objectives: physics-consistent simulation grounded in vis-viva equations, natural language interpretation via OpenAI GPT-4 tool-calling, multimodal VR interaction through Quest 3 voice input, real-time coherence sustained at 90 Hz, modular architecture supporting independent scenes, and reproducible implementation. Validation occurred through complete interaction scenarios where learners progress from uncertainty (\textit{``what's a good altitude?''}) to informed design (\textit{``I'll build a circular orbit at 422 km''}) via dialogue with specialist agents, hands-on creation, and immersive observation.

This work is fundamentally a \textbf{feasibility demonstration}, not a pedagogical effectiveness study. The central research question asks: \textit{Is it operationally possible to combine immersive VR, generative AI agents, and voice interaction to create a functional educational platform for orbital mechanics?} The answer, validated through working prototype and complete interaction scenarios, is affirmative. The platform functions reliably, integrates complex technologies coherently, and executes the intended learning cycle. Whether this approach proves educationally superior to traditional methods requires controlled studies with learners—a research direction this work enables but does not conduct.

The pedagogical architecture distinguishes this platform. Orbital mechanics is not presented as declarative knowledge but as spatial phenomena to be designed, questioned, and inhabited. When a learner asks about altitude selection, CAPCOM routes them to ISS specialist who explains engineering rationale (atmospheric drag versus launch efficiency, accessibility from launch sites) and validates their reasoning. The learner then creates the orbit, observes the trajectory, and accelerates time to witness the orbital period. This cycle—consultation, creation, observation—validated across three scenarios (ISS circular, Hubble elliptical, Voyager escape) structures learning as negotiated discovery.

This work reveals insights about educational interfaces in spatial computing. Embodiment transforms comprehension: learners inhabit orbital phenomena rather than observe abstractions, witnessing speed variation at periapsis rather than calculating it. Conversation enables discovery: specialist agents interpret vague questions and provide context-appropriate guidance, enabling education as meaning-making. Modularity enables extension: architectural separation between intelligence (OpenAI), simulation (Unity), and rendering (Quest 3) allows new missions in five steps without core modifications. Reproducibility enables verification: releasing the complete implementation enables independent researchers to audit technical claims, replicate scenarios, and investigate alternative configurations without reimplementation costs.

Significant questions remain unanswered. This work demonstrates operational feasibility but not learning effectiveness through controlled studies. Does conversation improve conceptual understanding compared to traditional instruction? Does immersive visualization enhance retention? These require longitudinal studies with control groups. The demonstration validated one learning path; alternative pedagogical sequences remain unexplored. The voice pipeline depends on external APIs, creating dependencies unsuitable for offline deployment. The physics implements two-body dynamics validated against mission parameters but does not support perturbations or impulsive maneuvers required for advanced planning. These limitations define the research frontier this work establishes.

The platform establishes foundation for research trajectories spanning pedagogical studies (controlled effectiveness experiments, adaptive modeling, collaborative multiplayer), technical extensions (SGP4 propagation, local LLM deployment, AR passthrough), and cross-domain applications (atmospheric flight dynamics, thermodynamic cycles, circuit analysis). These domains share the challenge of making abstract phenomena spatially intuitive through the same architecture: conversational agents orchestrating physics simulation via tool-calling in immersive VR.

This work demonstrates feasibility. The central question—whether immersive VR combined with conversational AI can serve as an educational platform for orbital mechanics—is answered affirmatively through operational validation. The system functions reliably, integrates complex technologies coherently, and executes the pedagogical cycle from conceptual uncertainty to informed design. The implementation is publicly released under MIT License with modular architecture and comprehensive documentation, enabling reproducibility and extension. Whether this paradigm proves educationally superior to traditional instruction remains an open question requiring controlled studies—this work establishes that it is technically possible.
