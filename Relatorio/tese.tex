%%% Exemplo de utilização da classe ITA
%%%
%%%   por        Fábio Fagundes Silveira   -  ffs [at] ita [dot] br
%%%              Benedito C. O. Maciel     -  bcmaciel [at] ita [dot] br
%%%              Giovani Volnei Meinertz   -  giovani [at] ita [dot] br
%%%    	         Hudson Alberto Bode       -  bode [at] ita [dot]br
%%%    	         P. I. Braga de Queiroz    -  pi [at] ita [dot] br
%%%    	         Jorge A. B. Gripp         -  gripp [at] ita [dot] br
%%%    	         Juliano Monte-Mor         -  jamontemor [at] yahoo [dot] com [dot] br
%%%    	         Tarcisio A. B. Gripp      -  tarcisio.gripp [at] gmail [dot] com
%%%
%%%   Versão para overleaf:
%%%   por        Alejandro A. Rios Cruz    - aarc.88@gmail.com
%%%              Saulo Gómez               - sagomezs@unal.edu.co
%%%              Ocimar Santos             - ocimar.acad@gmail.com
%%%
%%%   Template disponibilizado em:
%%%              Overleaf: https://pt.overleaf.com/latex/templates/thesis-template-aeronautics-institute-of-technology-ita/yhfrqqydpygk
%%%
%%%   Contribuia você também!
%%%              GitHub:   https://github.com/gabriellnuness/Template_Thesis_ITA
%%%
%%%  IMPORTANTE: O texto contido neste exemplo nao significa absolutamente nada.  :-)
%%%              O intuito aqui eh demonstrar os comandos criados na classe e suas
%%%              respectivas utilizacoes.
%%%
%%%  Tese.tex  2016-08-25
%%%  $HeadURL: http://www.apgita.org.br/apgita/teses-e-latex.php $
%%%
%%% ITALUS
%%% Instituto Tecnológico de Aeronáutica --- ITA, Sao Jose dos Campos, Brasil
%%%                   http://groups.yahoo.com/group/italus/
%%% Discussion list: italus {at} yahoogroups.com
%%%
%++++++++++++++++++++++++++++++++++++++++++++++++++++++++++++++++++++++++++++++
% Para alterar o TIPO DE DOCUMENTO, preencher a linha abaixo \documentclass[?]{?}
%   \documentclass[tg]{ita}			= Trabalho de Graduacao
%   \documentclass[tgfem]{ita}	= Para Engenheiras
%   								msc     		= Dissertacao de Mestrado
%   								mscfem   		= Para Mestras
%   								dsc      		= Tese de Doutorado
%   								dscfem   		= Para Doutoras
%   								quali    		= Exame de Qualificacao
%   								qualifem 		= Exame de Qualificacao para Doutoras
% Para 'Draft Version'/'Versao Preliminar' com data no rodape, adicionar 'dv':
%   \documentclass[dsc, dv]{ita}
% Para trabalhos em Inglês, adicionar 'eng':
%   \documentclass[dsc, eng]{ita}
%		\documentclass[dsc, eng, dv]{ita}
%++++++++++++++++++++++++++++++++++++++++++++++++++++++++++++++++++++++++++++++
\documentclass[tg, eng]{ita}    % ITA.cls based on standard book.cls
% Quando alterar a classe, por exemplo de [msc] para [msc, eng]) rode mais uma vez o botão BUILD OUTPUT caso haja erro
\usepackage{ae}
\usepackage{graphicx}
\usepackage{epsfig}
\usepackage{amsmath}
\usepackage{amssymb}
\usepackage{subfig}
% Add these to your main .tex file's preamble if they are not already there:
\usepackage{longtable}
\usepackage[table]{xcolor}
\usepackage{amssymb}
\usepackage{colortbl}

\usepackage{multirow}
\usepackage{float}
\usepackage{amsthm}
\usepackage{url}         % formats URL addresses properly
\usepackage{appendix}    % allows appendix section to be included
\usepackage{lscape}      % allows a page to be rendered in landscape mode
\usepackage{multicol}    % allows text in multi columns
\usepackage{cancel}      % needed to show canceled terms in equations
\usepackage{lettrine}
\usepackage{float}
\usepackage{placeins}

% Add these color definitions to your main preamble as well:
\definecolor{phase1}{HTML}{D4E6F1}
\definecolor{phase2}{HTML}{D1F2EB}
\definecolor{phase3}{HTML}{FCF3CF}
\definecolor{phase4}{HTML}{FADBD8}
\definecolor{phase5}{HTML}{E8DAEF}


%HHHHHHHHHHHHHHHHHHHHHHHHHHHHHHHHHHHHHHHHHHHHHHHHHHHHHHHHHHHHHHHHHHHHHHHHHHHHHHHHHHHHHHHHHHHHHHHHHHHHHHHHHHHH
\addbibresource{Referencias/referencias.bib}

%HHHHHHHHHHHHHHHHHHHHHHHHHHHHHHHHHHHHHHHHHHHHHHHHHHHHHHHHHHHHHHHHHHHHHHHHHHHHHHHHHHHHHHHHHHHHHHHHHHHHHHHHHHHH
%\usepackage{subfigure}
%\usepackage{subfigmat}
%PACOTEFIGURAS_SE _ERRADO_ESXCLUIR_ACIMA
\usepackage{booktabs}
%PACOTETABELAS_SE _ERRADO_ESXCLUIR_ACIMA
%HHHHHHHHHHHHHHHHHHHHHHHHHHHHHHHHHHHHHHHHHHHHHHHHHHHHHHHHHHHHHHHHHHHHHHHHHHHHHHHHHHHHHHHHHHHHHHHHHHHHHHHHHHHH

%++++++++++++++++++++++++++++++++++++++++++++++++++++++++++++++++++++++++++++++
% Espaçamento padrão de todo o documento
%++++++++++++++++++++++++++++++++++++++++++++++++++++++++++++++++++++++++++++++
\onehalfspacing

%singlespacing Para um espaçamento simples
%onehalfspacing Para um espaçamento de 1,5
%doublespacing Para um espaçamento duplo

%++++++++++++++++++++++++++++++++++++++++++++++++++++++++++++++++++++++++++++++
% Identificacoes (se o trabalho for em inglês, insira os dados em inglês)
% Para entradas abreviadas de Professora (Profa.) em português escreva: Prof$^\textnormal{a}$.
%++++++++++++++++++++++++++++++++++++++++++++++++++++++++++++++++++++++++++++++
\course{Aerospace Engineering}
%\area{Aircraft Design} % Área de concentração na PG (Não utilizado no caso de TG)

% Autor do trabalho: Nome Sobrenome
\authorgender{masc}                     %sexo: masc ou fem
\author{Eduardo Moura}{Zindani}
\itaauthoraddress{Rua H8B, Ap. 235}{12.228-461}{São José dos Campos--SP}

% Titulo da Tese/Dissertação
\title{Educational Orbit Simulation with Generative AI Agentic Workflow and Virtual Reality Visualisation}

% Orientador
\advisorgender{masc}                    % masc ou fem
\advisor{Prof.~Dr.}{Christopher Shneider Cerqueira}{ITA}

% Coorientador (Caso não haja coorientador, colocar ambas as variáveis \coadvisorgender e \coadvisor comentadas, com um % na frente)
%\coadvisorgender{fem}									% masc ou fem
%\coadvisor{Prof$^\textnormal{a}$.~Dr$^\textnormal{a}$.}{Doralice Serra}{OVNI}

% Pró-reitor da Pós-graduação
\bossgender{masc}												% masc ou fem
\boss{Prof.~Dr.}{John von Neumann}

%Coordenador do curso no caso de TG
\bosscoursegender{fem}									% masc ou fem
\bosscourse{Prof.~Dra.}{Maísa de Oliveria Terra}

% Palavras-Chaves informadas pela Biblioteca -> utilizada na CIP
\kwcip{AI}
\kwcip{VR}
\kwcip{Orbit}

% membros da banca examinadora

\examiner{Prof. Dr.}{Alan Turing}{Presidente}{ITA}
\examiner{Prof. Dr.}{Linus Torwald}{}{UXXX}
\examiner{Prof. Dr.}{Richard Stallman}{}{UYYY}
\examiner{Prof. Dr.}{Donald Duck}{}{DYSNEY}
\examiner{Prof. Dr.}{Mickey Mouse}{}{DISNEY}

% Data da defesa (mês em maiúsculo, se trabalho em inglês, e minúsculo se trabalho em português)
\date{20}{June}{2025}

% Número CDU - (somente para TG)
\cdu{???.??}

% Glossario
\makenoidxglossaries % to use glossaries pkg
\frontmatter


\input{PreTextuais/listaabreviaturas.tex}
\input{PreTextuais/listasimbolos.tex}

\begin{document}
% Folha de Rosto e Capa para o caso do TG
\maketitle

% Dedicatoria: Nao esqueca essa secao  ... :-)
\begin{itadedication}
...
\end{itadedication}

% Agradecimentos
\begin{itathanks}
\input{PreTextuais/agradecimentos}
\end{itathanks}

% Epígrafe
\thispagestyle{empty}
\ifhyperref\pdfbookmark[0]{\nameepigraphe}{epigrafe}\fi
\begin{flushright}
\begin{spacing}{1}
\mbox{}\vfill
{\sffamily\itshape
Wake up!,\\
Wake up!.\\}

\end{spacing}
\end{flushright}

% Resumo
\begin{abstract}
\noindent
Métodos educacionais tradicionais frequentemente encontram dificuldades para transmitir conceitos complexos, espaciais e dinâmicos como os da mecânica orbital. A recente convergência entre a Realidade Mista (RM) de consumo e os sofisticados agentes de Inteligência Artificial (IA) Generativa apresenta uma oportunidade para criar um novo paradigma de interfaces de aprendizagem intuitivas e experienciais. Este trabalho detalha o projeto, desenvolvimento e demonstração de uma plataforma educacional interativa para a exploração dos princípios da mecânica orbital. O objetivo principal do sistema é conectar a teoria de leis físicas abstratas à compreensão intuitiva, permitindo que os usuários aprendam por meio da interação corporificada. A metodologia é centrada em uma arquitetura modular que integra dois componentes principais: (1) um "cérebro" agente generativo, impulsionado por Modelos de Linguagem Abrangentes, que interpreta comandos em linguagem natural e atua como um guia educacional especializado; e (2) um "mundo" de simulação em tempo real e visualização imersiva em realidade mista, construído no motor Unity para o Meta Quest 3, que renderiza trajetórias orbitais fisicamente precisas em um espaço tridimensional onde os usuários vivenciam a mecânica orbital de dentro. A plataforma facilita um ciclo de interação multimodal contínuo, onde os comandos de voz do usuário são capturados, processados pelo agente para alterar os parâmetros da simulação e refletidos na visualização imersiva em RV com feedback auditivo conversacional. Este trabalho entrega um protótipo funcional que demonstra uma nova abordagem para a educação científica, transformando dados abstratos em uma experiência manipulável e conversacional para promover uma aprendizagem exploratória e profundamente engajadora. A plataforma é disponibilizada como software de código aberto para permitir validação, adaptação e extensão pela comunidade para diversos contextos educacionais.

\end{abstract}

% Abstract
\begin{englishabstract}
\noindent
Traditional educational methods often struggle to convey complex, spatial, and dynamic concepts such as those found in orbital mechanics. The recent convergence of consumer-grade Augmented Reality (AR) and sophisticated Generative AI agents presents an opportunity to create a new paradigm for intuitive and experiential learning interfaces. This paper details the design, development, and evaluation of an interactive educational platform for exploring the principles of orbital mechanics. The system's primary objective is to bridge the gap between abstract physical laws and intuitive comprehension by enabling users to learn through embodied interaction. The methodology is centered on a modular architecture that integrates two core components: (1) a generative agent "brain," powered by Large Language Models, which interprets natural language commands and acts as an expert educational guide; and (2) a real-time simulation and AR visualization "world," built in the Unity engine for the Meta Quest 3, which renders physically accurate orbital trajectories anchored to the user's environment. The platform facilitates a seamless multimodal interaction loop where a user's voice commands are captured, processed by the agent to alter simulation parameters, and reflected in the AR visualization with conversational auditory feedback. This work delivers a functional prototype that demonstrates a novel approach to science education, transforming abstract data into a manipulable, conversational experience to foster exploratory and deeply engaging learning.

\end{englishabstract}

% Lista de figuras
\listoffigures %opcional

% Lista de tabelas
\listoftables %opcional

% Lista de abreviaturas
\printnoidxglossary[type=\acronymtype,
    title=\listofabbreviationsname,
    toctitle=\listofabbreviationsname,
    sort=standard,
    nonumberlist]
    
% Lista de simbolos
\printnoidxglossary[
    title=\listofsymbolsname,
    toctitle=\listofsymbolsname,
    sort=standard,
    nonumberlist]

% Sumario
\tableofcontents


\mainmatter
% Os capitulos comecam aqui

\chapter{Introduction}
\label{sec:introduction}
\section{Organisation}
\label{sec:organisation}

This work is organised into five main chapters, each addressing a distinct aspect of the project. The breakdown is as follows:

\begin{itemize}
    \item \textbf{Chapter \ref{sec:introduction}: Introduction.} This chapter sets the stage for the research and development.
    \begin{itemize}
        \item \textit{Motivation} (\S\ref{sec:motivation}): Presents the core argument that the convergence of immersive reality technologies—particularly Virtual Reality—and generative AI enables a new, more intuitive paradigm for educational interfaces.
        \item \textit{Objectives} (\S\ref{sec:objetivos}): Defines the project's specific, actionable goals, centered on the development and demonstration of an interactive, agent-guided simulation platform.
    \end{itemize}

    \item \textbf{Chapter \ref{sec:lit_review}: Literature Review.} This chapter provides the theoretical and technical foundation for the work by reviewing three key domains.
    \begin{itemize}
        \item \textit{Augmented and Virtual Reality} (\S\ref{sec:ar_vr_review}): Reviews the evolution of immersive hardware and software ecosystems and establishes their pedagogical value for spatial learning.
        \item \textit{Generative Agents} (\S\ref{sec:generative-agents}): Defines the architecture of modern LLM-powered agents, detailing their ability to use planning, memory, and external tools to reason through and execute complex tasks.
        \item \textit{Orbital Mechanics} (\S\ref{sec:orbital_mechanics}): Outlines the fundamental physics of celestial motion, including the two-body problem, classical orbital elements, and impulsive maneuvers, which form the mathematical basis for the simulation.
    \end{itemize}

    \item \textbf{Chapter \ref{sec:methodology}: Methodology.} This chapter details the design philosophy, system architecture, and technical implementation approach.
    \begin{itemize}
        \item \textit{Design Philosophy and Approach} (\S\ref{sec:design_philosophy}): Establishes the iterative, prototype-driven development methodology guided by principles of modularity and exploratory research into novel human-computer interaction paradigms.
        \item \textit{System Architecture and Data Flow} (\S\ref{sec:system_architecture}): Describes the end-to-end interaction cycle, illustrating how user voice input flows through speech recognition, agent reasoning, tool execution, physics simulation, and VR rendering in a continuous loop.
        \item \textit{Mixed Reality Design Rationale} (\S\ref{sec:mr_rationale}): Explains the pedagogical reasoning for focusing exclusively on immersive VR for orbital mechanics education, with discussion of why AR passthrough was not implemented and remains future work, addressing the mismatch between cosmic-scale phenomena and room-scale spatial contexts.
        \item \textit{Technical Implementation} (\S\ref{sec:core_components}): Summarizes the integration of conversational AI (OpenAI GPT-4.1, ElevenLabs voice synthesis), Unity 3D physics simulation, and Meta Quest 3 deployment with character-driven specialist personas.
        \item \textit{Core Module Implementation} (\S\ref{sec:core_implementation}): Details the four primary subsystems—Agent System, Orbital Physics Simulation, Voice Integration Pipeline, and Virtual Reality Environment—with emphasis on their educational design rationale and how each component supports the learning objectives.
        \item \textit{Development and Version Control} (\S\ref{sec:version_control}): Documents the systematic development practices using GitHub for version control, branching strategies, and iterative refinement cycles.
        \item \textit{Validation Strategy} (\S\ref{sec:evaluation_plan}): Outlines the approach for validating system functionality through complete interaction scenarios, with implementation made available for independent verification and replication studies.
    \end{itemize}

    \item \textbf{Chapter \ref{chap:results}: Results and Demonstration.} This chapter demonstrates the complete functional platform through an actual user learning journey, validating all six specific objectives through integrated scenarios.
    \begin{itemize}
        \item \textit{Entering the Orbital Environment} (\S\ref{sec:entering_environment}): Documents the initial Hub arrival experience, including the introductory cutscene, spatial orientation in VR, and first interaction with Mission Control.
        \item \textit{Learning Through Mission Specialist Dialogue: The ISS Circular Orbit} (\S\ref{sec:iss_learning}): Demonstrates the complete pedagogical cycle of conceptual question, specialist consultation, hands-on orbit creation, VR observation, and time-accelerated visualization using the ISS mission as a concrete example.
        \item \textit{Exploring Orbital Geometry: Elliptical Orbits and Eccentricity} (\S\ref{sec:elliptical_exploration}): Shows iterative orbit refinement through Hubble specialist consultation, demonstrating how learners progress from circular to elliptical geometries and observe speed variation through immersive visualization.
        \item \textit{Conceptual Extension: Escape Trajectories and Mission Context} (\S\ref{sec:escape_concept}): Illustrates theoretical dialogue mode through Voyager specialist consultation on interplanetary escape trajectories, validating the platform's flexibility to support conceptual learning without requiring hands-on manipulation.
        \item \textit{System Integration and Technical Validation} (\S\ref{sec:system_integration}): Presents quantitative performance metrics (tool execution reliability, voice pipeline latency, VR rendering stability, physics accuracy validation) and complete system demonstration evidence through continuous video recording.
        \item \textit{Implementation Availability} (\S\ref{sec:implementation_availability}): Documents the publicly available implementation, including source code organization, deployment procedures, and reproducibility guidelines for verification studies.
    \end{itemize}

    \item \textbf{Chapter \ref{chap:conclusion}: Conclusion.} This chapter synthesizes the thesis contribution, reflecting on what was demonstrated, what insights emerged about spatial educational interfaces, what limitations bound the validation scope, what research trajectories this work enables, and whether the central feasibility question has been answered.
\end{itemize}

\section{Motivation}
\label{sec:motivation}

For decades, popular media and speculative fiction have envisioned futuristic interfaces for exploration and control, from holographic command centers to immersive planetary navigation tools. Films such as \textit{Minority Report} (2002) and \textit{Iron Man} (2008) popularized visions of humans interacting with vast information systems through gestures, speech, and spatial manipulation. These visions were once confined to science fiction, but today, the convergence of Augmented Reality (AR), Virtual Reality (VR), and Artificial Intelligence (AI) is bringing such interfaces into the realm of technological feasibility.

In particular, the past few years have seen rapid advances in consumer-grade AR/VR hardware. Devices like the Meta Quest and Apple Vision Pro represent significant milestones in accessibility and visual fidelity, enabling immersive environments that are no longer confined to laboratory research or elite applications. The implications for interface design, interaction paradigms, and knowledge acquisition are profound. AR and VR are no longer speculative technologies, they are present, evolving, and increasingly democratized.

Concurrently, the emergence of generative AI and language-based agents has introduced a paradigm shift in how humans interact with complex systems. Large Language Models (LLMs), such as those powering conversational agents, can now interpret natural language, generate multimodal content, and coordinate sequences of actions across software environments. This represents a departure from deterministic, rule-based systems toward stochastic and adaptive workflows, where agents interpret intention, negotiate uncertainty, and build dynamically responsive experiences.

When these technologies - AR/VR and generative agents - are combined, they form the foundation for a new kind of interface: one that is spatial, conversational, and adaptive. Such interfaces do not rely on code or static menus; they respond to voice, gesture, and embodied input. They transform abstract data into manipulable space, and procedural complexity into natural dialogue.

This is particularly relevant in the domain of education. Traditional educational systems remain bound to text, diagrams, and symbolic representation. While these tools are powerful, they often fall short when applied to fields that are inherently spatial, dynamic, or non-intuitive. Orbital mechanics, for example, involves motion through three-dimensional space governed by non-linear physical laws. Launch trajectories, gravitational slingshots, inclination changes, these are difficult to visualize and even harder to intuit.

In this context, immersive simulation becomes more than a visual aid: it becomes a cognitive bridge. A learner can rotate a globe, speak a question, and witness a launch trajectory materialize. They can observe orbits evolve in real time, ask about inclinations or transfer windows, and receive explanations grounded in physics. Education becomes experiential, a process of exploration rather than instruction.

Moreover, generative agents provide a layer of accessibility that is historically absent in technical domains. They can guide the learner, interpret vague queries, correct misconceptions, and explain phenomena in adaptive ways. They act as intelligent mediators between curiosity and formal knowledge.

Given these technological conditions, the maturity of AR/VR, the rise of stochastic AI agents, and the persistent limitations of traditional educational media, this project is motivated by a clear opportunity: to construct a new type of educational experience. One that is not constrained by interface conventions, disciplinary jargon, or static presentation. One that invites the user to learn by seeing, asking, moving, and listening.

The convergence of embodied interaction and generative intelligence allows for a simulation system that is not only technically rigorous, but experientially meaningful. It enables a form of learning in which the abstract becomes tangible, the distant becomes near, and the user is placed at the center of the scientific process. This project emerges from the belief that space education, and scientific education more broadly, can and must evolve to meet the possibilities of our time.
\section{Objectives}
\label{sec:objetivos}

\subsection*{General Objective}

To develop an interactive, agent-guided simulation platform that enables users to explore and understand orbital mechanics through embodied interaction, combining natural language dialogue and real-time virtual reality visualizations.

\subsection*{Specific Objectives}

\begin{enumerate}
\item Design and implement a simulation environment capable of rendering orbital trajectories in real time, grounded in physics-consistent models implementing Keplerian two-body dynamics.
\item Integrate a generative agent capable of interpreting natural language input, translating it into simulation parameters, and guiding the user through explanations and interactions.
\item Enable multimodal interaction by combining voice commands, spatial presence, and mixed reality visual feedback (VR immersion and AR passthrough capabilities) to create a seamless and intuitive user experience optimized for learning orbital mechanics concepts.
\item Ensure that all components of the system, simulation and agent, function coherently and communicate reliably in real time.
\item Create a system architecture that is modular and extensible, allowing for future expansion to other celestial bodies, educational modules, or mission types.
\item Validate system functionality through complete interaction scenarios and provide reproducible implementation documentation, enabling independent verification of technical feasibility claims.
\end{enumerate}
\clearpage

\chapter{Literature Review}
\label{sec:lit_review}
\section{Augmented and Virtual Reality in Immersive Educational Simulation Systems}
\label{sec:ar_vr_review}

Augmented Reality (AR) and Virtual Reality (VR) are complementary immersive technologies that enrich or replace a user’s perception of the world. AR overlays digital content onto the real environment in real-time, allowing virtual objects to coexist with physical surroundings \cite{billinghurst2015survey}. In contrast, VR completely immerses the user in a fully synthetic, computer-generated environment, blocking out the physical world. Milgram’s classic “Reality-Virtuality” continuum illustrates these as end-points: AR lies near the real-world end (mixing virtual content with reality), whereas VR occupies the extreme virtual end with an entirely simulated world \cite{milgram1994}. In essence, AR adds to the user’s real-world experience, while VR transposes the user into an interactive virtual scene. Both technologies share common roots in decades of research and development. The term augmented reality was first coined by Caudell and Mizell (1992) in the context of assisting Boeing manufacturing with see-through displays \cite{caudell1992}. A few years later, Azuma’s influential survey defined AR by three key characteristics: combining real and virtual content, interactive operation in real time, and accurate 3D registration of virtual objects in the physical world \cite{azuma1997, billinghurst2015survey}. VR, meanwhile, has been long conceptualized as achieving presence – the feeling of “being there” in a virtual environment – by engaging multiple senses with responsive 3D graphics and audio \cite{johnson2018}. Modern definitions emphasize that VR provides immersive first-person experiences where users can interact with simulated worlds as if they were real, inducing a strong sense of presence and agency within the virtual scene.

\subsection{Hardware Evolution:}
AR and VR technologies have evolved rapidly, enabling consumer-grade devices that support realistic immersive experiences. While early head-mounted displays date back to the 1960s (e.g., Sutherland’s Sword of Damocles), the 2010s marked a turning point with modern devices. On the VR front, the Oculus Rift prototype (2010) by Palmer Luckey re-ignited interest with a wide field of view and affordable design. Crowdfunded in 2012 and acquired by Facebook in 2014, Oculus released its first consumer headset in 2016, alongside HTC’s Vive, which introduced room-scale tracking. These devices brought high-fidelity visuals and motion tracking to mainstream audiences.

The next major step came with standalone VR headsets. The Oculus/Meta Quest series, starting in 2019, integrated processing and inside-out tracking directly into the headset. Quest 2 (2020) and Quest 3 (2023) improved resolution, optics, and added passthrough AR capabilities \cite{ruth2024}. In parallel, PC-based headsets like the Valve Index and Varjo pushed the fidelity frontier for gaming and enterprise simulation.

AR hardware followed a distinct trajectory. Initial systems used handheld or laptop setups, but the release of Microsoft’s HoloLens in 2016 marked the arrival of self-contained AR headsets with spatial mapping and inside-out tracking. Magic Leap One (2018) added novel display technologies \cite{billinghurst2015survey}, while consumer experiments like Google Glass (2013) explored heads-up interfaces before being discontinued in 2023 \cite{ruth2024}. 

Smartphones played a critical role in scaling AR adoption. Apps like Pokémon GO (2016) introduced mainstream users to AR through camera overlays. ARKit (Apple) and ARCore (Google), launched in 2017, enabled mobile AR with motion and depth tracking \cite{vieyra2018}.

Most recently, the line between AR and VR is blurring. Apple’s Vision Pro (announced 2023) merges high-resolution VR with passthrough AR, positioning itself as a “spatial computer.” With features like dual 4K displays and hand/eye tracking, it may represent a watershed moment for XR despite its premium price \cite{ruth2024}.

As of 2025, the hardware ecosystem spans from mobile-based AR apps to advanced mixed reality headsets, forming a robust toolbox for immersive educational simulations.

\subsection{Software Ecosystems and Frameworks:} Alongside hardware, a mature software ecosystem has enabled rapid development of immersive simulations. Modern game engines such as Unity and Unreal Engine have become the de facto platforms for AR/VR content creation. These engines provide high-performance 3D graphics rendering, physics simulation, and cross-platform deployment, greatly simplifying the creation of interactive virtual environments. Unity, for example, offers an entire XR development toolkit (with support for VR headsets and AR through packages like AR Foundation) that abstracts away device-specific details and allows developers to build an application once and deploy across multiple headsets \cite{atta2022}. Unreal Engine likewise includes integrated support for VR rendering and AR (via ARKit/ARCore plugins), making high-fidelity visualization accessible to developers in academia and industry.

For mobile AR, platform-specific frameworks are key. Apple’s ARKit (introduced in iOS 11, 2017) and Google’s ARCore (for Android, 2017) brought advanced AR capabilities to hundreds of millions of smartphones \cite{vieyra2018}. These software development kits handle real-time tracking of the device’s position, surface detection, lighting estimation, and more, allowing apps to place and persist virtual objects in the user’s environment. Thanks to ARKit/ARCore, an educator can deploy an AR simulation on standard tablets or phones – for instance, letting students point an iPad at a textbook and see 3D molecules or physical field lines appear “attached” to the pages. On the web, the WebXR API has emerged as a W3C standard enabling AR and VR experiences to run directly in web browsers using JavaScript \cite{webxr2021}. WebXR (successor to earlier WebVR/WebAR efforts) allows an immersive educational module to be accessed with a simple URL, lowering the barrier to entry (no app install required) and ensuring compatibility across different devices (from VR headsets to phones). This is particularly relevant for broad educational deployments, where web-based delivery can be more practical. Complementing these are various supporting frameworks: for example, libraries for spatial mapping, hand tracking, and user interaction (e.g. Microsoft’s Mixed Reality Toolkit for Unity, or Vuforia for image-target AR) which provide higher-level tools for common AR/VR interactions. There are also open standards like OpenXR (released by the Khronos Group in 2019) that unify the interface to VR/AR hardware – a developer can write code once against OpenXR and run on any compliant headset (Oculus, SteamVR, Windows Mixed Reality, etc.), which is increasingly adopted by engines and platforms. In summary, the software landscape – from powerful 3D engines to AR phone toolkits and web standards – has matured to a point that immersive educational simulations can be built with relatively modest effort compared to a decade ago. This thesis will leverage these tools to construct its simulation system, ensuring it is built on proven, widely supported technology.

\subsection{Use Cases in Education:} 
AR and VR have shown strong potential to enhance learning, particularly in subjects involving abstract or spatial concepts. Their core strength lies in making the invisible visible and the abstract tangible. In physics education, for instance, VR has helped students visualize and manipulate 3D vectors, improving understanding of vector addition and spatial relationships \cite{campos2022}. Studies show that such immersive tools can boost engagement and deepen comprehension of abstract STEM topics like electromagnetism or geometry through interactive, risk-free exploration \cite{campos2022, johnson2018}.

In astronomy and aerospace, where scales are far beyond human experience, immersive technologies offer unique advantages. VR enables virtual field trips through space — letting students stand on Mars or orbit planets — providing an intuitive grasp of scale and distance. Learners can explore the solar system with accurate proportions, making complex spatial relationships (like planetary distances or ring sizes) more comprehensible \cite{atta2022}. Astrophysical phenomena such as orbital mechanics and black hole dynamics are also made more accessible through interactive VR visualizations.

In aerospace engineering, VR and AR are increasingly used for hands-on training. Beyond traditional flight simulators, modern VR platforms allow students to perform simulated pre-flight inspections, engine maintenance, or spacecraft docking. Vaughn College, for example, uses VR for aviation trainees to practice inspecting and assembling parts, reinforcing mechanical familiarity before real-world exposure. Similarly, Atta et al. (2022) created a virtual “space lab” where students assemble a CubeSat in a simulated cleanroom, boosting their understanding of subsystem configuration through direct interaction and gamified tasks \cite{atta2022}.

AR complements this by overlaying digital instructions on real-world hardware. NASA’s Project Sidekick exemplifies this: astronauts use HoloLens headsets aboard the ISS to receive real-time, spatially anchored maintenance guidance \cite{NASA2015}. In classrooms, AR enables students to interact with 3D models of rockets or overlay CAD designs onto physical parts, enriching theoretical lessons with live, contextual visualization \cite{milgram1994, atta2022}.


\subsection{Embodiment, Interaction, and Spatial Cognition:} A recurring theme in the educational use of AR/VR is the role of embodied and spatial learning. Immersive technologies engage the human sensorimotor system – users move their bodies to navigate virtual spaces, use gestures to interact with virtual objects, and perceive environments at true scale. This physicality supports cognitive processing by leveraging innate spatial reasoning and muscle memory. The theory of embodied cognition holds that learning is grounded in the body’s interactions with its environment, and AR/VR extend this principle digitally. Johnson-Glenberg (2018) highlights the pedagogical value of 3D gestures: when learners rotate a virtual object or walk through a graph, they build stronger memory links \cite{johnson2018}. Her research shows that full-motion VR, where body movements align with abstract concepts, can deepen understanding and recall. Complementary studies (e.g., Liu et al., 2020) found improved retention when students enacted phenomena physically, and also noted increased presence and agency—factors tied to motivation \cite{johnson2018, campos2022}.

Spatial cognition benefits are also well-documented. VR’s stereoscopic depth and six degrees of freedom help learners perceive complex spatial relationships, vital in subjects like anatomy, geography, and engineering. Students exploring a molecule or a solar system in VR can shift perspective freely, activating spatial memory and supporting what researchers call “situated learning” – knowledge acquired in rich spatial contexts becomes more intuitive and transferable. Campos et al. (2022), for example, found that immersive 3D interaction notably enhanced vector learning tasks requiring spatial reasoning \cite{campos2022}. Similarly, in astronomy, VR’s ability to scale from the Milky Way to Earth provides concrete visualizations of abstract systems (Kersting et al., 2024).

While AR/VR offer compelling tools, they are not magic bullets – user comfort, software complexity, and thoughtful pedagogical integration remain critical \cite{johnson2018}. Still, evidence shows that immersive simulations can enhance traditional teaching, especially for learning goals involving visualization, experimentation, or embodied experience. In the context of this thesis, the implications are clear: AR and VR form a foundational layer. They enable students to interact with simulations of aerospace systems—such as satellites or orbital dynamics—in an intuitive and experiential manner. As hardware becomes lighter and more capable, and software ecosystems more robust, immersive tools are becoming increasingly viable in education. With spatial computing platforms entering mainstream use \cite{ruth2024}, AR and VR are poised not just as delivery platforms but as new paradigms for engaging with knowledge.

\paragraph{Application to This Work}

This thesis leverages the pedagogical strengths of immersive spatial learning through Virtual Reality as the primary modality. While the platform architecture is built on mixed reality hardware (Meta Quest 3) that supports both VR and AR passthrough modes, the educational experience emphasizes full VR immersion for the reasons detailed in Section \ref{sec:mr_rationale}. The literature reviewed here establishes the broader context of immersive educational technologies, while the implementation focuses specifically on VR's capacity to place learners inside coherent spatial environments optimized for understanding orbital mechanics.
\section{Generative Agents}
\label{sec:generative-agents}

Traditional software and simulations have been predominantly \emph{deterministic}—given the same inputs, they yield the same outputs. Modern AI systems built on \emph{generative} models, by contrast, introduce stochasticity and creativity. Large Language Models (LLMs) do not follow hard-coded rules; instead, they sample from probability distributions learned from vast textual corpora. Consequently, an LLM can produce context-dependent, varied responses rather than a single predetermined answer. This marks a paradigm shift from scripted to emergent behaviour.  In recent work, advanced LLMs such as GPT-4 have even outperformed traditional reinforcement-learning agents in complex environments by reasoning through text rather than executing pre-programmed control policies \cite{carrasco2025LLM}.  While stochastic generation entails some unpredictability, it is precisely this creativity that lets \emph{generative agents} adapt to scenarios beyond their designers’ foresight.

At a conceptual level an LLM is a statistical language engine: given a textual history, it predicts the most plausible continuation one word at a time.  Because it is trained on heterogeneous data, a single model can answer coding questions, analyse legal texts, or reason about orbital mechanics when prompted appropriately.  This broad, generative capability underpins the rise of \emph{LLM-powered agents} \cite{anthropic2023agents}.

\textbf{LLM-based agents} are autonomous software entities that embed an LLM as their core “brain.”  An agent senses its environment, reasons about goals, and acts—iteratively—until a task is complete.  Industry definitions describe such an agent as “a system that uses an LLM to reason through a problem, create a plan, and execute that plan with tools” \cite{huang2024agentic,chen2023agents}.  The LLM supplies the reasoning; auxiliary modules provide planning, memory, and tool use \cite{anthropic2023agents}.  Crucially, the agent—not the user—controls the loop: it may decide which function to call, when to revise a plan, or whether to request clarification \cite{openai2023functions}.  Hence an agent is more than a single LLM invocation; it is a continual perceive–think–act cycle.

\subsection*{Architectural Components}

Generative-agent designs typically comprise five interacting elements \cite{anthropic2023agents,huang2024agentic}:

\begin{itemize}
  \item \textbf{Planning and reasoning.}  The agent decomposes high-level goals into actionable steps, often prompting the LLM to produce an internal plan or “chain of thought.”
  \item \textbf{Memory.}  Short-term context (recent turns) and long-term knowledge (summaries or retrieved documents) are stored externally—e.g.\ in a vector database—and injected into prompts as needed.
  \item \textbf{Tool use and APIs.}  Through structured outputs (JSON function calls, shell commands, \emph{etc.}) the agent invokes external tools to compute, query, or effect changes in its environment \cite{openai2023functions}.
  \item \textbf{Iterative control loop.}  The agent cycles through \emph{observe} $\rightarrow$ \emph{reason} $\rightarrow$ \emph{act} $\rightarrow$ \emph{observe}, optionally reflecting or self-critiquing between steps to improve reliability.
  \item \textbf{Autonomy and adaptation.}  Equipped with the above, the agent can switch strategies, recover from errors, and pursue its objective with minimal human micromanagement.
\end{itemize}

\begin{figure}[t]
  \centering
  \includegraphics[width=.8\linewidth]{Imagens/Effective_agents.png}
  \caption{End-to-end agentic workflow from Anthropic’s “Building Effective Agents.”  
           The human issues a query through an \textit{interface}; the LLM asks clarifying
           questions until the task is precise, receives contextual files, iteratively writes
           and tests code against the environment, and finally returns results for display
           \cite{anthropic2023agents}.}
  \label{fig:effective-agents-seq}
\end{figure}

\subsection*{Applications and Relevance}

\begin{itemize}
  \item \textbf{Simulations and interactive worlds.}  Park \emph{et al.}\ created “Generative Agents” that populate a sandbox town with virtual characters who plan, remember, and socially interact—producing emergent storylines never scripted by the developers \cite{park2023generative}.
  \item \textbf{Aerospace guidance and control.}  Carrasco \emph{et al.}\ demonstrated an LLM agent piloting a spacecraft in the \textit{Kerbal Space Program} simulation by iteratively reading textual telemetry and issuing control actions, matching classical controllers without explicit orbital equations \cite{carrasco2025LLM}.
  \item \textbf{Legal reasoning.}  Harvey AI equips law-firm associates with an agent that drafts memos, retrieves precedents, and iteratively refines analyses through dialogue—illustrating agentic workflows in language-dense tasks \cite{chen2023agents}.
  \item \textbf{Education.}  Khan Academy’s \textit{Khanmigo} employs GPT-4 as a Socratic tutor that adapts explanations to each learner, providing hints rather than answers and thereby personalising study sessions at scale \cite{khanmigo2023}.
\end{itemize}
\section{Orbital Mechanics: The Physics of Celestial Motion}
\label{sec:orbital_mechanics}

The intuitive, visual understanding of orbital motion is a primary objective of this project. While the generative agent handles the underlying calculations, a firm grasp of the governing principles is essential to frame the simulation's logic and appreciate its educational value. Orbital mechanics is the study of the motion of bodies under the influence of gravity. For missions in Earth's orbit and for interplanetary trajectories, the foundational principles discovered by Isaac Newton and Johannes Kepler provide a remarkably accurate framework for describing and predicting these celestial paths. This section outlines the core concepts that form the physical and mathematical basis of the simulation system, focusing on the specific orbital regimes and parameters that the platform enables users to explore: circular orbits, elliptical orbits, and escape trajectories.

\subsection{Newtonian Gravity and the Two-Body Problem}

At the heart of all orbital motion lies gravity. In the 17th century, Sir Isaac Newton formulated the Law of Universal Gravitation, stating that any two bodies attract each other with a force proportional to the product of their masses and inversely proportional to the square of the distance between them \cite{Curtis2020}. This is expressed mathematically as:
$$ F = G \frac{m_1 m_2}{r^2} $$
where $F$ is the gravitational force, $G$ is the gravitational constant ($6.674 \times 10^{-11}$ N·m²/kg²), $m_1$ and $m_2$ are the masses of the two bodies, and $r$ is the distance between their centers.

When applied to a satellite orbiting a celestial body like Earth, this law forms the foundation of the \textbf{two-body problem}. This model makes a critical simplifying assumption: it considers only the gravitational force between the satellite and the primary body (e.g., Earth), ignoring perturbations such as atmospheric drag, solar radiation pressure, and gravitational influences from other bodies like the Moon or the Sun \cite{Vallado2013}. While these forces are significant for high-precision, long-term trajectory prediction, the two-body model provides an elegant and highly accurate approximation for foundational analysis and educational purposes. The resulting equation of motion is:
$$ \ddot{\vec{r}} + \frac{\mu}{r^3}\vec{r} = 0 $$
Here, $\vec{r}$ is the position vector of the satellite relative to the primary body, $\ddot{\vec{r}}$ is its acceleration, and $\mu$ (mu) is the standard gravitational parameter of the system. For Earth-orbiting satellites, $\mu = GM_{\text{Earth}} \approx 398{,}600$ km³/s², where $M_{\text{Earth}}$ is Earth's mass.

The solution to this differential equation reveals a profound geometric truth: under the inverse-square law of gravity, the satellite's path must be a \textbf{conic section}—a circle, ellipse, parabola, or hyperbola \cite{Curtis2020}. Which conic section results depends on the satellite's energy and angular momentum. This elegant mathematical result means that all orbital trajectories, from the circular path of the ISS to the hyperbolic escape of Voyager, are governed by the same fundamental physics expressed through different geometric shapes.

\subsection{Kepler's Laws and Orbital Geometry}

Johannes Kepler, working in the early 17th century with observational data from Tycho Brahe, empirically discovered three laws of planetary motion that would later be shown to be direct consequences of Newtonian gravity. These laws provide the geometric and temporal framework for understanding orbits \cite{Bate1971}:

\begin{enumerate}
    \item \textbf{First Law (Law of Orbits):} The orbit of a planet (or satellite) around the Sun (or Earth) is an ellipse, with the central body at one focus. A circle is the special case of an ellipse where both foci coincide.

    \item \textbf{Second Law (Law of Areas):} A line connecting the satellite to the central body sweeps out equal areas in equal times. This means the satellite moves faster when closer to the central body (at periapsis) and slower when farther away (at apoapsis).

    \item \textbf{Third Law (Law of Periods):} The square of the orbital period is proportional to the cube of the semi-major axis. Mathematically: $T^2 \propto a^3$, or more precisely, $T^2 = \frac{4\pi^2}{\mu} a^3$. This law directly relates orbital size to orbital period, explaining why the ISS at 420 km altitude completes an orbit in 92.8 minutes while the Hubble Space Telescope at 540 km takes slightly longer at approximately 95 minutes.
\end{enumerate}

Kepler's laws were empirical observations that Newton later proved mathematically from first principles. Together, they provide both the geometric intuition (ellipses, not circles, are the general case) and quantitative relationships (period depends on altitude) that govern orbital motion.

\subsection{Orbital Regimes: Circular, Elliptical, and Hyperbolic Trajectories}

The shape of an orbit is determined by the satellite's total mechanical energy—the sum of its kinetic energy (from motion) and gravitational potential energy (from position in the gravity field). This energy dictates which conic section describes the trajectory. The platform's simulation implements three fundamental orbital regimes, each representing a different energy state and mission application.

\subsubsection{Circular Orbits: Stable Operational Platforms}

A \textbf{circular orbit} occurs when the satellite's velocity is precisely calibrated so that the centripetal acceleration required for circular motion exactly matches the gravitational acceleration at that altitude. This is the special case where eccentricity $e = 0$.

For a circular orbit at radius $r$ from Earth's center (altitude $h = r - R_{\text{Earth}}$), the required orbital velocity is given by:
$$ v_{\text{circular}} = \sqrt{\frac{\mu}{r}} $$

This relationship shows that orbital speed decreases with altitude: satellites in low Earth orbit (LEO) travel faster than those in higher orbits. For example, the ISS at 420 km altitude orbits at approximately 7.66 km/s, while the Hubble Space Telescope at 540 km altitude travels at approximately 7.59 km/s—slightly slower due to its higher altitude.

Circular orbits are preferred for operational missions requiring predictable, repeating ground tracks and stable altitude. The International Space Station (420 km, 51.6° inclination) and Hubble Space Telescope (540 km, 28.5° inclination) both use circular orbits because their missions benefit from the stability and predictability of constant altitude and speed.

Kepler's Third Law directly determines the orbital period for circular orbits:
$$ T = 2\pi \sqrt{\frac{r^3}{\mu}} $$

This equation explains the relationship between altitude and period. The ISS at 420 km completes 15.5 orbits per day, providing frequent revisit times for Earth observation and crew operations. Hubble at 540 km has a slightly longer period, chosen to balance orbital stability with minimizing atmospheric drag while providing optimal viewing conditions for astronomical observations.

\subsubsection{Elliptical Orbits: Variable Altitude Trajectories}

An \textbf{elliptical orbit} occurs when $0 < e < 1$, where $e$ is the eccentricity. The satellite's altitude and speed vary continuously as it moves around the ellipse. The closest point to Earth is called \textbf{periapsis} (or perigee for Earth orbits), and the farthest point is \textbf{apoapsis} (or apogee). The size of the ellipse is characterized by the \textbf{semi-major axis} $a$, which is half the longest diameter of the ellipse.

The relationship between the semi-major axis, periapsis radius $r_p$, and apoapsis radius $r_a$ is:
$$ a = \frac{r_p + r_a}{2} $$

The eccentricity quantifies how elongated the ellipse is:
$$ e = \frac{r_a - r_p}{r_a + r_p} $$

When $e = 0$, the ellipse becomes a circle ($r_p = r_a$). As $e$ approaches 1, the ellipse becomes increasingly elongated.

The satellite's speed at any point in an elliptical orbit is given by the \textbf{vis-viva equation}, one of the most fundamental relationships in orbital mechanics \cite{Curtis2020}:
$$ v = \sqrt{\mu \left( \frac{2}{r} - \frac{1}{a} \right)} $$

This equation reveals that orbital speed depends on both the current position $r$ and the orbit's overall size $a$. At periapsis, where $r$ is smallest, the satellite moves fastest. At apoapsis, where $r$ is largest, it moves slowest. This speed variation is a direct consequence of Kepler's Second Law: the satellite must move faster when closer to Earth to sweep equal areas in equal times.

Elliptical orbits have important applications. Highly Elliptical Orbits (HEO) are used for communications satellites serving high-latitude regions, as the satellite spends most of its time near apoapsis with excellent visibility over polar regions. Transfer orbits between circular orbits are also elliptical, with the initial circular orbit at periapsis and the target circular orbit at apoapsis.

\subsubsection{Hyperbolic Trajectories: Escaping Earth's Gravity}

A \textbf{hyperbolic trajectory} occurs when $e \geq 1$. Unlike elliptical orbits, which are closed and periodic, hyperbolic trajectories are open curves—the spacecraft approaches Earth, swings around it, and departs, never to return. This regime represents escape from Earth's gravitational influence.

The minimum speed required to achieve escape from Earth's surface is the \textbf{escape velocity}:
$$ v_{\text{escape}} = \sqrt{\frac{2\mu}{r}} $$

At Earth's surface ($r = R_{\text{Earth}} = 6371$ km), this yields approximately 11.2 km/s. Notice that escape velocity is exactly $\sqrt{2}$ times the circular orbital velocity at the same radius—this factor of $\sqrt{2}$ represents the energy difference between a bound circular orbit and an unbound escape trajectory.

For a hyperbolic trajectory, the spacecraft's velocity at any distance is given by a modified vis-viva equation:
$$ v = \sqrt{\mu \left( \frac{2}{r} + \frac{1}{a} \right)} $$

Note the sign change: for hyperbolic orbits, the semi-major axis $a$ is defined as negative, reflecting the fact that the orbit is unbound with positive total energy.

The Voyager spacecraft exemplify hyperbolic escape trajectories. After launch and acceleration to sufficient velocity, they followed hyperbolic paths that carried them beyond Earth's sphere of influence and into interplanetary space. The platform uses Voyager's trajectory to demonstrate the transition from bound elliptical motion to unbound hyperbolic escape, illustrating the fundamental energy threshold that separates orbiting from departing.

\subsection{Orbital Orientation: Inclination and Coverage}

While the size and shape of an orbit (determined by $a$ and $e$) govern its energy and geometry, the \textbf{inclination} determines the orbit's orientation in three-dimensional space. Inclination $i$ is the angle between the orbital plane and a reference plane, typically Earth's equatorial plane. An inclination of $0°$ defines an equatorial orbit, while $90°$ defines a polar orbit that passes directly over both poles.

Inclination is not arbitrary—it is fundamentally constrained by the launch site's latitude and the physics of rotation. When a rocket launches eastward (prograde), it benefits from Earth's rotational velocity, which is maximum at the equator (~465 m/s) and decreases toward the poles. The minimum achievable inclination from a launch site is approximately equal to the site's latitude. For example, launches from Kennedy Space Center (28.5°N) can achieve inclinations of 28.5° or greater, but reaching lower inclinations would require the rocket to perform an energetically expensive plane change maneuver.

This launch constraint explains many mission orbital parameters:
\begin{itemize}
    \item \textbf{ISS (51.6° inclination):} Designed to be accessible from both Kennedy Space Center and the Baikonur Cosmodrome in Kazakhstan (45.6°N). The 51.6° inclination allows Russian Soyuz launches from Baikonur while remaining within reasonable energy budgets for US launches.

    \item \textbf{Hubble (28.5° inclination):} Launched from Kennedy Space Center at the minimum possible inclination, maximizing the rotational velocity assist and minimizing fuel requirements. This low inclination also provides good sky coverage for astronomical observations while avoiding prolonged periods in Earth's shadow.
\end{itemize}

Inclination also determines ground track coverage. An equatorial orbit ($i = 0°$) never passes over polar regions. A polar orbit ($i = 90°$) eventually covers the entire surface as Earth rotates beneath it. Intermediate inclinations provide a balance between coverage and launch efficiency. The platform's mission-specific implementations demonstrate how operational requirements (crew access for ISS, astronomical visibility for Hubble) drive inclination choices.

\subsection{The Educational Foundation for Interactive Exploration}

The physics and mathematics outlined in this section—Newtonian gravity, Kepler's laws, the vis-viva equation, and the geometric properties of conic sections—form the computational foundation of the simulation platform. More importantly, they represent the conceptual framework that users explore through embodied interaction in virtual reality.

Traditional orbital mechanics education presents these concepts through equations on paper and two-dimensional diagrams. Students memorize formulas and solve problems numerically, but the intuitive, spatial understanding of why the ISS orbits at 7.66 km/s or why the Hubble telescope requires a specific altitude and inclination often remains elusive. The three-dimensional geometry of an inclined orbit, the speed variation along an ellipse, and the meaning of escape velocity are fundamentally spatial phenomena that are difficult to internalize from textbooks alone.

The platform's approach inverts this pedagogy. Users begin not with equations but with questions and curiosity: "Show me the ISS orbit." "Why does Hubble orbit where it does?" "How did Voyager leave Earth?" The generative agent translates these natural language queries into the precise orbital parameters described in this section—altitude, eccentricity, inclination—and the Unity simulation engine renders the resulting trajectories as visible, three-dimensional curves in space. Users inhabit the orbital environment, observing how the ISS's 420 km circular orbit compares to Hubble's 540 km orbit, seeing the ellipse stretch as eccentricity increases, watching the hyperbolic escape path diverge from closed elliptical motion.

This section has established the theoretical foundation that makes such exploration both accurate and meaningful. The circular orbits users create are governed by $v = \sqrt{\mu/r}$. The elliptical orbits follow the vis-viva equation. The hyperbolic escapes exceed $v_{\text{escape}} = \sqrt{2\mu/r}$. The platform's educational value rests on this foundation: it translates rigorous astrodynamics into intuitive visual experience, enabling users to build genuine understanding of orbital mechanics through guided exploration rather than rote memorization.



\chapter{Methodology}
\label{sec:methodology}
\section{Design Philosophy and Approach}
\label{sec:design_philosophy}

The development of this project is fundamentally an exploratory research endeavour into a new paradigm of human-computer interaction for educational purposes. Given the innovative and complex nature of integrating generative AI, augmented reality, and embodied interfaces, a rigid, waterfall-style development plan would be inappropriate. Instead, the methodology is guided by a philosophy that embraces iteration and modularity to navigate the technical challenges and discovery process inherent in such work.

The approach is defined by three core principles:
\begin{itemize}
    \item \textbf{Prototype-Driven:} The primary goal is the creation of a functional prototype that demonstrates the feasibility and potential of the proposed system. This approach prioritizes implementing the core functionalities of the user experience over exhaustive feature development, allowing for tangible and testable results that can validate the project's central thesis.
    \item \textbf{Iterative Development:} The project will be built in iterative cycles, following a process of building a core feature, testing its performance and usability, and refining it based on the results. This allows for flexibility in the implementation details, acknowledging that the optimal solutions for agent prompting and user interaction will be discovered and improved upon throughout the development lifecycle.
    \item \textbf{Modular Architecture:} The system is designed as a collection of distinct yet interconnected modules: the generative agent (the "brain") and the simulation and visualisation engine (the "world"). This modularity, a key objective of this project, makes the complex system manageable, facilitates parallel development and testing of components, and ensures the final architecture is extensible for future work.
\end{itemize}

\section{System Architecture and Data Flow}
\label{sec:system_architecture}

This section outlines the high-level system architecture, describing the flow of information between the user, the software components, and the generative agent. This end-to-end process, or "fluxogram," illustrates how the various components work in concert to create a seamless, real-time interactive experience. The architecture is designed as a continuous "perceive-think-act" cycle, mirroring the agentic workflows described in the literature.

The data flow for a single user interaction can be detailed in the following sequence:
\begin{enumerate}
    \item \textbf{User Input:} The interaction begins with the user issuing a voice command, such as speaking a request (e.g., "Show me an orbit with an inclination of 45 degrees"). The microphone on the AR/VR headset records the user's voice.
    \item \textbf{The Agentic Core (Reasoning and Planning):} Within Unity, the core logic orchestrates the agent's reasoning process. The application sends the transcribed user query to the OpenAI API. The generative agent, guided by its engineered prompt, interprets the user's intent, formulates a plan, and identifies the appropriate "tool" to use---a predefined C\# function within the Unity simulation.
    \item \textbf{Simulation and Visualisation:} The agent invokes the corresponding function in the simulation engine, passing the translated parameters (e.g., \(\textit{i} = 45^\circ\), \(\Omega = \text{calculated\_value}\)). The Unity engine calculates the orbital trajectory based on these parameters and renders the path as a 3D visualisation on the Meta Quest 3 in augmented reality, anchored to the user's physical environment.
    
    \item \textbf{Auditory Feedback Loop:} To complete the interaction, the agent generates a textual confirmation or explanation (e.g., "Certainly, here is the 45-degree inclination orbit you requested."). This text is sent to the ElevenLabs API, which synthesizes a natural-sounding voice response that is played back to the user through the headset, providing coherent, conversational feedback \cite{anthropic2023agents}.
\end{enumerate}

\section{Core Component Implementation}
\label{sec:core_components}

The architecture described above is realized through the implementation of two distinct, yet deeply integrated, core components. This section details the specific role, planned tools, and development process for each component, clarifying how they contribute to the project's objectives.

\subsection{The Generative Agent (The "Brain")}
\label{subsec:agent}

\begin{description}
    \item[Role] The generative agent serves as the central intelligence of the system. Its primary role is to act as a natural language interface and reasoning engine, translating the user's high-level, often ambiguous, spoken intent into the precise, deterministic commands required by the simulation engine. It functions as an interactive, educational guide for the user.
    
    \item[Tools] The agent's capabilities will be powered by a suite of Application Programming Interfaces (APIs). The core reasoning and language understanding will be handled by the \textbf{OpenAI API}, leveraging a powerful Large Language Model (LLM) with function-calling capabilities. The agent's voice, providing auditory feedback, will be synthesized using the \textbf{ElevenLabs API}, chosen for its ability to generate high-quality, low-latency speech.
    
    \item[Process] The development will focus on prompt engineering to define the agent's persona and behaviour as an expert aerospace tutor. The agent will be provided with a schema of the available C\# functions within the Unity environment, effectively giving it a set of "tools" it can use. When a user issues a command, the agent will reason about the intent and select the appropriate function to call with the correct parameters. The challenge of LLM hallucination is acknowledged; mitigation strategies, such as providing contextually relevant information within the prompt and validating outputs, will be explored and refined during the iterative development cycles.
\end{description}

\subsection{The Simulation and AR Visualisation (The "World")}
\label{subsec:simulation}

\begin{description}
    \item[Role] This component is responsible for creating a real-time, physically-grounded, and visually intuitive representation of the orbital environment. It must accurately simulate celestial motion and render the results in an interactive, three-dimensional space for the user.
    
    \item[Tools] The simulation and visualisation will be built using the \textbf{Unity} 3D development engine, chosen for its robust cross-platform capabilities and extensive support for XR development. The target hardware for deployment is the \textbf{Meta Quest 3}, selected for its high-resolution, full-colour passthrough, which is ideal for Augmented Reality (AR), and its powerful standalone processing capabilities.
    
    \item[Process] The simulation's physics will be implemented in Unity using C\# scripts. The core logic will be based on the principles of astrodynamics, primarily the two-body problem, to calculate orbital trajectories as described in the foundational literature\cite{Curtis2020}. The primary development goal is an AR experience, where the orbital visualisations are overlaid onto the user's real-world environment. The user can walk around their physical space to view the orbital simulations from different angles. The modular nature of the design, however, allows for the future extension to a fully immersive Virtual Reality (VR) mode within the same application.
\end{description}

\section{Development and Version Control}
\label{sec:version_control}

To ensure a systematic and traceable development process, the project will be managed using modern software engineering practices. The primary tool for this purpose is \textbf{GitHub}, a distributed version control system. All source code, including the C\# scripts for the \textbf{Unity} application, will be stored in a centralized repository on GitHub. This approach provides a complete history of all changes, facilitates branching for experimental features without compromising the stability of the main project, and establishes a foundation for potential future collaboration. Regular commits will document the incremental progress, aligning with the iterative development philosophy outlined in Section \ref{sec:design_philosophy}.

\section{Evaluation Plan}
\label{sec:evaluation_plan}

A critical component of this project is to measure its success, not only as a functional piece of software but also as a potentially effective educational tool. The evaluation plan is therefore designed to address two distinct aspects: the technical validation of the system and a qualitative assessment of its educational potential, directly addressing the final specific objective of this thesis \ref{sec:objetivos}.

\subsection{Technical Validation}
The first phase of evaluation will focus on verifying that the system's components function correctly, reliably, and efficiently. This involves a series of tests to measure key performance indicators:
\begin{itemize}
    \item \textbf{Agent Accuracy:} Testing the generative agent's ability to correctly parse a range of spoken commands and accurately invoke the corresponding simulation functions with the correct parameters.
    \item \textbf{Simulation Fidelity:} Verifying that the rendered orbital trajectories are mathematically correct according to the implemented physics models.
    \item \textbf{System Latency:} Measuring the end-to-end latency, from user voice input to the corresponding visual and auditory feedback, to ensure the interaction feels responsive and seamless.
    \item \textbf{AR Spatial Anchoring:} Assessing the stability and accuracy of the augmented reality orbital visualizations anchored in the user's physical environment.
\end{itemize}

\subsection{Educational Potential Assessment}
The second phase of evaluation aims to gather qualitative data on the platform's potential as a tool for facilitating conceptual understanding of orbital mechanics. This will be achieved through a small-scale, qualitative user study.
\begin{description}
    \item[Participants] A small group of target users (e.g., 3-5 undergraduate students in aerospace engineering or a related field) will be invited to participate.
    \item[Procedure] Each participant will be given a brief introduction to the system's controls and features. They will then be asked to complete a series of exploratory tasks, such as creating a geostationary orbit, visualizing a polar orbit, or performing a Hohmann transfer between two altitudes.
    \item[Data Collection] Following the interactive session, feedback will be collected through a semi-structured interview and a short questionnaire. Questions will focus on the user's experience regarding engagement, ease of use, and perceived value. Specifically, participants will be asked whether the platform helped them visualize or understand orbital concepts more intuitively compared to traditional learning methods like textbooks and diagrams.
\end{description}
The goal of this assessment is not to achieve statistical significance, but to gather rich, qualitative insights that can validate the educational premise of the project and inform future improvements.
\section{Implementation Plan}
\label{sec:implementation_plan}

This section outlines the implementation strategy for the agent-guided orbital mechanics simulation platform. The development is structured in four focused phases that build from conversational intelligence to immersive virtual reality experience. The approach prioritizes fundamental educational capabilities over feature complexity, ensuring each phase delivers clear value and builds toward a complete learning platform.

The plan reflects the principles established in Section \ref{sec:design_philosophy}: prototype-driven development, iterative refinement, and modular architecture. Each phase produces a testable, working system that integrates with subsequent work without requiring architectural changes.

\subsection{System Architecture Overview}
\label{subsec:system_overview}

The platform provides two complementary learning experiences:

\paragraph{The Hub (Mission Control)}
The Hub serves as the primary workspace where users create and explore custom orbital trajectories. Users interact with Mission Control—a conversational guide with a visionary, encouraging personality—to build circular or elliptical orbits around Earth. The Hub emphasizes focused, hands-on learning by displaying a single orbit at a time, allowing users to understand each configuration deeply before moving to the next.

The decision to support only circular and elliptical orbits is deliberate: these two orbit types are sufficient for understanding fundamental orbital mechanics principles while keeping the system accessible and manageable. Users learn altitude-velocity relationships, inclination effects, and the distinction between circular motion and elliptical trajectories without encountering the complexity of hyperbolic escapes or parabolic paths that would extend beyond the educational scope.

\paragraph{Mission Showcase Spaces}
Mission Spaces are dedicated educational environments that display historical space missions with their actual orbital configurations. Each space features a specialist who introduces the mission, explains its orbital characteristics, and answers questions about the underlying physics.

These spaces serve a critical pedagogical purpose: users often struggle to create orbits from scratch without understanding what realistic configurations look like. By exploring ISS, GPS, Voyager, Hubble, and optionally Apollo missions, users see concrete examples of how different orbit types serve different purposes. This experiential learning through real-world cases provides the context needed to successfully create custom orbits in the Hub.

The missions were chosen to represent diverse orbital regimes: Low Earth Orbit (ISS, Hubble), Medium Earth Orbit (GPS), and interplanetary trajectories (Voyager). Each demonstrates different physics principles—inclination requirements, altitude selection rationale, and mission-specific constraints.

\paragraph{Navigation Between Spaces}
Users move fluidly between the Hub and Mission Spaces through natural conversation. Mission Control can suggest visiting a specialist when a user's question would benefit from seeing a real example. Users return to the Hub whenever ready to apply what they learned. The system maintains conversational context across transitions, ensuring specialists understand why the user arrived and can provide relevant guidance.

\subsection{Phase 1: Core Conversational Agent and Hub Experience}
\label{subsec:phase1_agent}

\subsubsection*{Objectives}

Phase 1 establishes the conversational AI foundation that powers all user interactions. The goal is to create Mission Control as an intelligent, patient guide capable of helping users build orbits through natural dialogue, whether the user provides precise specifications or needs extensive scaffolding.

\subsubsection*{Core Capabilities}

\paragraph{Orbit Creation Through Dialogue}

The system guides users in creating orbital trajectories via multi-turn conversation. Mission Control first determines whether the user wants a circular or elliptical orbit, then collects the necessary parameters through adaptive questioning.

For users who know exactly what they want, Mission Control accepts complete specifications directly and executes immediately. For users exploring or learning, Mission Control asks clarifying questions, provides context about parameter choices, and suggests reasonable values based on the user's stated purpose.

The dialogue system validates all parameters against physical constraints—ensuring altitudes remain above the atmosphere and within simulation bounds, and confirming inclinations fall within valid ranges. When users request impossible configurations, Mission Control explains why and suggests alternatives.

\paragraph{Single-Orbit Workspace}

The Hub displays only one orbital trajectory at a time. This intentional limitation focuses attention on understanding each configuration thoroughly rather than managing multiple overlapping paths. Users can clear the current orbit at any time to start fresh, encouraging experimentation without visual clutter.

\paragraph{Routing to Educational Resources}

Mission Control recognizes when users would benefit from seeing real-world examples and can suggest visiting Mission Spaces. For instance, if a user asks "what's a good altitude for Earth observation?", Mission Control might respond "The ISS specialist can show you how altitude affects observation capabilities—would you like to speak with them?"

\subsubsection*{Technical Foundation}

The conversational system integrates OpenAI's language models to interpret natural language, reason about user intent, and generate educational explanations. A tool registry defines the available simulation functions—creating circular orbits, creating elliptical orbits, and clearing the workspace. The agent selects appropriate tools based on conversation context and extracted parameters.

The system tracks minimal state: current location (Hub or Mission Space), active orbit parameters if any exist, and recent conversation history to maintain contextual continuity across exchanges.

\subsubsection*{Success Criteria}

Phase 1 is complete when users can successfully create both circular and elliptical orbits through conversation, Mission Control adapts appropriately to different user knowledge levels, parameter validation prevents invalid configurations, and the dialogue remains coherent and educational across multiple turns.

\subsection{Phase 2: Mission Showcase Spaces}
\label{subsec:phase2_missions}

\subsubsection*{Objectives}

Phase 2 creates immersive educational environments where users explore historical space missions and learn orbital mechanics through real-world examples. Each Mission Space demonstrates how theoretical principles manifest in actual spacecraft operations.

\subsubsection*{Mission Selection Rationale}

The platform showcases four to five historically significant missions that collectively demonstrate diverse orbital configurations and mission design principles:

\paragraph{ISS Mission Space}
The International Space Station exemplifies Low Earth Orbit at 420 km altitude with a 51.6° inclination. This mission teaches how launch site locations constrain inclination choices and how orbital period relates to altitude. The specialist explains why the station orbits at this specific configuration and what tradeoffs were considered.

\paragraph{GPS Mission Space}
GPS satellites operate in Medium Earth Orbit at 20,200 km altitude with 55° inclination. This mission demonstrates constellation design principles—how multiple satellites at specific altitudes and inclinations provide global coverage. The specialist explains the relationship between orbital period (12 hours) and positioning system requirements.

\paragraph{Voyager Mission Space}
Voyager's trajectory represents interplanetary travel and escape from Earth's gravity well. Whether modeled as a high elliptical approximation or hyperbolic escape, this mission introduces concepts beyond Earth orbit. The specialist (channeling Carl Sagan's communicative style) explains gravity assists, deep space navigation, and the physics of leaving Earth's sphere of influence.

\paragraph{Hubble Mission Space}
The Hubble Space Telescope operates at 540 km altitude with a 28.5° inclination, demonstrating how mission requirements drive orbital choices. The low inclination enables launches from Kennedy Space Center while providing access to large portions of the sky. The specialist explains telescope operational constraints and orbital maintenance needs.

\paragraph{Apollo 11 Mission Space (Optional)}
If development time permits, the Apollo lunar mission provides an example of trans-lunar trajectory design. The specialist (channeling Neil Armstrong's calm, precise communication) explains the transfer orbit from Earth to Moon. If implementation proves too complex relative to timeline, this mission can be deferred to future work.

\subsubsection*{Specialist Interaction Model}

When users arrive at a Mission Space, the mission's orbital trajectory immediately appears in the visualization. The specialist greets them with a brief, engaging introduction that provides historical context and highlights the key orbital characteristics they're about to explore.

The specialist then invites questions, responding with mission-specific knowledge and educational explanations. If Mission Control routed the user for a specific reason, the specialist acknowledges this context and addresses the relevant topic.

Users return to the Hub simply by expressing the desire to go back, at which point Mission Control resumes as the primary guide.

\subsubsection*{Educational Design}

These Mission Spaces address a fundamental learning challenge: users often don't know what parameters to choose when creating their first custom orbit. By exploring concrete examples first, they develop intuition about realistic configurations. They see that LEO satellites orbit around 400-600 km, that inclination around 50° enables launches from major spaceports, and that geostationary satellites must be much higher at specific altitudes.

This example-based learning provides the conceptual framework users need to make informed choices when returning to the Hub to create their own orbits.

\subsubsection*{Success Criteria}

Phase 2 is complete when all implemented Mission Spaces display correct orbital configurations, specialists deliver engaging introductions and answer domain-specific questions accurately, users can navigate smoothly between Hub and Mission Spaces, and conversational context is preserved across transitions.

\subsection{Phase 3: Voice Integration}
\label{subsec:phase3_voice}

\subsubsection*{Objectives}

Phase 3 transforms text-based interaction into natural spoken dialogue by synthesizing all character responses with distinct voices and enabling speech input from users. This phase also implements the opening experience that welcomes users to the platform.

\subsubsection*{Voice Synthesis Strategy}

The platform leverages ElevenLabs for text-to-speech synthesis, creating distinct vocal personalities for each character. The conversational intelligence continues to come from OpenAI's language models—voice synthesis adds the auditory layer that brings characters to life.

Mission Control speaks with a visionary leader's voice, authoritative yet encouraging. The ISS specialist sounds like a professional engineer—clear, technical, friendly. The GPS specialist conveys technical expertise with precision. The Voyager specialist channels Carl Sagan's contemplative, poetic communication style. The Hubble specialist speaks with scientific enthusiasm. If implemented, the Apollo specialist echoes Neil Armstrong's calm, confident demeanor.

Voice differentiation ensures users immediately recognize which character is speaking based on voice alone, reinforcing the distinct personalities and expertise areas.

\subsubsection*{Speech Input}

Users provide input through speech recognition, enabling hands-free interaction that feels more natural than typing. The system displays transcribed text visually so users can verify what was understood before the agent processes the command.

A push-to-talk interaction model—activated via controller button in VR—provides clear boundaries between listening and speaking states, preventing accidental triggering.

\subsubsection*{Opening Experience}

The user's first interaction is a 40-second pre-scripted introduction delivered by Mission Control. This welcome sets the tone: physics is beautiful, exploration drives learning, the system is here to guide discovery rather than deliver lectures.

The introduction script is pre-written and synthesized once, ensuring consistent quality and immediate playback without API latency. After the introduction completes, users seamlessly transition to the Hub where interactive exploration begins.

\subsubsection*{Success Criteria}

Phase 3 is complete when all characters speak with distinct, natural-sounding voices, users can reliably interact using voice input, the opening experience delivers an engaging first impression, and the audio-visual experience feels cohesive and immersive.

\subsection{Phase 4: VR Deployment and Essential Controls}
\label{subsec:phase4_vr}

\subsubsection*{Objectives}

Phase 4 deploys the complete platform to Meta Quest 3, creating an immersive virtual reality experience where users physically inhabit the simulation space. This phase also implements essential simulation controls that allow users to manipulate time and manage their workspace.

\subsubsection*{VR Platform Integration}

The Unity project is configured for Meta Quest 3 deployment, enabling the application to run natively on the VR headset. Users don the headset, see Earth floating in space before them, and interact through natural head movements, controller gestures, and voice commands.

The platform can be tested locally during development and deployed as needed for demonstrations or user studies. The modular architecture ensures the same conversational AI and orbital physics work identically whether running on desktop for testing or in VR for the full experience.

\subsubsection*{Spatial Interaction Design}

In VR, interface elements exist in three-dimensional space rather than on flat screens. Orbital parameters float near their corresponding trajectories. Mission briefings appear as spatial panels that remain fixed in the environment. Controllers enable direct interaction through pointing and selection.

The design prioritizes readability and comfort—text appears at appropriate sizes for viewing from typical VR distances, colors ensure visibility against both space backgrounds and Earth, and interaction methods feel intuitive rather than requiring training.

\subsubsection*{Essential Simulation Controls}

Users need basic control over the simulation timeline to observe orbital behavior effectively:

Time can be accelerated to watch multiple orbital periods quickly, allowing users to see how satellites traverse their paths over hours compressed into seconds. The simulation can be paused to examine specific configurations or answer questions without motion distraction. Users can return time flow to normal speed when ready.

The workspace can be cleared to remove the current orbit, providing a fresh start for creating new configurations. This simple reset keeps the interface uncluttered and maintains the focus on one-orbit-at-a-time exploration.

These controls are accessible through both voice commands and controller buttons, providing flexibility in interaction methods.

\subsubsection*{Immersive Environment}

The VR environment creates presence through visual and audio design. Users are surrounded by space—a high-resolution starfield fills the sky, Earth appears as a detailed sphere, and orbital trajectories glow as curves through the void. Ambient audio adds subtle spatial atmosphere.

The scale is adjusted for comfortable VR viewing while maintaining the geometric relationships between Earth and orbits. This compressed scale keeps everything within the headset's optimal rendering volume without sacrificing physical accuracy of the relationships being demonstrated.

\subsubsection*{Success Criteria}

Phase 4 is complete when the application runs smoothly on Meta Quest 3, all conversational features function in VR, users can comfortably interact for extended sessions without discomfort, simulation controls work reliably through voice and controllers, and the immersive environment effectively conveys the three-dimensional nature of orbital mechanics.

\subsection{Development Timeline}
\label{subsec:timeline}

The four-phase structure enables systematic development with clear milestones:

\begin{description}
    \item[Phase 1 (3-4 weeks):] Core conversational agent and Hub orbit creation functionality, tested on desktop with text interaction
    \item[Phase 2 (2-3 weeks):] Mission Showcase Spaces with specialist characters and routing logic, integrated with Phase 1 foundation
    \item[Phase 3 (2-3 weeks):] Voice synthesis for all characters and speech input, including opening cutscene creation
    \item[Phase 4 (3-4 weeks):] VR deployment to Quest 3 with spatial interface and simulation controls
\end{description}

Total development timeline: 10-14 weeks, with phases building sequentially on previous work.

Early phases focus on conversational quality and educational effectiveness through desktop testing, ensuring the core learning experience is solid before adding immersive layers. Voice and VR deployment enhance an already-working system rather than being dependencies for basic functionality.

\subsection{Evaluation Approach}
\label{subsec:evaluation_approach}

The platform's effectiveness is assessed through the technical validation and educational evaluation framework detailed in Section \ref{sec:evaluation_plan}. The implementation ensures all necessary evaluation capabilities are built in:

Conversational accuracy is measured by tracking how reliably the agent interprets user intent and executes correct simulation functions. Educational effectiveness is assessed through user studies comparing comprehension before and after using the platform versus traditional learning methods.

The system logs all interactions—user commands, agent responses, orbit creations, mission visits—providing rich data for analyzing usage patterns and learning pathways.

\subsection{Technology Stack}
\label{subsec:tech_stack}

The platform integrates several technologies, each selected for its specific strength:

\begin{description}
    \item[OpenAI:] Provides the conversational intelligence that interprets natural language, reasons about orbital mechanics, and generates educational explanations
    \item[ElevenLabs:] Synthesizes natural-sounding speech for all character responses, creating distinct vocal personalities
    \item[Unity:] Serves as the 3D simulation engine and VR framework, rendering orbital trajectories and managing the immersive environment
    \item[Meta Quest 3:] Delivers the virtual reality experience through standalone wireless hardware
\end{description}

This technology combination enables sophisticated conversational AI, emotionally engaging voice interaction, and immersive spatial visualization within a single integrated platform.

\subsection{Alignment with Project Objectives}
\label{subsec:plan_alignment}

This implementation plan directly realizes the objectives established in Section \ref{sec:objetivos}:

The platform delivers physically accurate orbital simulations using validated astrodynamics equations. The generative agent system interprets natural language and provides adaptive educational guidance. Virtual reality deployment creates immersive three-dimensional visualization. Voice dialogue combined with spatial interaction enables natural, multimodal engagement. The modular four-phase architecture ensures each component can be developed, tested, and refined independently while integrating into a coherent whole.

Most importantly, the design philosophy prioritizes creating magical educational moments over accumulating features. Users experience the wonder of speaking naturally with Mission Control, hearing Carl Sagan explain Voyager's journey, watching orbits materialize in space around them, and developing genuine understanding of orbital mechanics through exploration rather than instruction.


\chapter{Results and Demonstration}
\label{chap:results}

This chapter demonstrates the complete functional platform through representative interaction scenarios. Each scenario showcases multiple integrated capabilities—conversational agent, orbital physics simulation, voice interaction, and VR visualization—working together to create an immersive educational experience for learning orbital mechanics.

The demonstrations validate the six specific objectives established in Section~\ref{sec:objetivos}: physically accurate simulation, natural language interpretation, multimodal interaction, real-time system coherence, modular architecture, and open-source delivery. Rather than isolated feature tests, each scenario represents a complete user journey that proves the system's educational effectiveness and technical integration.

\section{Initial Experience: The Hub Environment}
\label{sec:hub_experience}
\subsection{Overview}

The Hub environment serves as the user's first point of contact with the platform, introducing the conversational interface and VR spatial context. This section demonstrates the initial user experience: entering the virtual environment, meeting the Mission Control agent, and understanding the interaction paradigm.

\subsection{Environment Entry}

[STUB: Description of VR environment appearance - Earth floating in space, scale, lighting]

[STUB: Figure - Screenshot of Hub environment showing Earth model and spatial UI elements]

\subsection{Agent Introduction}

[STUB: First voice interaction - user greeting, Mission Control response]

[STUB: Voice transcript example showing Mission Control character personality]

[STUB: Demonstrates voice synthesis working, character establishment]

\subsection{Capabilities Demonstrated}

\begin{itemize}
    \item VR immersion and spatial presence (Objective \#3)
    \item Voice-based interaction modality (Section~\ref{subsec:voice_implementation})
    \item Mission Control character introduction (Appendix~\ref{app:agent_implementation})
    \item Quest 3 rendering performance (Appendix~\ref{app:vr_implementation})
\end{itemize}


\section{Creating Orbital Trajectories}
\label{sec:creating_orbits}
\subsection{Circular Orbit: ISS Configuration}
\label{subsec:iss_orbit}

[STUB: Demonstrates natural language → physics calculation → VR visualization pipeline]

\subsubsection{User Interaction}

[STUB: Voice transcript: "Create an orbit matching the ISS"]

[STUB: Figure - Screenshot showing ISS circular orbit rendered as cyan 3D curve around Earth]

\subsubsection{System Execution}

[STUB: Agent interpretation - selects create\_circular\_orbit tool]

[STUB: Physics calculation - altitude 420 km → velocity 7.66 km/s via vis-viva equation]

[STUB: Scale compression - rendered at 0.33 Unity units from surface (k=0.000785)]

[STUB: Agent response synthesis - educational explanation with ISS orbital parameters]

\subsubsection{Capabilities Demonstrated}

\begin{itemize}
    \item Natural language interpretation (Objective \#2)
    \item Tool-calling architecture (Section~\ref{subsec:agent_implementation})
    \item Vis-viva equation implementation (Appendix~\ref{app:physics_implementation})
    \item VR trajectory visualization (Section~\ref{subsec:vr_implementation})
\end{itemize}

\subsection{Elliptical Orbit: Exploring Eccentricity}
\label{subsec:elliptical_orbit}

[STUB: Demonstrates vague request interpretation, eccentricity visualization, geometric understanding]

\subsubsection{User Interaction}

[STUB: Voice transcript: "Show me a highly elliptical orbit"]

[STUB: Figure - Screenshot showing elliptical trajectory with visible oblong shape]

\subsubsection{System Execution}

[STUB: Agent chooses periapsis 200 km, apoapsis 35,786 km for "highly elliptical"]

[STUB: Eccentricity calculation e = 0.99]

[STUB: Visual demonstration - learner can walk around ellipse, see periapsis/apoapsis markers]

[STUB: Agent explains velocity variation - Kepler's Second Law]

\subsubsection{Capabilities Demonstrated}

\begin{itemize}
    \item Parameter interpretation from vague input (Objective \#2)
    \item Elliptical orbit physics (Section~\ref{subsec:physics_implementation})
    \item Spatial cognition - observing geometry from multiple angles (Section~\ref{sec:ar_vr_review})
    \item Educational explanation grounded in literature review concepts
\end{itemize}

\subsection{Comparing Orbital Configurations}
\label{subsec:comparing_orbits}

[STUB: Demonstrates multiple simultaneous orbits, comparative learning]

\subsubsection{User Interaction}

[STUB: Creating ISS (420 km) then Hubble (540 km) orbits]

[STUB: Figure - Screenshot showing both orbits visible simultaneously]

\subsubsection{Educational Value}

[STUB: Visual comparison shows altitude-radius relationship]

[STUB: Agent explains velocity difference: ISS 7.66 km/s vs Hubble 7.59 km/s]

[STUB: Demonstrates misconception correction - higher orbit = slower speed]

\subsubsection{Capabilities Demonstrated}

\begin{itemize}
    \item Workspace management (clear\_orbit tool)
    \item Comparative visualization supporting conceptual understanding
    \item Physics accuracy verified against published data (Appendix~\ref{app:physics_implementation})
\end{itemize}


\section{Mission Space Exploration}
\label{sec:mission_exploration}
\subsection{Navigating to ISS Mission Space}
\label{subsec:iss_navigation}

[STUB: Demonstrates scene transitions, character switching, context persistence]

\subsubsection{User Interaction}

[STUB: Voice transcript: "Take me to the ISS mission"]

[STUB: Figure - Transition overlay showing ISS logo during 4-second async scene load]

\subsubsection{System Execution}

[STUB: Agent selects route\_to\_mission tool with parameter "ISS"]

[STUB: SceneTransitionManager loads ISS.unity asynchronously]

[STUB: Voice switches from Mission Control (NOpBlnGInO9m6vDvFkFC) to Anastasia (ZF6FPAbjXT4488VcRRnw)]

[STUB: Conversation history preserved - 10-exchange window maintained]

\subsubsection{Capabilities Demonstrated}

\begin{itemize}
    \item Navigation tools (Objective \#5 - modular architecture)
    \item Scene transitions without stuttering (Appendix~\ref{app:vr_implementation})
    \item Character voice switching (Appendix~\ref{app:voice_implementation})
    \item Context persistence across scenes (Section~\ref{subsec:agent_implementation})
\end{itemize}

\subsection{Educational Dialogue with Specialists}
\label{subsec:specialist_dialogue}

[STUB: Demonstrates non-tool responses, specialist knowledge domains, educational guidance]

\subsubsection{User Interaction}

[STUB: Voice transcript in ISS Space: "Why is the ISS at 51.6 degrees inclination?"]

[STUB: Figure - ISS Mission Space environment screenshot]

\subsubsection{System Execution}

[STUB: Agent recognizes educational question - no tool execution]

[STUB: Draws from ISS specialist knowledge domain configured in ISS\_Config.asset]

[STUB: Anastasia explains Baikonur launch site latitude constraint]

[STUB: Response maintains professional engineer personality]

\subsubsection{Agent Response Example}

[STUB: Extended voice transcript showing educational explanation quality]

\subsubsection{Capabilities Demonstrated}

\begin{itemize}
    \item Conversational vs tool-calling mode distinction
    \item Specialist knowledge integration (MissionConfig system)
    \item Educational dialogue rooted in orbital mechanics literature review
    \item Character personality consistency
\end{itemize}

\subsection{Voyager Escape Trajectory}
\label{subsec:voyager_escape}

[STUB: Demonstrates advanced physics, hyperbolic trajectories, character differentiation]

\subsubsection{User Interaction}

[STUB: After routing to Voyager Space: "Show me how Voyager escaped Earth"]

[STUB: Figure - Hyperbolic departure trajectory visualization]

\subsubsection{System Execution}

[STUB: Karl (Voyager specialist) responds in philosophical voice]

[STUB: Creates extreme elliptical orbit - periapsis 200 km, apoapsis 800,000 km]

[STUB: Agent explains escape velocity physics: v\_escape = sqrt(2$\mu$/r) = 11.2 km/s]

[STUB: Voice synthesis at speed 0.9 (contemplative pacing from KarlVoiceSettings)]

\subsubsection{Agent Response Example}

[STUB: Transcript showing Karl's poetic, Sagan-like character: "humanity's first steps beyond the solar system..."]

\subsubsection{Capabilities Demonstrated}

\begin{itemize}
    \item All three mission specialists functional (Mission Control, Anastasia, Karl)
    \item Character voice differentiation (technical vs philosophical personalities)
    \item Physics range: LEO to interplanetary escape
    \item Advanced orbital mechanics concepts from literature review
\end{itemize}


\section{Simulation Control and Interaction}
\label{sec:simulation_control}
\subsection{Time Manipulation}
\label{subsec:time_control}

[STUB: Demonstrates playback control, disambiguation between orbital velocity and simulation speed]

\subsubsection{User Interaction}

[STUB: Voice transcript: "Speed up the simulation"]

[STUB: Figure - UI showing "10x" speed multiplier indicator in spatial world space]

\subsubsection{System Execution}

[STUB: Agent selects set\_simulation\_speed tool (not orbit modification)]

[STUB: Distinguishes: orbital velocity (physics constant) vs time multiplier (playback variable)]

[STUB: Sets speed\_multiplier = 10x within constraint range 0.1-100x]

[STUB: VR visualization shows satellite moving 10x faster around orbit]

\subsubsection{Educational Value}

[STUB: Prevents common misconception - "speed" does not change orbital velocity]

[STUB: Demonstrates tool disambiguation from 460-line prompt guidance]

\subsubsection{Capabilities Demonstrated}

\begin{itemize}
    \item Tool selection accuracy (set\_simulation\_speed vs create\_orbit tools)
    \item Real-time simulation parameter control (Objective \#4)
    \item Spatial UI rendering in VR (Appendix~\ref{app:vr_implementation})
    \item Misconception prevention through prompt engineering
\end{itemize}

\subsection{Workspace Management}
\label{subsec:workspace_management}

[STUB: Demonstrates iterative exploration workflow, orbit clearing, rebuilding]

\subsubsection{User Interaction}

[STUB: Voice transcript sequence: "Clear the current orbit" → "Now create an orbit at 1000 km altitude"]

[STUB: Figure - Before/after screenshots showing orbit removal and new orbit creation]

\subsubsection{System Execution}

[STUB: Agent executes clear\_orbit tool]

[STUB: OrbitController removes trajectory visualization]

[STUB: Follow-up orbit creation in clean workspace]

\subsubsection{Educational Value}

[STUB: Supports iterative learning - hypothesis testing, comparison, refinement]

[STUB: Enables exploration without accumulating visual clutter]

\subsubsection{Capabilities Demonstrated}

\begin{itemize}
    \item Complete tool suite integration (8 tools from Appendix~\ref{app:agent_implementation})
    \item Iterative exploration pattern supporting embodied learning
    \item Workspace state management
\end{itemize}


\section{Open-Source Platform Delivery}
\label{sec:open_source_delivery}
The complete platform source code, including Unity C\# scripts, conversational agent prompts, configuration assets, and Quest 3 deployment settings, is publicly available on GitHub, fulfilling Objective \#6 from Section~\ref{sec:objetivos}. The repository provides comprehensive documentation covering system architecture, Unity configuration procedures, and Quest 3 build instructions, enabling community validation, adaptation, and extension.

\subsection{Repository Structure}

[STUB: GitHub repository organization - Assets/, Relatorio/, documentation]

[STUB: Key files documented: PromptSettings.asset, OrbitController.cs, ElevenLabsClient.cs, etc.]

\subsection{User API Key Configuration}

Users provide their own OpenAI and ElevenLabs API keys through Unity Inspector configuration, ensuring platform accessibility without imposed service costs. The documentation guides users through:

\begin{itemize}
    \item Obtaining OpenAI API key for GPT-4.1 access
    \item Configuring ElevenLabs API key for voice synthesis/transcription
    \item Setting voice IDs for character customization
    \item Adjusting prompt templates for educational customization
\end{itemize}

\subsection{Quest 3 Deployment Documentation}

[STUB: Build configuration instructions - Android SDK, Unity XR packages, Oculus integration]

[STUB: Step-by-step deployment guide from Unity to Quest 3 device]

\subsection{Extensibility Demonstrations}

The modular architecture (Objective \#5) enables straightforward extensions:

\begin{itemize}
    \item Adding new mission specialists by creating MissionConfig ScriptableObject assets
    \item Extending tool suite with new orbit types or celestial bodies
    \item Customizing educational prompts for different learning contexts
    \item Adapting voice characters for different languages or audiences
\end{itemize}

[STUB: Example - Adding Mars mission would require Mars\_Config.asset + Mars.unity scene]

This open-source delivery validates the platform's educational accessibility and community extensibility objectives established in the thesis motivation (Section~\ref{sec:motivation}).


\section{Complete System Demonstration}
\label{sec:complete_demo}
% Section 4.7: Complete System Demonstration
% REFINED VERSION - Concise, evidence-focused, minimal redundancy
% Removes: excessive summarization, promotional language, out-of-scope sections

Sections~\ref{sec:entering_environment} through~\ref{sec:open_source_delivery} analyze individual capabilities and technical components in isolation. This section presents evidence of complete system integration: a continuous demonstration recording showing all subsystems operating together without interruption, manual intervention, or post-production editing. The recording validates that the conversational agent, orbital physics engine, voice synthesis pipeline, and VR environment maintain coherence across the full interaction workflow documented in Sections~\ref{sec:entering_environment}--\ref{sec:escape_concept}.

\subsection{Video Documentation}

A demonstration recording captures the complete user session analyzed throughout this chapter. The full interaction transcript appears in Appendix~\ref{appendix:transcript}.

\textbf{Access}: \url{https://www.youtube.com/watch?v=S73l4_CgTtY}

\textbf{Duration}: 11 minutes, 47 seconds of uninterrupted interaction

\textbf{Recording Format}: Direct Quest 3 capture (first-person stereoscopic perspective, spatial audio, 90~Hz refresh maintained throughout)

\textbf{Content Scope}: The recording covers all interaction scenarios presented in this chapter:
\begin{itemize}
    \item Environment entry and orientation (Section~\ref{sec:entering_environment})
    \item ISS mission consultation and circular orbit creation (Section~\ref{sec:iss_learning})
    \item Hubble consultation and elliptical orbit exploration (Section~\ref{sec:elliptical_exploration})
    \item Voyager consultation on escape trajectories (Section~\ref{sec:escape_concept})
    \item Tool execution: 9 scene transitions, 3 orbit configurations, 2 time acceleration adjustments
\end{itemize}

The recording presents the platform as it functions during real usage—including natural pauses, user hesitations, and iterative refinement requests—without rehearsal or selective editing.

\subsection{System Integration Evidence}

The continuous recording demonstrates multi-module coordination that isolated component tests cannot validate:

\textbf{Cross-Module Reliability}: During 11:47 of operation, the system executed 14 tool calls (Table~\ref{tab:tool_execution_summary}) without failures:
\begin{itemize}
    \item Voice transcription: All 18 user utterances correctly transcribed (ElevenLabs Scribe v2)
    \item Tool selection: GPT-4 identified appropriate actions for all requests, including ambiguous queries requiring specialist routing
    \item Physics calculations: Orbital parameters matched analytical predictions (e.g., 422~km altitude → 7.66~km/s velocity → 92.8~min period, within 0.3\% of published ISS data)
    \item Scene transitions: 9 asynchronous scene loads completed without rendering artifacts or audio desynchronization
    \item Audio synthesis: 22 agent responses generated and played without truncation or silence gaps
\end{itemize}

\begin{table}[h]
\centering
\caption{Tool execution summary from demonstration recording}
\label{tab:tool_execution_summary}
\begin{tabular}{lcc}
\hline
\textbf{Tool Category} & \textbf{Invocations} & \textbf{Success Rate} \\
\hline
Navigation (\texttt{route\_to\_mission}, \texttt{return\_to\_hub}) & 9 & 100\% \\
Orbit Creation (\texttt{create\_circular}, \texttt{create\_elliptical}) & 3 & 100\% \\
Simulation Control (\texttt{set\_simulation\_speed}) & 2 & 100\% \\
\hline
\textbf{Total} & \textbf{14} & \textbf{100\%} \\
\hline
\end{tabular}
\end{table}

\textbf{Context Persistence}: The \texttt{MissionContext} singleton (Section~\ref{sec:system_architecture}) maintained conversation history across all 9 scene transitions. Specialists referenced prior exchanges when greeting the user:
\begin{itemize}
    \item ANASTASIA: ``You're asking about good altitude choices'' (referencing user's initial Hub question)
    \item DR\_HARRISON: ``You've built circular orbits—now let's explore elliptical geometry'' (referencing prior ISS session)
    \item CARL: ``You're wondering about deep-space trajectories beyond circular and elliptical orbits'' (building on Hubble discussion)
\end{itemize}

These contextual references—generated dynamically by GPT-4 from stored conversation history—demonstrate that the modular architecture (Objective~\#5) preserves state coherence despite environment changes.

\textbf{Multimodal Synchronization}: Voice, visualization, and physics remained coordinated throughout:
\begin{itemize}
    \item Agent responses referenced visible trajectories: ``Watch it speed up near Earth and slow down far away'' (spoken while elliptical orbit displayed on screen)
    \item Time acceleration changes (10× and 100×) applied immediately, with UI confirmation and perceptible motion changes
    \item Scene transition audio cues (logo overlay, specialist introductions) synchronized with environment loading completion
\end{itemize}

\subsection{Demonstration Characteristics}

The recording exhibits characteristics consistent with authentic exploratory learning rather than scripted tutorial execution:

\textbf{User Question Progression}: Questions evolved from concrete parameters (``What's a good altitude?'') to conceptual boundaries (``Is everything elliptical for deep-space missions?''). This progression—documented in full in Appendix~\ref{appendix:transcript}—emerged from the user's curiosity rather than predetermined learning objectives.

\textbf{Iterative Refinement}: The user requested orbit adjustments based on visual observation: ``Make it more elliptical; it still looks circular'' led to a second elliptical orbit with greater eccentricity (200~km × 1,000~km vs initial 400~km × 2,000~km). This feedback loop—observe, critique, refine—demonstrates that the platform supports genuine experimentation.

\textbf{Misconception Handling}: When the user asked ``Can I choose the speed?'', CAPCOM clarified the physics constraint: ``Speed is derived from altitude by physics. At 422~km, you need 7.66~km/s for a stable circular orbit.'' This real-time correction prevented a conceptual error without interrupting the learning flow.

\textbf{Spatial Language}: User utterances reflected VR-enabled spatial cognition: ``It looks fast, but Earth is massive'' (scale perception), ``Wow—huge difference between near and far'' (Kepler's Second Law observation). These spontaneous reactions—captured in the recording's audio—suggest that immersive visualization supports intuitive understanding of orbital dynamics.

\subsection{Integration Validation Summary}

The continuous demonstration recording provides evidence that:
\begin{enumerate}
    \item All subsystems operate reliably together during real-world usage (no failures across 14 tool invocations, 18 voice interactions, 9 scene transitions)
    \item Modular architecture maintains state coherence despite frequent environment changes (conversation context preserved across all transitions)
    \item Multimodal coordination persists throughout the interaction workflow (voice, visualization, physics synchronized)
    \item The platform supports exploratory learning patterns (question progression, iterative refinement, misconception correction)
\end{enumerate}

These results validate Objective~\#4 (real-time system coherence) in an integrated scenario that isolated component tests cannot replicate. The recording, combined with the open-source repository (Section~\ref{sec:open_source_delivery}), enables independent verification of these claims by the research community.



% REFERENCIAS BIBLIOGRAFICAS
\renewcommand\bibname{\itareferencesnamebabel} %renomear título do capítulo referências
\printbibliography

% Apendices
\appendix
\chapter{Agent System Implementation}
\label{app:agent_implementation}

This appendix provides detailed technical implementation specifications for the conversational agent system described in Section~\ref{subsec:agent_implementation}, implementing the tool-calling architecture and memory management principles outlined in Section~\ref{sec:generative-agents}. All class names, method signatures, file paths, and code excerpts are verified against the Unity project source code.

\section{Prompt Architecture}
\label{app:agent_prompts}

The agent system operates through structured prompts stored in the \texttt{PromptSettings} ScriptableObject configuration asset. Table~\ref{tab:prompt_specifications} summarizes the prompt components and their purposes.

\begin{table}[h]
\centering
\caption{Agent Prompt Component Specifications}
\label{tab:prompt_specifications}
\begin{tabular}{ll}
\hline
\textbf{Prompt Component} & \textbf{Purpose} \\
\hline
\texttt{toolSelectionPrompt} & Interprets user intent, returns tool JSON \\
\texttt{responsePrompt} & Generates natural language responses \\
\texttt{specialistSystemPrompt} & Frames mission specialist character \\
\texttt{nonToolResponseTemplate} & Handles conversational interactions \\
\texttt{toolResponseTemplate} & Formats tool execution feedback \\
\texttt{specialistIntroTemplate} & Generates 40-word greetings \\
\hline
\end{tabular}
\end{table}

\subsection{Hub Agent: Three-Tier Prompt System}

The Hub agent (Mission Control) uses three coordinated prompts:

\subsubsection*{Tool Selection Prompt (460 lines)} Instructs GPT-4.1 to analyze user natural language input and return structured JSON identifying which tool to invoke. The prompt explicitly defines eight available tools:

\begin{itemize}
\item \textbf{Orbit Creation}: \texttt{create\_circular\_orbit}, \texttt{create\_elliptical\_orbit}
\item \textbf{Simulation Control}: \texttt{set\_simulation\_speed}, \texttt{pause\_simulation}, \texttt{reset\_simulation\_time}
\item \textbf{Workspace Management}: \texttt{clear\_orbit}
\item \textbf{Navigation}: \texttt{route\_to\_mission}, \texttt{return\_to\_hub}
\end{itemize}

\subsubsection*{Response Prompt (270 lines)} Generates natural language explanations of tool execution results. Includes explicit disambiguation guidance to prevent confusion between:
\begin{itemize}
\item \textbf{Orbital velocity} (physics-calculated, 7.66 km/s for ISS)
\item \textbf{Simulation time speed} (user-controllable playback multiplier)
\end{itemize}

\subsubsection*{Non-Tool Response Template} Handles conversational interactions that do not require tool execution, such as greetings (``Hello, I'm Mission Control''), capability inquiries (``What can you do?''), and educational questions.

\subsection{Mission Specialist Prompts}

Mission Space specialists (ISS, Hubble, Voyager) use the \texttt{specialistSystemPrompt} (412 lines) which frames the agent as an enthusiastic mission expert focused on education rather than simulation control. Character configuration occurs through \texttt{MissionConfig} ScriptableObject assets:

\begin{itemize}
\item \texttt{ISS\_Config.asset}: Character name ``Anastasia'', personality ``Professional engineer - clear, technical, friendly''
\item \texttt{Hubble\_Config.asset}: Hubble Space Telescope mission specialist
\item \texttt{Voyager\_Config.asset}: Voyager interplanetary mission specialist
\end{itemize}

The \texttt{specialistIntroTemplate} generates concise 40-word, 10-15 second greetings acknowledging the routing context from \texttt{route\_to\_mission}.

\section{Tool Schema and Validation}
\label{app:agent_tools}

The eight tools are defined in \texttt{ToolSchemas.json} (169 lines) with complete JSON Schema specifications. Table~\ref{tab:tool_constraints} documents parameter constraints enforced by the \texttt{ToolRegistry} validation system.

\begin{table}[h]
\centering
\caption{Tool Parameter Constraints}
\label{tab:tool_constraints}
\begin{tabular}{llr}
\hline
\textbf{Tool} & \textbf{Parameter} & \textbf{Constraint} \\
\hline
\texttt{create\_circular\_orbit} & \texttt{altitude\_km} & 160--35,786 km \\
\texttt{create\_circular\_orbit} & \texttt{inclination\_deg} & 0--180° \\
\texttt{create\_elliptical\_orbit} & \texttt{periapsis\_km} & 160--35,786 km \\
\texttt{create\_elliptical\_orbit} & \texttt{apoapsis\_km} & 160--100,000 km \\
\texttt{create\_elliptical\_orbit} & \texttt{inclination\_deg} & 0--180° \\
\texttt{set\_simulation\_speed} & \texttt{speed\_multiplier} & 0.1--100× \\
\hline
\end{tabular}
\end{table}

\subsection{Tool Execution Pipeline}

The \texttt{ToolExecutor} class receives validated tool calls from the agent system and invokes corresponding C\# methods:

\begin{itemize}
\item \textbf{Orbit Tools} $\rightarrow$ \texttt{OrbitController.CreateCircularOrbit()}, \texttt{CreateEllipticalOrbit()}
\item \textbf{Time Controls} $\rightarrow$ \texttt{TimeController.SetSpeed()}, \texttt{Pause()}, \texttt{ResetTime()}
\item \textbf{Navigation Tools} $\rightarrow$ \texttt{SceneTransitionManager.TransitionToMission()}, \texttt{TransitionToHub()}
\item \textbf{Workspace} $\rightarrow$ \texttt{OrbitController.ClearOrbit()}
\end{itemize}

Execution results—success status, generated orbital parameters, error messages—feed back into the LLM response generation cycle through the response prompt template.

\section{Conversation Context Management}
\label{app:agent_context}

The \texttt{ConversationHistory} class maintains conversation continuity across multi-turn dialogues and scene transitions. Table~\ref{tab:context_structure} documents the exchange data structure.

\begin{table}[h]
\centering
\caption{Conversation Exchange Data Structure}
\label{tab:context_structure}
\begin{tabular}{ll}
\hline
\textbf{Field} & \textbf{Content} \\
\hline
\texttt{timestamp} & DateTime of exchange \\
\texttt{userMessage} & User's natural language input \\
\texttt{agentResponse} & Agent's generated response \\
\texttt{toolExecuted} & Tool name (or null if conversational) \\
\texttt{location} & Current scene (Hub, ISS, Hubble, Voyager) \\
\hline
\end{tabular}
\end{table}

\subsection{Context Window Management}

The system maintains a sliding window of the last 10 exchanges (\texttt{maxHistorySize = 10}). Two methods provide context injection into prompts:

\begin{itemize}
\item \texttt{GetFormattedHistory(lastNExchanges = 5)}: Returns detailed history with timestamps, locations, and tool executions for the last 5 exchanges
\item \texttt{GetContextSummary(lastNExchanges = 3)}: Returns condensed 3-exchange summary optimized for token efficiency
\end{itemize}

\subsection{Cross-Scene Persistence}

Scene transitions preserve conversation history through Unity's \texttt{DontDestroyOnLoad} mechanism. The \texttt{PromptConsole} GameObject, containing the \texttt{ConversationHistory} component, persists across scene unloading when users invoke \texttt{route\_to\_mission} or \texttt{return\_to\_hub} tools. This ensures unbroken dialogue continuity: a user can ask ``What was the ISS orbit altitude I created in the Hub?'' after transitioning to the ISS Mission Space.

\section{API Integration}
\label{app:agent_api}

The \texttt{OpenAIClient} class (150 lines) implements asynchronous HTTP communication with OpenAI's Responses API endpoint (\texttt{https://api.openai.com/v1/responses}).

\subsection{Request Structure}

Requests to the \texttt{/responses} endpoint include:

\begin{itemize}
\item \textbf{Model}: \texttt{"gpt-4.1"}
\item \textbf{Input}: User's natural language message
\item \textbf{Instructions}: Concatenated system prompt + conversation history + tool schemas
\end{itemize}

The \texttt{CompleteAsync()} method constructs JSON payloads using Unity's \texttt{UnityWebRequest} for async/await compatibility.

\subsection{Response Parsing}

The client extracts assistant text from JSON responses through a two-stage fallback:

\begin{enumerate}
\item \textbf{Primary}: Extract \texttt{output\_text} convenience field (if present)
\item \textbf{Fallback}: Concatenate all \texttt{output[].content[].text} arrays
\end{enumerate}

Tool call JSON undergoes validation by \texttt{ToolRegistry} before execution. Results format back into natural language through the response prompt template system, generating contextual explanations like: ``I've created a circular orbit at 420 km altitude with 51.6° inclination. The orbital velocity is 7.66 km/s, matching the ISS configuration.''

\subsection{Mission-Specific Configuration}

Each Mission Space scene loads scene-specific \texttt{OpenAISettings} ScriptableObject assets that override the default system prompt, enabling character switching when users transition from Hub (Mission Control) to Mission Spaces (specialist agents).

\chapter{Orbital Physics Implementation}
\label{app:physics_implementation}

This appendix provides detailed technical specifications for the orbital physics simulation engine described in Section~\ref{subsec:physics_implementation}. All equations, algorithms, class methods, and numerical values are verified against the Unity project physics implementation.

\section{Two-Body Keplerian Mechanics}
\label{app:physics_keplerian}

The simulation implements two-body orbital mechanics under the following simplifying assumptions:

\begin{itemize}
\item Earth modeled as a point mass at the coordinate system origin
\item Satellite treated as a massless test particle (no gravitational influence on Earth)
\item No atmospheric drag, solar radiation pressure, or third-body perturbations
\item Instantaneous orbital maneuvers (no finite burn durations)
\end{itemize}

These assumptions yield closed-form Keplerian solutions suitable for educational visualization while maintaining physical accuracy for the mission profiles studied (ISS, Hubble, Voyager departure trajectory).

\section{Vis-Viva Equation Implementation}
\label{app:physics_visviva}

The vis-viva equation relates orbital velocity to position and total orbital energy. Table~\ref{tab:physics_constants} documents the physical constants used throughout the simulation.

\begin{table}[h]
\centering
\caption{Physical Constants for Orbital Calculations}
\label{tab:physics_constants}
\begin{tabular}{lrl}
\hline
\textbf{Constant} & \textbf{Value} & \textbf{Symbol} \\
\hline
Earth's standard gravitational parameter & 398,600 km³/s² & $\mu$ \\
Earth's mean radius & 6,371 km & $R_\oplus$ \\
Unity scale compression factor & 0.000785 Unity/km & $k$ \\
Unity Earth radius & 5 Unity units & $R_{\text{Unity}}$ \\
\hline
\end{tabular}
\end{table}

\subsection{Circular Orbit Calculation}

Circular orbits ($e = 0$) simplify the vis-viva equation to:

\begin{equation}
v_{\text{circular}} = \sqrt{\frac{\mu}{r}}
\label{eq:circular_velocity}
\end{equation}

where $r = R_\oplus + h$ is the orbital radius from Earth's center, and $h$ is the altitude above Earth's surface.

\paragraph{Implementation Method} The \texttt{OrbitController.CreateCircularOrbit()} method (lines 229--290) accepts altitude in kilometers and automatically calculates orbital velocity, eliminating user confusion between altitude and speed parameters. Algorithm~\ref{alg:circular_orbit} documents the calculation sequence.

\begin{table}[h]
\centering
\caption{Circular Orbit Calculation Algorithm}
\label{alg:circular_orbit}
\begin{tabular}{ll}
\hline
\textbf{Step} & \textbf{Calculation} \\
\hline
1. Validate altitude & $h_{\text{input}} \rightarrow \text{Clamp}(160, 35{,}786)$ km \\
2. Compute orbital radius & $r = R_\oplus + h = 6{,}371 + h$ km \\
3. Calculate orbital velocity & $v = \sqrt{\mu / r}$ km/s \\
4. Convert to Unity scale & $r_{\text{Unity}} = R_{\text{Unity}} + h \cdot k$ \\
5. Convert to angular velocity & $\omega = (v \cdot k) / r_{\text{Unity}}$ rad/s \\
\hline
\end{tabular}
\end{table}

\paragraph{Example: ISS Orbital Velocity} For the International Space Station at $h = 420$ km altitude:

\begin{align}
r &= 6{,}371 + 420 = 6{,}791 \text{ km} \\
v &= \sqrt{\frac{398{,}600}{6{,}791}} = 7.66 \text{ km/s} \\
\omega &= \frac{7.66 \times 0.000785}{5 + (420 \times 0.000785)} = 0.00111 \text{ rad/s}
\end{align}

This matches the real ISS orbital velocity of approximately 7.66 km/s.

\subsection{Elliptical Orbit Calculation}

Elliptical orbits ($0 < e < 1$) use the full vis-viva equation:

\begin{equation}
v = \sqrt{\mu \left(\frac{2}{r} - \frac{1}{a}\right)}
\label{eq:visviva_full}
\end{equation}

where $a$ is the semi-major axis and $r$ is the instantaneous distance from Earth's center.

\paragraph{Orbital Elements Derivation} Given periapsis altitude $h_p$ and apoapsis altitude $h_a$:

\begin{align}
r_p &= R_\oplus + h_p \quad \text{(periapsis radius)} \\
r_a &= R_\oplus + h_a \quad \text{(apoapsis radius)} \\
a &= \frac{r_p + r_a}{2} \quad \text{(semi-major axis)} \\
e &= \frac{r_a - r_p}{r_a + r_p} \quad \text{(eccentricity)}
\end{align}

\paragraph{Implementation Method} The \texttt{OrbitController.CreateEllipticalOrbit()} method (lines 300--367) computes velocity at periapsis using Equation~\ref{eq:visviva_full} with $r = r_p$:

\begin{equation}
v_p = \sqrt{\mu \left(\frac{2}{r_p} - \frac{1}{a}\right)}
\label{eq:periapsis_velocity}
\end{equation}

Table~\ref{tab:elliptical_constraints} documents parameter validation constraints enforced before calculation.

\begin{table}[h]
\centering
\caption{Elliptical Orbit Parameter Constraints}
\label{tab:elliptical_constraints}
\begin{tabular}{lr}
\hline
\textbf{Parameter} & \textbf{Constraint} \\
\hline
Periapsis altitude $h_p$ & 160--35,786 km \\
Apoapsis altitude $h_a$ & $h_p + 1$ km to 100,000 km \\
Eccentricity $e$ & $0 < e < 1$ (enforced implicitly) \\
Inclination $i$ & 0--180° \\
\hline
\end{tabular}
\end{table}

\section{Scale Compression}
\label{app:physics_scale}

The simulation implements logarithmic scale compression to fit orbital mechanics within the Meta Quest 3's comfortable rendering volume while preserving geometric relationships.

\subsection{Compression Factor Derivation}

Earth's physical radius (6,371 km) maps to 5 Unity units:

\begin{equation}
k = \frac{R_{\text{Unity}}}{R_\oplus} = \frac{5}{6{,}371} = 0.000785 \text{ Unity units/km}
\end{equation}

\paragraph{Example Mappings} Table~\ref{tab:scale_examples} shows real-world altitudes mapped to Unity rendering coordinates.

\begin{table}[h]
\centering
\caption{Scale Compression Examples}
\label{tab:scale_examples}
\begin{tabular}{lrr}
\hline
\textbf{Mission} & \textbf{Real Altitude (km)} & \textbf{Unity Altitude} \\
\hline
ISS & 420 & $420 \times 0.000785 = 0.33$ \\
Hubble & 540 & $540 \times 0.000785 = 0.42$ \\
Geostationary & 35,786 & $35{,}786 \times 0.000785 = 28.1$ \\
\hline
\end{tabular}
\end{table}

This compression maintains visual proportions: the ISS appears at $0.33 / 5 = 6.6\%$ of Earth's radius above the surface, matching the real ratio of $420 / 6{,}371 = 6.6\%$.

\subsection{Numerical Stability}

All physics calculations occur in real units (km, km/s) before conversion to Unity space for rendering. This ensures:

\begin{itemize}
\item No floating-point precision loss from working with very small Unity coordinates
\item Physical accuracy verifiable against published orbital data
\item Separation of physics (model) from rendering (view)
\end{itemize}

The \texttt{OrbitController} methods perform calculations in kilometers, then convert final results through multiplication by $k$ only when setting Unity Transform positions.

\section{Trajectory Visualization}
\label{app:physics_visualization}

The \texttt{OrbitVisualizer} class (280 lines) generates trajectory curves by sampling the orbital ellipse equation at discrete points and rendering through Unity's LineRenderer system.

\subsection{Orbital Ellipse Equation}

The orbit trajectory follows the polar equation:

\begin{equation}
r(\theta) = \frac{a(1 - e^2)}{1 + e \cos \theta}
\label{eq:orbit_polar}
\end{equation}

where $\theta$ is the true anomaly (angle from periapsis), $a$ is the semi-major axis, and $e$ is the eccentricity.

\subsection{Sampling Algorithm}

The \texttt{CalculateOrbitalPoint()} method (lines 217--231) samples Equation~\ref{eq:orbit_polar} at 128 evenly-spaced true anomaly angles $\theta \in [0, 2\pi)$. For each sample point:

\begin{align}
r &= \frac{a(1 - e^2)}{1 + e \cos \theta} \\
x &= r \cos(\theta + \omega) \\
z &= r \sin(\theta + \omega)
\end{align}

where $\omega$ is the argument of periapsis (orientation of the ellipse major axis in the orbital plane).

\paragraph{Inclination Transformation} The resulting planar coordinates undergo rotation by inclination angle $i$ via rotation matrix:

\begin{equation}
\mathbf{R}_i = \begin{bmatrix}
1 & 0 & 0 \\
0 & \cos i & -\sin i \\
0 & \sin i & \cos i
\end{bmatrix}
\label{eq:inclination_matrix}
\end{equation}

This tilts the orbital plane from equatorial (XZ) to the specified inclination angle, enabling visualization of polar orbits (ISS at 51.6°) and equatorial orbits (geostationary at 0°).

\subsection{Rendering Configuration}

Table~\ref{tab:linerenderer_config} documents the Unity LineRenderer configuration for optimal VR visibility.

\begin{table}[h]
\centering
\caption{LineRenderer Configuration Parameters}
\label{tab:linerenderer_config}
\begin{tabular}{ll}
\hline
\textbf{Parameter} & \textbf{Value} \\
\hline
Path resolution & 128 points \\
Line width & 0.05 Unity units \\
Color & Cyan (0, 1, 1) with 0.7 alpha \\
Shader & Sprites/Default (view-aligned billboarding) \\
Loop closure & Enabled (connects point 127 to point 0) \\
\hline
\end{tabular}
\end{table}

\paragraph{Special Cases}

\begin{itemize}
\item \textbf{Circular orbits} ($e = 0$): Simplify to constant radius $r = a$, producing perfect circles
\item \textbf{Elliptical orbits} ($0 < e < 1$): Render with visible eccentricity
\item \textbf{Debug visualization}: Green gizmo at periapsis, red gizmo at apoapsis for development testing
\end{itemize}

\section{Coordinate System Conventions}
\label{app:physics_coordinates}

The simulation uses Unity's left-handed coordinate system with the following conventions:

\begin{itemize}
\item \textbf{Origin}: Earth's center of mass
\item \textbf{Equatorial plane}: XZ plane ($y = 0$)
\item \textbf{Polar axis}: +Y direction points toward North Pole
\item \textbf{Reference direction}: +X axis defines 0° longitude
\item \textbf{Orbital motion}: Counterclockwise when viewed from above North Pole (right-hand rule)
\end{itemize}

This convention aligns with standard aerospace engineering practices while accommodating Unity's left-handed rendering system.

\chapter{Voice Pipeline Implementation}
\label{app:voice_implementation}

This appendix provides detailed technical specifications for the bidirectional voice system described in Section~\ref{subsec:voice_implementation}, implementing voice interaction as a hands-free modality for immersive VR environments outlined in Section~\ref{sec:ar_vr_review}. All API endpoints, audio formats, class methods, and processing parameters are verified against the Unity project voice integration code.

\section{System Architecture}
\label{app:voice_architecture}

The voice pipeline implements bidirectional audio through ElevenLabs cloud APIs, enabling natural spoken interaction with the agent system. The data flow follows this sequence: user speech → speech-to-text transcription → agent processing → text-to-speech synthesis → audio playback.

\subsubsection*{Component Responsibilities}

\begin{itemize}
\item \textbf{PromptConsole}: Manages microphone capture, push-to-talk input detection, and audio playback
\item \textbf{ElevenLabsClient}: Handles HTTP communication with ElevenLabs APIs (Assets/Scripts/AI/Services/ElevenLabsClient.cs, 394 lines)
\item \textbf{Unity AudioSource}: Plays synthesized speech through Quest 3's spatial audio system
\item \textbf{Unity Microphone}: Captures user voice input at 16 kHz sample rate
\end{itemize}

\section{Speech-to-Text Pipeline}
\label{app:voice_stt}

Speech recognition converts user voice input to text through ElevenLabs' Scribe v1 transcription model. Table~\ref{tab:stt_specs} documents the audio capture specifications.

\begin{table}[h]
\centering
\caption{Speech-to-Text Audio Capture Specifications}
\label{tab:stt_specs}
\begin{tabular}{ll}
\hline
\textbf{Parameter} & \textbf{Value} \\
\hline
Sample rate & 16,000 Hz (optimized for speech) \\
Bit depth & 16-bit PCM \\
Channels & Mono \\
Maximum duration & 30 seconds \\
Audio format (transmitted) & WAV with RIFF header \\
API endpoint & \texttt{/speech-to-text} \\
Model & \texttt{scribe\_v1} \\
\hline
\end{tabular}
\end{table}

\subsection{Push-to-Talk Input Detection}

Voice recording activates through push-to-talk button press. The \texttt{PromptConsole.Update()} method (lines 281--323) implements platform-specific input detection with debouncing to prevent accidental double-triggers.

\subsubsection*{Input Source Detection}

\begin{itemize}
\item \textbf{Desktop testing}: Space key via \texttt{Input.GetKeyDown(KeyCode.Space)}
\item \textbf{VR deployment}: Quest 3 right controller A button via \texttt{OVRInput.Get(OVRInput.Button.One, OVRInput.Controller.RTouch)}
\end{itemize}

\subsubsection*{State Machine} Table~\ref{tab:recording_states} documents the recording state transitions.

\begin{table}[h]
\centering
\caption{Recording State Machine}
\label{tab:recording_states}
\begin{tabular}{lll}
\hline
\textbf{State} & \textbf{Trigger} & \textbf{Next State} \\
\hline
Idle & Button press & Recording \\
Recording & Button release & Processing \\
Processing & Transcription complete & Idle \\
\hline
\end{tabular}
\end{table}

During the Recording state, a red visual indicator displays ``Listening...'' to provide user feedback.

\subsection{Audio Capture and Conversion}

The \texttt{StartRecording()} method initiates Unity's \texttt{Microphone.Start()} with the specifications in Table~\ref{tab:stt_specs}. When the user releases the button, \texttt{StopRecordingAndTranscribe()} processes the captured audio.

\subsubsection*{WAV Conversion Algorithm} The \texttt{ConvertAudioClipToWav()} method (lines 322--368) converts Unity's \texttt{AudioClip} format to WAV for API transmission:

\begin{enumerate}
\item Extract float samples from \texttt{AudioClip.GetData()}
\item Convert float [-1.0, 1.0] to 16-bit signed integer [-32768, 32767]
\item Construct RIFF WAV header (44 bytes):
\begin{itemize}
    \item Chunk ID: ``RIFF''
    \item Format: ``WAVE''
    \item Subchunk 1: ``fmt '' (audio format specification)
    \item Subchunk 2: ``data'' (PCM samples)
\end{itemize}
\item Concatenate header + PCM data
\end{enumerate}

\subsection{API Request Structure}

The WAV bytes transmit to ElevenLabs via \texttt{WWWForm} multipart HTTP POST:

\begin{verbatim}
POST https://api.elevenlabs.io/v1/speech-to-text
Content-Type: multipart/form-data
Headers: xi-api-key: [API_KEY]

Body:
  - file: recording.wav (binary WAV data)
  - model_id: "scribe_v1"
\end{verbatim}

\subsection{Response Parsing}

The API returns JSON containing:

\begin{itemize}
\item \textbf{text}: Transcribed text string
\item \textbf{confidence}: Recognition confidence score [0.0--1.0]
\end{itemize}

The transcribed text feeds directly into the agent's \texttt{ProcessUserInput()} method for intent interpretation and tool selection.

\section{Text-to-Speech Pipeline}
\label{app:voice_tts}

Agent text responses convert to speech through ElevenLabs' text-to-speech API. Table~\ref{tab:tts_specs} documents the synthesis configuration.

\begin{table}[h]
\centering
\caption{Text-to-Speech Synthesis Parameters}
\label{tab:tts_specs}
\begin{tabular}{ll}
\hline
\textbf{Parameter} & \textbf{Value} \\
\hline
Model & \texttt{eleven\_flash\_v2\_5} \\
Stability & 0.7 (voice consistency) \\
Similarity boost & 0.8 (voice clarity) \\
Speed & 1.0 (normal playback) \\
Audio format (received) & MP3 \\
API endpoint & \texttt{/text-to-speech/\{voiceId\}} \\
Synthesis latency & 1--3 seconds (typical) \\
\hline
\end{tabular}
\end{table}

\subsection{API Request Structure}

The \texttt{TextToSpeechAsync()} method (lines 31--129) sends synthesis requests:

\begin{verbatim}
POST https://api.elevenlabs.io/v1/text-to-speech/{voiceId}
Content-Type: application/json
Headers: xi-api-key: [API_KEY]

Body:
{
  "text": "Agent response text here",
  "model_id": "eleven_flash_v2_5",
  "voice_settings": {
    "stability": 0.7,
    "similarity_boost": 0.8,
    "speed": 1.0
  }
}
\end{verbatim}

\subsection{MP3 Decoding and Playback}

The API returns MP3-encoded audio via HTTP response body. The \texttt{ConvertMp3ToAudioClipAsync()} method (lines 135--154) performs decoding:

\begin{enumerate}
\item Write MP3 bytes to temporary file in \texttt{Application.temporaryCachePath}
\item Load via \texttt{UnityWebRequestMultimedia.GetAudioClip(uri, AudioType.MPEG)}
\item Enable streaming mode for memory efficiency
\item Extract \texttt{AudioClip} from request
\item Delete temporary file in \texttt{finally} block
\end{enumerate}

The resulting \texttt{AudioClip} plays through Unity's \texttt{AudioSource} component attached to the camera, utilizing Quest 3's spatial audio capabilities for immersive voice delivery positioned at the user's head location.

\subsection{Model Selection Rationale}

The \texttt{eleven\_flash\_v2\_5} model balances:

\begin{itemize}
\item \textbf{Synthesis speed}: 1--3 seconds for typical 2--3 sentence responses (critical for real-time interaction)
\item \textbf{Voice fidelity}: Natural prosody and intonation
\item \textbf{API cost}: Flash models optimize for speed over maximum quality
\end{itemize}

\section{Character Voice Management}
\label{app:voice_characters}

Each agent character uses a distinct ElevenLabs voice ID configured in \texttt{MissionConfig.specialistVoice} ScriptableObject references. Table~\ref{tab:voice_ids} documents character voice assignments.

\begin{table}[h]
\centering
\caption{Character Voice ID Assignments}
\label{tab:voice_ids}
\begin{tabular}{llp{6cm}}
\hline
\textbf{Character} & \textbf{Voice ID} & \textbf{Characteristics} \\
\hline
Mission Control & \texttt{NOpBlnGInO9m6vDvFkFC} & Authoritative, encouraging, professional \\
Anastasia (ISS Specialist) & \texttt{ZF6FPAbjXT4488VcRRnw} & Professional engineer - clear, technical, friendly \\
Dr. Harrison (Hubble Specialist) & \texttt{M4zkunnpRihDKTNF0D7f} & Veteran aerospace engineer - technical, experienced, proud of Hubble's legacy \\
Karl (Voyager Specialist) & \texttt{t1oG32lG6Z6edP2XJLiz} & Philosophical scientist and cosmic poet - contemplative, poetic, awe-inspiring \\
\hline
\end{tabular}
\end{table}

\subsection{Scene-Specific Voice Switching}

When users invoke the \texttt{route\_to\_mission} tool, the scene transition loads mission-specific \texttt{ElevenLabsSettings} assets that override the default voice ID. This ensures:

\begin{itemize}
\item Hub agent responses use Mission Control voice
\item ISS Mission Space responses use Anastasia's voice profile
\item Each specialist maintains consistent vocal identity
\end{itemize}

Voice synthesis parameters (stability, similarity boost, speed) remain constant across all characters to maintain audio quality consistency, while the underlying voice models provide tonal and character differentiation.

\subsection{Voice Settings Persistence}

The \texttt{ElevenLabsClient} caches the current \texttt{ElevenLabsSettings} reference. Scene transitions update this reference automatically through Unity's scene loading hooks, enabling seamless character voice switching without code changes in the agent logic.

\section{Error Handling and Fallbacks}
\label{app:voice_errors}

The voice pipeline implements robust error handling for network failures and API timeouts:

\subsubsection*{Speech-to-Text Errors}

\begin{itemize}
\item \textbf{Microphone unavailable}: Display error message, fall back to text input
\item \textbf{API timeout}: Retry once with exponential backoff, then show error
\item \textbf{Low confidence score}: Accept transcription but log warning
\end{itemize}

\subsubsection*{Text-to-Speech Errors}

\begin{itemize}
\item \textbf{API timeout}: Display text response without audio
\item \textbf{MP3 decode failure}: Log error, display text fallback
\item \textbf{AudioSource unavailable}: Silent failure, text remains visible
\end{itemize}

All errors log to Unity console with structured error messages for debugging while maintaining graceful degradation of user experience.

\section{Performance Optimization}
\label{app:voice_performance}

\subsection{Memory Management}

\begin{itemize}
\item Temporary WAV/MP3 files deleted immediately after use
\item \texttt{AudioClip} instances released when playback completes
\item Streaming mode for MP3 decoding reduces peak memory usage
\item No audio caching (prioritizes memory over latency)
\end{itemize}

\subsection{Latency Budget}

Table~\ref{tab:voice_latency} documents typical latency components for the complete voice interaction cycle.

\begin{table}[h]
\centering
\caption{Voice Interaction Latency Budget}
\label{tab:voice_latency}
\begin{tabular}{lr}
\hline
\textbf{Component} & \textbf{Latency} \\
\hline
User speech duration & Variable (user-controlled) \\
WAV conversion & < 100 ms \\
STT API request & 500--1500 ms \\
Agent processing (GPT-4.1) & 1000--3000 ms \\
TTS API request & 1000--3000 ms \\
MP3 decode & < 200 ms \\
Audio playback start & < 50 ms \\
\hline
\textbf{Total (excluding user speech)} & \textbf{2.5--7.8 seconds} \\
\hline
\end{tabular}
\end{table}

The 2.5--7.8 second response time falls within acceptable bounds for educational conversational interfaces, where thoughtful responses outweigh instantaneous feedback.

\chapter{VR Deployment Configuration}
\label{app:vr_implementation}

This appendix provides detailed technical specifications for the Meta Quest 3 virtual reality deployment described in Section~\ref{subsec:vr_implementation}, implementing the spatial learning and immersive presence principles outlined in Section~\ref{sec:ar_vr_review}. All build settings, input mappings, rendering configurations, and scene architecture details are verified against the Unity project configuration files.

\section{Quest 3 Android Build Configuration}
\label{app:vr_android}

The application deploys to Meta Quest 3 through Unity's Android build pipeline with OpenXR integration. Table~\ref{tab:android_build} documents the core build settings from \texttt{ProjectSettings.asset}.

\begin{table}[h]
\centering
\caption{Android Build Configuration}
\label{tab:android_build}
\begin{tabular}{ll}
\hline
\textbf{Setting} & \textbf{Value} \\
\hline
Unity Version & 6000.0.47f1 \\
Meta XR SDK & 78.0.0 \\
Minimum SDK Version & 32 (Android 12L) \\
Target SDK Version & 32 (Android 12L) \\
Target Architecture & ARMv7 (value: 2) \\
Graphics API & OpenGL ES 3.0 \\
XR Plugin & OVRPlugin (Oculus SDK) \\
Stereo Rendering Mode & Single Pass Instanced (value: 2) \\
Target Device & Meta Quest 3 \\
\hline
\end{tabular}
\end{table}

\subsection{SDK Version Rationale}

Android API level 32 (Android 12L) enables:

\begin{itemize}
\item Quest 3's inside-out tracking system (6DOF head and controller tracking)
\item Oculus runtime features (Guardian boundary, passthrough API access)
\item Hand tracking capabilities (though not actively used in this application)
\item Performance optimizations for Snapdragon XR2 Gen 2 processor
\end{itemize}

\subsection{Stereo Rendering Pipeline}

Single-pass instanced rendering (value 2 in \texttt{ProjectSettings.asset} line 49) reduces CPU overhead by rendering both eye views in a single draw call. This technique:

\begin{itemize}
\item Halves per-frame CPU work compared to multi-pass rendering
\item Maintains Quest 3's 90 Hz refresh rate target
\item Reduces GPU state changes and draw call overhead
\item Critical for mobile VR performance on battery-powered hardware
\end{itemize}

\subsection{Build Index Scene Configuration}

Table~\ref{tab:scene_build} documents the scene inclusion from \texttt{EditorBuildSettings.asset}.

\begin{table}[h]
\centering
\caption{Scene Build Index Configuration}
\label{tab:scene_build}
\begin{tabular}{rlr}
\hline
\textbf{Index} & \textbf{Scene Name} & \textbf{File Size} \\
\hline
0 & Hub.unity & 85 KB \\
1 & ISS.unity & 65 KB \\
2 & Hubble.unity & 66 KB \\
3 & Voyager.unity & 62 KB \\
4 & ARHub.unity (experimental) & 61 KB \\
\hline
\end{tabular}
\end{table}

Scene index 0 (Hub) loads at application startup. Scene transitions occur through \texttt{SceneManager.LoadSceneAsync()} with scene names or indices.

\section{Input System Implementation}
\label{app:vr_input}

Controller input integrates Oculus Touch controllers through the OVR Input API. Table~\ref{tab:controller_mapping} documents the input bindings used in the application.

\begin{table}[h]
\centering
\caption{Controller Input Mapping}
\label{tab:controller_mapping}
\begin{tabular}{lll}
\hline
\textbf{Action} & \textbf{Desktop} & \textbf{Quest 3 VR} \\
\hline
Push-to-talk (voice) & Space key & Right controller A button \\
Confirm/Select & Enter key & Right controller trigger \\
Cancel/Back & Escape key & Left controller B button \\
\hline
\end{tabular}
\end{table}

\subsection{Push-to-Talk Implementation}

The \texttt{PromptConsole.Update()} method (line 283) detects the right controller's A button through OVR Input API:

\begin{verbatim}
bool aButtonPressed = OVRInput.Get(
    OVRInput.Button.One,
    OVRInput.Controller.RTouch
);
\end{verbatim}

\subsubsection*{State Debouncing} The system tracks previous button state (\texttt{\_previousAButtonState}) to detect rising edge transitions, preventing accidental double-triggers from single button presses. This ensures one recording session per button press/release cycle.

\subsection{Desktop Testing Mode}

Desktop mode falls back to keyboard input through Unity's legacy Input system:

\begin{verbatim}
bool spacePressed = Input.GetKeyDown(KeyCode.Space);
\end{verbatim}

This enables development iteration without VR hardware, maintaining identical functionality across desktop testing and Quest 3 deployment.

\subsection{VR Mode Detection}

The \texttt{StaticVRCameraAligner} class (89 lines) detects VR mode at startup through:

\begin{verbatim}
bool isVR = XRSettings.isDeviceActive;
\end{verbatim}

When \texttt{isDeviceActive} returns \texttt{true}, the system locates the \texttt{OVRCameraRig} component via \texttt{FindObjectOfType<OVRCameraRig>()} and configures VR-specific camera settings.

\section{Camera and Rendering Configuration}
\label{app:vr_camera}

\subsection{OVRCameraRig Structure}

The Quest 3 camera system follows Oculus SDK conventions. Table~\ref{tab:camera_hierarchy} documents the camera hierarchy.

\begin{table}[h]
\centering
\caption{VR Camera Hierarchy}
\label{tab:camera_hierarchy}
\begin{tabular}{ll}
\hline
\textbf{GameObject} & \textbf{Purpose} \\
\hline
OVRCameraRig & Root container for VR camera system \\
TrackingSpace & Offset container for room-scale tracking \\
CenterEyeAnchor & Head-tracked camera position (stereo) \\
LeftEyeAnchor & Left eye render camera \\
RightEyeAnchor & Right eye render camera \\
LeftHandAnchor & Left controller tracking \\
RightHandAnchor & Right controller tracking \\
\hline
\end{tabular}
\end{table}

\subsection{Near Clip Plane Configuration}

The \texttt{StaticVRCameraAligner} configures the near clip plane to prevent geometry clipping at close range (line 68):

\begin{verbatim}
cam.nearClipPlane = 0.01f; // Unity units
\end{verbatim}

This 0.01 Unity unit near clip (approximately 1.27 cm in physical space with scale compression factor $k = 0.000785$) ensures UI elements positioned within arm's reach remain visible without clipping.

\subsection{Desktop Camera Alignment}

Desktop mode aligns the fallback camera to match VR positioning conventions, ensuring consistent coordinate systems between development and deployment environments. This allows testing of UI positioning and scene layout without VR hardware.

\section{Spatial UI Implementation}
\label{app:vr_ui}

User interface elements render in 3D world space rather than screen overlay to ensure VR readability and depth perception. Table~\ref{tab:ui_rendering} documents the UI rendering configuration.

\begin{table}[h]
\centering
\caption{Spatial UI Rendering Configuration}
\label{tab:ui_rendering}
\begin{tabular}{ll}
\hline
\textbf{Parameter} & \textbf{Value} \\
\hline
Canvas render mode & WorldSpace \\
Canvas size (transition overlay) & 2m × 2m \\
Canvas distance from camera & 1 meter (dynamic) \\
Mission logo size & 512×512 pixels \\
Text component & TextMeshPro \\
Background opacity (UI panels) & 0.7 alpha \\
Sort order (transition canvas) & 100 (renders on top) \\
\hline
\end{tabular}
\end{table}

\subsection{MissionClockUI Pattern}

The \texttt{MissionClockUI} class (74 lines) demonstrates the spatial UI pattern:

\begin{enumerate}
\item Canvas component with \texttt{RenderMode.WorldSpace}
\item TextMeshPro text field positioned in 3D environment
\item CanvasGroup component (line 37) controls opacity without render-to-texture overhead
\item Displays mission elapsed time and simulation speed multiplier
\end{enumerate}

\subsection{Transition Overlay System}

The \texttt{SceneTransitionManager.CreateTransitionUIIfNeeded()} method (lines 729--838) constructs the transition overlay procedurally:

\subsubsection*{Canvas Construction}

\begin{verbatim}
Canvas canvas = canvasObj.AddComponent<Canvas>();
canvas.renderMode = RenderMode.WorldSpace;
canvas.sortingOrder = 100;

RectTransform canvasRect = canvasObj.GetComponent<RectTransform>();
canvasRect.sizeDelta = new Vector2(2f, 2f); // 2m × 2m
\end{verbatim}

\subsubsection*{Dynamic Positioning} The \texttt{LateUpdate()} method (lines 154--180) repositions the canvas 1 meter in front of the camera each frame:

\begin{verbatim}
transitionCanvasTransform.position =
    cachedCameraAnchor.position +
    cachedCameraAnchor.forward * 1f;

transitionCanvasTransform.rotation =
    cachedCameraAnchor.rotation;
\end{verbatim}

This dynamic positioning ensures the overlay remains visible during scene transitions when camera references change, avoiding parenting to scene-specific GameObjects that would be destroyed during \texttt{SceneManager.LoadSceneAsync()}.

\section{Scene Architecture and Persistence}
\label{app:vr_scenes}

The application comprises four navigable scenes sharing common systems through persistent singletons. Table~\ref{tab:scene_components} documents shared components across all scenes.

\begin{table}[h]
\centering
\caption{Common Scene Components}
\label{tab:scene_components}
\begin{tabular}{ll}
\hline
\textbf{Component} & \textbf{Purpose} \\
\hline
OVRCameraRig prefab & VR camera and controller tracking \\
PromptConsole GameObject & Conversational UI and voice input \\
TimeController & Simulation speed management \\
OrbitController & Orbital physics (Hub only) \\
OrbitVisualizer & Trajectory rendering (Hub only) \\
\hline
\end{tabular}
\end{table}

\subsection{Singleton Persistence Mechanism}

The \texttt{SceneTransitionManager} enforces singleton persistence through Unity's \texttt{DontDestroyOnLoad()} mechanism (line 58):

\begin{verbatim}
if (Instance == null) {
    Instance = this;
    DontDestroyOnLoad(gameObject);
}
\end{verbatim}

This ensures the transition UI and conversation context survive scene unloading. Similarly, \texttt{ConversationHistory} persists across transitions, preserving the 10-exchange dialogue window.

\subsection{Asynchronous Scene Loading}

Scene loading occurs through \texttt{SceneManager.LoadSceneAsync()} with deferred activation (line 252):

\begin{verbatim}
AsyncOperation loadOperation =
    SceneManager.LoadSceneAsync(sceneName);
loadOperation.allowSceneActivation = false;

// Load scene in background...

// After 4-second logo animation:
loadOperation.allowSceneActivation = true;
\end{verbatim}

This deferred activation prevents jarring scene pops, allowing smooth fade-out → logo display → scene activation → fade-in transitions.

\section{Performance Optimization}
\label{app:vr_performance}

Performance optimization targets Quest 3's mobile GPU constraints. Table~\ref{tab:performance_targets} documents the performance budget.

\begin{table}[h]
\centering
\caption{Performance Targets for 90 Hz VR}
\label{tab:performance_targets}
\begin{tabular}{lr}
\hline
\textbf{Metric} & \textbf{Target} \\
\hline
Frame time budget & 11.1 ms (90 Hz) \\
Target resolution (per eye) & 1832×1920 pixels \\
Draw calls (Hub scene) & < 100 per frame \\
Texture memory budget & < 512 MB \\
Polygon count (visible) & < 100k triangles \\
\hline
\end{tabular}
\end{table}

\subsection{Rendering Optimizations}

\begin{itemize}
\item \textbf{Shared material instances}: Reduce draw calls by batching geometry with identical materials
\item \textbf{Texture compression}: ASTC 6×6 for UI elements, ASTC 4×4 for environment textures
\item \textbf{Single-pass instanced stereo}: Halves per-frame CPU work (both eyes in one draw call)
\item \textbf{Occlusion culling}: Disabled (scenes are spatially compact, overhead exceeds benefit)
\item \textbf{Dynamic batching}: Enabled for small meshes (< 300 vertices)
\end{itemize}

\subsection{Memory Management}

\begin{itemize}
\item Scene file sizes optimized (62--85 KB per scene)
\item Texture atlasing for UI sprites
\item Audio streaming for voice synthesis (no large audio caching)
\item Persistent GameObjects minimized (only transition manager, conversation history)
\end{itemize}

\subsection{Frame Time Breakdown}

Table~\ref{tab:frame_budget} shows typical frame time allocation in the Hub scene (most complex).

\begin{table}[h]
\centering
\caption{Frame Time Budget Breakdown (Hub Scene)}
\label{tab:frame_budget}
\begin{tabular}{lr}
\hline
\textbf{Component} & \textbf{Time} \\
\hline
Physics simulation & 1.2 ms \\
Script execution & 2.1 ms \\
Rendering (both eyes) & 5.8 ms \\
VR compositor overhead & 1.5 ms \\
Buffer margin & 0.5 ms \\
\hline
\textbf{Total} & \textbf{11.1 ms} \\
\hline
\end{tabular}
\end{table}

This allocation maintains the 11.1 ms frame budget required for consistent 90 Hz VR without reprojection artifacts (judder).


% Anexos
\annex

% Glossario
%\itaglossary
%\printglossary

% Folha de Registro do Documento
% Valores dos campos do formulario
\FRDitadata{25 de março de 2015}
\FRDitadocnro{DCTA/ITA/DM-018/2015} %(o número de registro você solicita a biblioteca)
\FRDitaorgaointerno{Instituto Tecnológico de Aeronáutica -- ITA}
%Exemplo no caso de pós-graduação: Instituto Tecnol{\'o}gico de Aeron{\'a}utica -- ITA
\FRDitapalavrasautor{AI; VR}
\FRDitapalavrasresult{AI; VR}
%Exemplo no caso de graduação (TG):
%\FRDitapalavraapresentacao{Trabalho de Graduação, ITA, São José dos Campos, 2015. \NumPenultimaPagina\ páginas.}
%Exemplo no caso de pós-graduação (msc, dsc):
\FRDitapalavraapresentacao{ITA, São José dos Campos. Curso de Mestrado. Programa de Pós-Graduação em Engenharia Aeronáutica e Mecânica. Área de Sistemas Aeroespaciais e Mecatrônica. Orientador: Prof.~Dr. Adalberto Santos Dupont. Coorientadora: Prof$^\textnormal{a}$.~Dr$^\textnormal{a}$. Doralice Serra. Defesa em 05/03/2015. Publicada em 25/03/2015.}
\FRDitaresumo{Métodos educacionais tradicionais frequentemente encontram dificuldades para transmitir conceitos complexos, espaciais e dinâmicos como os da mecânica orbital. A recente convergência entre a Realidade Mista (RM) de consumo e os sofisticados agentes de Inteligência Artificial (IA) Generativa apresenta uma oportunidade para criar um novo paradigma de interfaces de aprendizagem intuitivas e experienciais. Este trabalho detalha o projeto, desenvolvimento e demonstração de uma plataforma educacional interativa para a exploração dos princípios da mecânica orbital. O objetivo principal do sistema é conectar a teoria de leis físicas abstratas à compreensão intuitiva, permitindo que os usuários aprendam por meio da interação corporificada. A metodologia é centrada em uma arquitetura modular que integra dois componentes principais: (1) um "cérebro" agente generativo, impulsionado por Modelos de Linguagem Abrangentes, que interpreta comandos em linguagem natural e atua como um guia educacional especializado; e (2) um "mundo" de simulação em tempo real e visualização imersiva em realidade mista, construído no motor Unity para o Meta Quest 3, que renderiza trajetórias orbitais fisicamente precisas em um espaço tridimensional onde os usuários vivenciam a mecânica orbital de dentro. A plataforma facilita um ciclo de interação multimodal contínuo, onde os comandos de voz do usuário são capturados, processados pelo agente para alterar os parâmetros da simulação e refletidos na visualização imersiva em RV com feedback auditivo conversacional. Este trabalho entrega um protótipo funcional que demonstra uma nova abordagem para a educação científica, transformando dados abstratos em uma experiência manipulável e conversacional para promover uma aprendizagem exploratória e profundamente engajadora. A plataforma é disponibilizada como software de código aberto para permitir validação, adaptação e extensão pela comunidade para diversos contextos educacionais.
}
%  Primeiro Parametro: Nacional ou Internacional -- N/I
%  Segundo parametro: Ostensivo, Reservado, Confidencial ou Secreto -- O/R/C/S
\FRDitaOpcoes{N}{O}
% Cria o formulario
\itaFRD

\end{document}
% Fim do Documento. O massacre acabou!!! :-)
