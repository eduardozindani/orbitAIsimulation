Métodos educacionais tradicionais frequentemente encontram dificuldades para transmitir conceitos complexos, espaciais e dinâmicos como os da mecânica orbital. A recente convergência entre a Realidade Aumentada (RA) de consumo e os sofisticados agentes de Inteligência Artificial (IA) Generativa apresenta uma oportunidade para criar um novo paradigma de interfaces de aprendizagem intuitivas e experienciais. Este trabalho detalha o projeto, desenvolvimento e avaliação de uma plataforma educacional interativa para a exploração dos princípios da mecânica orbital. O objetivo principal do sistema é conectar a teoria de leis físicas abstratas à compreensão intuitiva, permitindo que os usuários aprendam por meio da interação corporificada. A metodologia é centrada em uma arquitetura modular que integra três componentes principais: (1) um "cérebro" agente generativo, impulsionado por Modelos de Linguagem Abrangentes, que interpreta comandos em linguagem natural e atua como um guia educacional especializado; (2) um "mundo" de simulação em tempo real e visualização em RA, construído no motor Unity para o Meta Quest 3, que renderiza trajetórias orbitais fisicamente precisas e ancoradas no ambiente do usuário ; e (3) uma interface "corpo" corporificada, consistindo em um globo físico com um sensor rotacional, que permite o controle tangível da simulação. A plataforma facilita um ciclo de interação multimodal contínuo, onde os comandos de voz e as manipulações físicas do usuário são capturados, processados pelo agente para alterar os parâmetros da simulação e refletidos na visualização em RA com feedback auditivo conversacional. Este trabalho entrega um protótipo funcional que demonstra uma nova abordagem para a educação científica, transformando dados abstratos em uma experiência manipulável e conversacional para promover uma aprendizagem exploratória e profundamente engajadora.
